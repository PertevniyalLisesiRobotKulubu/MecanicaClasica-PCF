\documentclass[a4paper,10pt]{article}
\usepackage[utf8]{inputenc}
\usepackage[spanish]{babel}
\usepackage[affil-it]{authblk}
\usepackage{enumerate}
\usepackage{graphicx}
\usepackage{hyperref}
\usepackage{amsmath}
\usepackage{amssymb}
\usepackage{cancel}
\usepackage[usenames, dvipsnames]{color}
\usepackage{tikz}
\usepackage[labelfont=bf]{caption}
\usepackage{subcaption} %Multiple images
\usepackage{multicol} % Multiple columns
\usepackage{float}
\usepackage{cleveref}
 \usepackage{relsize} % bigger math symbols
\usepackage[margin=1.4in]{geometry}
\usepackage[titletoc,toc,title]{appendix}
\usepackage{enumitem}
\usepackage{etoolbox}
\usetikzlibrary{calc}
\numberwithin{equation}{section}

% Circled words
\newcommand{\circled}[2][]{%
  \tikz[baseline=(char.base)]{%
    \node[shape = circle, draw, inner sep = 1pt]
    (char) {\phantom{\ifblank{#1}{#2}{#1}}};%
    \node at (char.center) {\makebox[0pt][c]{#2}};}}
\robustify{\circled}

%Appendices in spanish
\renewcommand{\appendixname}{Ap\'endices}
\renewcommand{\appendixtocname}{Ap\'endices}
\renewcommand{\appendixpagename}{Ap\'endices}

%Zero delimiter
\newcommand{\zerodel}{.\kern-\nulldelimiterspace}

%Columns separation
\setlength{\columnsep}{1cm}

%Indentation
\setlength{\parindent}{0ex}

%Multiple References

\crefrangelabelformat{equation}{(#3#1#4--#5\crefstripprefix{#1}{#2}#6)}

\usepackage{xparse}

%Boxes

\newcommand*{\boxcolor}{blue}
\makeatletter
\renewcommand{\boxed}[1]{\textcolor{\boxcolor}{%
\tikz[baseline={([yshift=-1ex]current bounding box.center)}] \node [rectangle, minimum width=1ex,rounded corners,draw] {\normalcolor\m@th$\displaystyle#1$};}}
 \makeatother

%Constantes
\newcommand{\euler}{\mathrm{e}}
\newcommand{\im}{i}

%Lemas, teoremas, definiciones y pruebas
\newcommand{\definicion}{\textbf{Definición: }}
\newcommand{\lema}{\textbf{Lema: }}
\newcommand{\teorema}{\textbf{Teorema: }}
\newcommand{\prueba}{\textbf{Prueba: }}
\newcommand{\proposicion}{\textbf{Proposición: }}
\newcommand{\corolario}{\textbf{Corolario: }}

% Definición de las secciones y su numeración

\makeatletter
\def\@seccntformat#1{%
  \expandafter\ifx\csname c@#1\endcsname\c@section\else
  \csname the#1\endcsname\quad
  \fi}
\makeatother

%opening
\title{Mecánica Clásica Tarea \# 14}
\author{Favio Vázquez\thanks{Correo: favio.vazquezp@gmail.com}}\affil{Instituto de Ciencias Nucleares. Universidad Nacional Autónoma de México.}
\date{}

\begin{document}

\makeatletter
\def\@maketitle{%
  \newpage
  \null
  \vskip 2em%
  \begin{center}%
  \let \footnote \thanks
    {\Large\bfseries \@title \par}%
    \vskip 1.5em%
    {\normalsize
      \lineskip .5em%
      \begin{tabular}[t]{c}%
        \@author
      \end{tabular}\par}%
    \vskip 1em%
    {\normalsize \@date}%
  \end{center}%
  \par
  \vskip 1.5em}
\makeatother

\maketitle

\section{Problema 1}

Utilizando únicamente la ecuación de la eikonal deduzca la ley de Snell.

\vspace{.3cm}

\underline{Solución:} \vspace{.3cm}

Recordemos la expresión para la ecuación de la eikonal,

\begin{equation}
 |\nabla S|^2(\mathbf{r}) = n^2(\mathbf{r}).
\end{equation}

Esta ecuación es simplemente el módulo de la ecuación de Huygens que puede 
escribirse como 

\begin{equation}
 \nabla S = n(\mathbf{r})\hat{\mathbf{s}},
\end{equation}

y también sabemos que el gradiente de esta ecuación define el vector de rayo de 
magnitud $|\mathbf{n}| = n$. Si ahora recordamos que la integración del gradiente 
sobre un camino cerrado se hace cero, tenemos que 

\begin{equation}
\oint_P \nabla S(\mathbf{r}) \cdot d\mathbf{r} = \oint_P \mathbf{n}(\mathbf{r}) 
\cdot d\mathbf{r} = 0.
\end{equation}

Consideremos ahora el caso en que el camino cerrado $P$ rodea una frontera que 
separa dos medios diferentes. Si hacemos que los lados del bucle perpendicular 
a la interfaz vayan a cero, entonces únicamente las partes de la integral de línea 
tangenciales al camino de la interfaz contribuirán en la misma. Ahora debido a que 
estas contribuciones deben sumar cero, las componentes tangenciales de los vectores 
de rato deben preservarse, esto es 

\begin{equation}
 (\mathbf{n} - \mathbf{n}') \times \hat{\mathbf{z}} = 0,
 \label{eq:preservacionTangencial}
\end{equation}

donde el primo se refiere al lado de la frontera al cual el rayo es transmitido, 
cuyo vector normal es $\hat{\mathbf{z}}$. Ahora imaginemos a un rayo atravesando 
la frontera y pasando a través de la región encerrada por el bucle de integración. 
Si $\theta$ y $\theta'$ son los ángulos de incidencia y transmisión, respectivamente, 
medidos desde la normal $\hat{\mathbf{z}}$ a través de la frontera, entonces la 
preservación de la componente tangencial del vector de rayo significa que, 
tomando en cuenta \eqref{eq:preservacionTangencial} y la definición del 
producto vectorial,

\begin{equation}
 \mathbf{n} \times \hat{\mathbf{z}} - \mathbf{n}' \times \hat{\mathbf{z}} = 0,
\end{equation}

\begin{equation}
 n\sen{\theta} - n'\sen{\theta'} = 0,
\end{equation}

esto debido a que $|\hat{\mathbf{z}}| = 1$, por lo tanto 

\begin{equation}
 \boxed{n\sen{\theta} = n'\sen{\theta'}.}
\end{equation}

Que es la ley de la refracción de Snell, descubierta primero por Ibn Sahl en 984 \cite{rashed}, 
y luego por Willebrod Snellius en 1621 \cite{holm}. Un análisis similar puede aplicarse al 
caso de los rayos reflejaos para mostrar que el ángulo de incidencia debe ser igual 
al ángulo de reflexión.

\section{Problema 2}

En el curso se mostró que la ecuación de la eikonal es una aproximación de onda 
pequeña de la ecuación de ondas. Encuentre ahora, a partir únicamente del principio 
de Fermat, las ecuaciones diferenciales que determinan los rayos de luz y la ecuación 
de la eikonal. Comente sobre la situación análoga entre las ecuaciones de movimiento 
de la mecánica y el principio de Hamilton.

\vspace{.3cm}

\underline{Solución:} \vspace{.3cm}

\section{Problema 3}

En el curso se demostró que la función principal de Hamilton es una solución completa 
de la ecuación de Hamilton-Jacobi correspondiente. ¿Será cierto el enunciado inverso 
de este, esto es, que una solución completa de la ecuación de Hamilton-Jacobi se puede 
ver como la función principal de Hamilton? Argumente su respuesta.

\vspace{.3cm}

\underline{Solución:} \vspace{.3cm}

\section{Problema 4}

Demuestre que un sistema es integrable sí y solo sí existen sistemas de coordenadas 
canónicas en las que la ecuación de Hamilton-Jacobi es totalmente separable.

\vspace{.3cm}

\underline{Solución:} \vspace{.3cm}

\section{Problema 5}

Reduzca a cuadraturas por el método de Liouville el ejemplo del péndulo esférico.

\vspace{.3cm}

\underline{Solución:} \vspace{.3cm}

\begin{thebibliography}{10}
\bibitem{rashed}
R. Rashed, \emph{“Géométrie et dioptrique au Xe siècle: Ibn Sahl, 
Al-Quhi et Ibn Al-Haytham}, Las Belles Lettres, 1993.
\bibitem{holm}
D. Holm, \emph{Geometric Mechanics, Part I: Dynamics and Symmetry}, World Scientific, 
Imperial College Press, 2008.
\end{thebibliography}

\end{document}