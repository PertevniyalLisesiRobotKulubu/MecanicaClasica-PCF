\documentclass[a4paper,10pt]{article}
\usepackage[utf8]{inputenc}
\usepackage[spanish]{babel}
\usepackage[affil-it]{authblk}
\usepackage{enumerate}
\usepackage{graphicx}
\usepackage{hyperref}
\usepackage{amsmath}
\usepackage{amssymb}
\usepackage{cancel}
\usepackage[usenames, dvipsnames]{color}
\usepackage{tikz}
\usepackage{multimedia}
\usepackage{subcaption} %Multiple images
\usepackage{multicol} % Multiple columns
\usepackage{float}
\usepackage{cleveref}
\usepackage[margin=1.4in]{geometry}
\usepackage[labelfont=bf]{caption}
\usetikzlibrary{calc}
\numberwithin{equation}{section}

%Columns separation
\setlength{\columnsep}{1cm}

%Indentation
\setlength{\parindent}{0ex}

%Multiple References

\usepackage{xparse}
\ExplSyntaxOn
\NewDocumentCommand{\mref}{m}{\quinn_mref:n {#1}}
\seq_new:N \l_quinn_mref_seq
\cs_new:Npn \quinn_mref:n #1
 {
  \seq_set_split:Nnn \l_quinn_mref_seq { , } { #1 }
  \seq_pop_right:NN \l_quinn_mref_seq \l_tmpa_tl
  ( % print the left parenthesis
  \seq_map_inline:Nn \l_quinn_mref_seq
    { \ref{##1},\nobreakspace } % print the first references
  \exp_args:NV \ref \l_tmpa_tl 
  ) 
 }
\ExplSyntaxOff


%Boxes

\newcommand*{\boxcolor}{blue}
\makeatletter
\renewcommand{\boxed}[1]{\textcolor{\boxcolor}{%
\tikz[baseline={([yshift=-1ex]current bounding box.center)}] \node [rectangle, minimum width=1ex,rounded corners,draw] {\normalcolor\m@th$\displaystyle#1$};}}
 \makeatother

%Constantes
\newcommand{\euler}{\mathrm{e}}
\newcommand{\im}{i}

%Lemas, teoremas, definiciones y pruebas
\newcommand{\definicion}{\textbf{Definición: }}
\newcommand{\lema}{\textbf{Lema: }}
\newcommand{\teorema}{\textbf{Teorema: }}
\newcommand{\prueba}{\textbf{Prueba: }}


%opening
\title{Mecánica Clásica Tarea \# 5}
\author{Favio Vázquez\thanks{Correo: favio.vazquezp@gmail.com}}\affil{Instituto de Ciencias Nucleares. Universidad Nacional Autónoma de México.}
\date{}

\begin{document}

\makeatletter
\def\@maketitle{%
  \newpage
  \null
  \vskip 2em%
  \begin{center}%
  \let \footnote \thanks
    {\Large\bfseries \@title \par}%
    \vskip 1.5em%
    {\normalsize
      \lineskip .5em%
      \begin{tabular}[t]{c}%
        \@author
      \end{tabular}\par}%
    \vskip 1em%
    {\normalsize \@date}%
  \end{center}%
  \par
  \vskip 1.5em}
\makeatother

\maketitle

\section{Problema 1}

Dos partículas, una de masa $m_1$ y otra de masa $m_2$, interaccionan gravitacionalmente 
en el espacio tridimensional. En adición a su interacción mutua se encuentran en un campo 
de fuerzas externo que les produce una aceleración constante $\mathbf{a}$en cierta 
dirección también constante. En un sistema de coordenadas generalizadas conveniente para este 
sistema, encuentre la función lagrangiana y las ecuaciones de movimiento.

\vspace{.3cm}

\underline{Solución:} \vspace{.3cm}

\section{Problema 2}

Demuestre que si la energía cinética de un sistema mecánico se puede expresar como

$$
T = \sum_i f_i(q^i)(\dot{q}^i)^2,
$$

y la energía potencial como

$$
V = \sum_i V_i(q^i),
$$

entonces podemos reducir a cuadraturas las ecuaciones de movimiento.

\vspace{.3cm}

\underline{Solución:} \vspace{.3cm}

\section{Problema 3}

Un péndulo de longitud $l$ y masa $m$ tiene su extremo fijo anclado a una masa $M$ que 
puede desplazarse libremente en la dirección horizontal. Encuentre las ecuaciones de 
movimiento. ¿Hay suficientes integrales de movimiento como para reducir el problema 
a cuadraturas? En un caso de ser así, utilizando el procedimiento gráfico para una 
dimensión, discuta el movimiento de este sistema.

\vspace{.3cm}

\underline{Solución:} \vspace{.3cm}

\section{Problema 4}

Para un número de grados de libertad mayor que uno, dos lagrangianas distintas $L_1$
y $L_2$ generan ecuaciones de movimiento idénticas, demuestre que 

$$
L_1 - L_2 = \frac{d}{dt}f(q),
$$

donde $f$ es una función de la variedad de configuración en los reales. ¿Es esto cierto 
en sentido inverso?

\vspace{.3cm}

\underline{Solución:} \vspace{.3cm}

\section{Problema 5}

En presencia de la gravedad una partícula de masa $m$, inicialmente en reposo, se 
mueve a lo largo de una cicloide dada por 

$$
x = \pm a \cos^{-1}{\left(\frac{a-y}{a}\right)} + \sqrt{2ay-y^2}
$$

Encuentre y resuelva las ecuaciones de movimiento. Demuestre que el tiempo que tarda 
la partícula en llegar a la parte más baja del cicloide es independiente de su 
posición inicial sobre la cicloide.

\vspace{.3cm}

\underline{Solución:} \vspace{.3cm}


\end{document}

