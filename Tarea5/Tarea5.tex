\documentclass[a4paper,10pt]{article}
\usepackage[utf8]{inputenc}
\usepackage[spanish]{babel}
\usepackage[affil-it]{authblk}
\usepackage{enumerate}
\usepackage{graphicx}
\usepackage{hyperref}
\usepackage{amsmath}
\usepackage{amssymb}
\usepackage{cancel}
\usepackage[usenames, dvipsnames]{color}
\usepackage{tikz}
\usepackage{multimedia}
\usepackage{subcaption} %Multiple images
\usepackage{multicol} % Multiple columns
\usepackage{float}
\usepackage{cleveref}
\usepackage[margin=1.4in]{geometry}
\usepackage[labelfont=bf]{caption}
\usetikzlibrary{calc}
\numberwithin{equation}{section}

%Columns separation
\setlength{\columnsep}{1cm}

%Indentation
\setlength{\parindent}{0ex}

%Multiple References

\usepackage{xparse}
\ExplSyntaxOn
\NewDocumentCommand{\mref}{m}{\quinn_mref:n {#1}}
\seq_new:N \l_quinn_mref_seq
\cs_new:Npn \quinn_mref:n #1
 {
  \seq_set_split:Nnn \l_quinn_mref_seq { , } { #1 }
  \seq_pop_right:NN \l_quinn_mref_seq \l_tmpa_tl
  ( % print the left parenthesis
  \seq_map_inline:Nn \l_quinn_mref_seq
    { \ref{##1},\nobreakspace } % print the first references
  \exp_args:NV \ref \l_tmpa_tl 
  ) 
 }
\ExplSyntaxOff


%Boxes

\newcommand*{\boxcolor}{blue}
\makeatletter
\renewcommand{\boxed}[1]{\textcolor{\boxcolor}{%
\tikz[baseline={([yshift=-1ex]current bounding box.center)}] \node [rectangle, minimum width=1ex,rounded corners,draw] {\normalcolor\m@th$\displaystyle#1$};}}
 \makeatother

%Constantes
\newcommand{\euler}{\mathrm{e}}
\newcommand{\im}{i}

%Lemas, teoremas, definiciones y pruebas
\newcommand{\definicion}{\textbf{Definición: }}
\newcommand{\lema}{\textbf{Lema: }}
\newcommand{\teorema}{\textbf{Teorema: }}
\newcommand{\prueba}{\textbf{Prueba: }}


%opening
\title{Mecánica Clásica Tarea \# 5}
\author{Favio Vázquez\thanks{Correo: favio.vazquezp@gmail.com}}\affil{Instituto de Ciencias Nucleares. Universidad Nacional Autónoma de México.}
\date{}

\begin{document}

\makeatletter
\def\@maketitle{%
  \newpage
  \null
  \vskip 2em%
  \begin{center}%
  \let \footnote \thanks
    {\Large\bfseries \@title \par}%
    \vskip 1.5em%
    {\normalsize
      \lineskip .5em%
      \begin{tabular}[t]{c}%
        \@author
      \end{tabular}\par}%
    \vskip 1em%
    {\normalsize \@date}%
  \end{center}%
  \par
  \vskip 1.5em}
\makeatother

\maketitle

\section{Problema 1}

Dos partículas, una de masa $m_1$ y otra de masa $m_2$, interaccionan gravitacionalmente 
en el espacio tridimensional. En adición a su interacción mutua se encuentran en un campo 
de fuerzas externo que les produce una aceleración constante $\mathbf{a}$ en cierta 
dirección también constante. En un sistema de coordenadas generalizadas conveniente para este 
sistema, encuentre la función lagrangiana y las ecuaciones de movimiento.

\vspace{.3cm}

\underline{Solución:} \vspace{.3cm}

\section{Problema 2}

Demuestre que si la energía cinética de un sistema mecánico se puede expresar como

$$
T = \sum_i f_i(q^i)(\dot{q}^i)^2,
$$

y la energía potencial como

$$
V = \sum_i V_i(q^i),
$$

entonces podemos reducir a cuadraturas las ecuaciones de movimiento.

\vspace{.3cm}

\underline{Solución:} \vspace{.3cm}

\section{Problema 3}

Un péndulo de longitud $l$ y masa $m$ tiene su extremo fijo anclado a una masa $M$ que 
puede desplazarse libremente en la dirección horizontal. Encuentre las ecuaciones de 
movimiento. ¿Hay suficientes integrales de movimiento como para reducir el problema 
a cuadraturas? En un caso de ser así, utilizando el procedimiento gráfico para una 
dimensión, discuta el movimiento de este sistema.

\vspace{.3cm}

\underline{Solución:} \vspace{.3cm}

Para tener más claro la configuración física del problema que tratamos, nos referimos 
a la figura \mref{fig:problema3fig1}.

\begin{figure}[H]
 \center
 \includegraphics[scale=0.4]{problema3fig1}
 \caption{Representación gráfica del péndulo simple que tiene en su extremo fijo 
 una masa $M$ que puede moverse libremente en la dirección horizontal. Como indica 
 el muñequito la gravedad va dirigida hacia abajo.}
 \label{fig:problema3fig1}
\end{figure}

\section{Problema 4}

Para un número de grados de libertad mayor que uno, dos lagrangianas distintas $L_1$
y $L_2$ generan ecuaciones de movimiento idénticas, demuestre que 

$$
L_1 - L_2 = \frac{d}{dt}f(q),
$$

donde $f$ es una función de la variedad de configuración en los reales. ¿Es esto cierto 
en sentido inverso?

\vspace{.3cm}

\underline{Solución:} \vspace{.3cm}

\section{Problema 5}

En presencia de la gravedad una partícula de masa $m$, inicialmente en reposo, se 
mueve a lo largo de una cicloide dada por 

$$
x = \pm a \cos^{-1}{\left(\frac{a-y}{a}\right)} + \sqrt{2ay-y^2}
$$

Encuentre y resuelva las ecuaciones de movimiento. Demuestre que el tiempo que tarda 
la partícula en llegar a la parte más baja del cicloide es independiente de su 
posición inicial sobre la cicloide.

\vspace{.3cm}

\underline{Solución:} \vspace{.3cm}

En la figura de abajo se encuentra el gráfico del cicloide en cuestión, de la ecuación 
vemos que este es válido para $y \in [0,2a]$ y en el eje $x$ lo graficamos de $[-\pi a, \pi a]$.

\begin{figure}[H]
 \center
 \includegraphics[scale=0.4]{problema5fig1}
 \caption{Gráfico del cicloide en consideración.}
 \label{fig:problema5fig1}
\end{figure}

Debido a las simetrías de este problema es claro que la coordenada generalizada que 
mejor describirá el sistema, en una manera más simple también, es la longitud de 
arco $s$ a lo largo de la línea definida por el cicloide (ver figura \mref{fig:problema5fig1}. 
En este caso, la energía cinética estará dada por 

\begin{equation}
 T = \frac{1}{2} m \dot{s}^2.
 \label{eq:energCinetCiclo1}
\end{equation}

Por otra parte, el potencial a la cual está sujeta la partícula, se puede obtener
simplemente de la fuerza de gravedad. Si imponemos que este campo gravitacional de 
magnitud $g$ está en la dirección negativa de $y$, entonces el potencial será,

\begin{equation}
 V = mgy.
 \label{eq:energPotenCiclo1}
\end{equation}

Debemos encontrar entonces la relación entre $y$ y $s$ para poder escribir la energía 
potencial en términos de nuestra coordenada generalizada. Recordando que la longitud 
de arco diferencial en coordenadas cartesianas se puede escribir como,

\begin{equation}
 s = \int \sqrt{dx^2+dy^2} = \int \sqrt{1+\left( \frac{dx}{dy} \right)^2} dy.
 \label{eq:arcoCartesiano}
\end{equation}

Ahora debemos obtener la derivada de $x$ con respecto a $y$, que hacemos explícitamente 
debajo,

\begin{align}
 \begin{split}
  %
  \frac{dx}{dy} &= \left(\pm a \cos^{-1}{\left(\frac{a-y}{a}\right)}\right)' + 
  \left(\sqrt{2ay-y^2}\right)', \\
  %
		&= a \left[ \frac{-1}{\sqrt{1 - \left(\frac{a-y}{a} \right)^2}}
		\left(\frac{a-y}{a} \right)'\right] + \frac{1}{2\sqrt{2ay-y^2}}
		(2ay - y^2)', \\
  %
		&= \cancel{a}\left[ \frac{-1}{\sqrt{1 - \left(\frac{a-y}{a} \right)^2}}
		\left(- \frac{1}{\cancel{a}} \right) \right] + \frac{1}{\cancel{2}\sqrt{2ay-y^2}}
		[\cancel{2}(a - y)], \\
  %
		&= \frac{1}{\sqrt{1 - \frac{(a-y)^2}{a^2}}} + \frac{a-y}{\sqrt{2ay - y^2}}.
  %
 \end{split}
\end{align}

Y simplificando llegamos a que 

\begin{equation}
 \frac{dx}{dy} = \frac{2a - y}{\sqrt{2ay - y^2}}
 \label{eq:dxdyCiclo}
\end{equation}

Ahora sustituyendo \mref{eq:dxdyCiclo} en \mref{eq:arcoCartesiano} nos queda 

\begin{equation}
 s = \int \sqrt{1 + \frac{(2a - y)^2}{(\sqrt{2ay - y^2})^2}} dy,
\end{equation}

que podemos reescribir como

\begin{align}
\begin{split}
 %
 s &= \int \sqrt{1 + \frac{(2a - y)^2}{2ay - y^2}} dy, \\
 % 
   &= \int \sqrt{\frac{2ay - \cancel{ y^2} + 4a^2 - 4ay + \cancel{y^2}}{2ay - y^2}} dy, \\  
 % 
   &= \int \sqrt{\frac{4a^2 -2ay}{2ay - y^2}} dy.  \\
 %
   &= \int \sqrt{2} \sqrt{\frac{a}{y}} dy.
 \end{split}
\end{align}

Evaluando esta integral obtenemos que 

\begin{equation}
 s = 2 \sqrt{2ay}.
\end{equation}

Y por lo tanto

\begin{equation}
 y = \frac{s^2}{8a}.
 \label{eq:yCiclo}
\end{equation}

Sustituyendo \mref{eq:yCiclo} en \mref{eq:energPotenCiclo1}, obtenemos 

\begin{equation}
 V = \frac{mgs^2}{8a}
\end{equation}

Entonces la lagrangiana del sistema, $L = T - V$ es 

\begin{equation}
 L = \frac{1}{2} m \dot{s}^2 - \frac{mgs^2}{8a}
\end{equation}

Y utilizando las ecuaciones de Lagrange\footnote{Tomando en cuenta que el sistema es conservativo.} 


\begin{equation}
 \frac{d}{dt}\frac{\partial L}{\partial \dot{s}} - \frac{\partial L}{\partial s} = 0,
\end{equation}

tenemos que la ecuación de movimiento del sistema será:

\begin{equation}
 \frac{d}{dt} (m\dot{s}) - \frac{mgs}{4a} = 0,
\end{equation}

\begin{equation}
 \boxed{\therefore \ddot{s} = \frac{g}{4a} s.}
\end{equation}

Claramente esta es la ecuación de un oscilador armónico, y podemos ver esto de una 
forma un poco más clara si llamamos $k = g/4a$, entonces la ecuación de movimiento 
se convierte en

\begin{equation}
 \ddot{s} - ks = 0.
\end{equation}

Cuya solución es\footnote{Esta solución ya se ha encontrado en otras tareas 
varias veces, y es muy simple obtenerla. Debido a que es una ecuación diferencial lineal
de segundo orden homogénea proponemos una solución del tipo $\euler^{\lambda t}$, obtenemos la 
ecuación característica de la EDO. Luego obtenemos las raíces $\lambda$ que sustituiremos
en la solución que hemos propuesto, haciendo uso de la linealidad de la ecuación para 
usar el principio de superposición y sumar estas soluciones. Por último hacemos la 
sustitución $A = \sqrt{C_1^2+ C_2^2}$, donde $C_1$ y $C_2$ son las constantes de 
integración de la ecuación diferencial, y con un poco de trigonometría, colocamos 
$C_1$ y $C_2$ en un triángulo rectángulo de hipotenusa $A$, y proponemos un ángulo
$\delta$ entre $A$ y $C_1$, al cual llamamos fase de la oscilación.},

\begin{equation}
 s = A\cos{(\omega t + \delta)}
\end{equation}

donde $\omega = \frac{1}{2}\sqrt{g/a}$ es la frecuencia del oscilador, $A$ es la 
amplitud de la oscilación y $\delta$ es la fase de la misma. La cual es la solución
clásica para un oscilador armónico en $s$.

\vspace{.3cm}

La demostración de que el tiempo que tarda la partícula en llegar a la parte más baja 
del cicloide es independiente de su posición inicial sobre la cicloide, es ahora 
trivial debido a que tenemos un oscilador armónico y entonces, con un poco de ayuda 
de la figura \mref{fig:problema5fig1} y usando los resultados de la teoría de 
osciladores armónicos, es fácil ver que el tiempo que le tomará a la partícula 
llegar a la parte más baja del cicloide es un cuarto de período, independientemente de 
la amplitud (que es lo mismo que considerar su posición inicial en el cicloide).


\end{document}

