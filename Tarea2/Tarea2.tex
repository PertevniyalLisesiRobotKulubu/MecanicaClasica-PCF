\documentclass[a4paper,10pt]{article}
\usepackage[utf8]{inputenc}
\usepackage[spanish]{babel}
\usepackage[affil-it]{authblk}
\usepackage{enumerate}
\usepackage{graphicx}
\usepackage{hyperref}
\usepackage{amsmath}
\usepackage{amssymb}
\usepackage{cancel}
\usepackage{tikz}
\usetikzlibrary{calc}
\usepackage{listings}

%Boxes
\newcommand*{\boxcolor}{blue}
\makeatletter
\renewcommand{\boxed}[1]{\textcolor{\boxcolor}{%
\tikz[baseline={([yshift=-1ex]current bounding box.center)}] \node [rectangle, minimum width=1ex,rounded corners,draw] {\normalcolor\m@th$\displaystyle#1$};}}
 \makeatother

%Constants
\newcommand{\euler}{\mathrm{e}}
\newcommand{\im}{\mathrm{i}}

\title{Mecánica Clásica Tarea \# 2}
\author{Favio Vázquez\thanks{Correo: favio.vazquezp@gmail.com}}\affil{Instituto de Física. Universidad Nacional Autónoma de México}
\date{}

\begin{document}

\makeatletter
\def\@maketitle{%
  \newpage
  \null
  \vskip 2em%
  \begin{center}%
  \let \footnote \thanks
    {\Large\bfseries \@title \par}%
    \vskip 1.5em%
    {\normalsize
      \lineskip .5em%
      \begin{tabular}[t]{c}%
        \@author
      \end{tabular}\par}%
    \vskip 1em%
    {\normalsize \@date}%
  \end{center}%
  \par
  \vskip 1.5em}
\makeatother

\maketitle

%Styling for code
\definecolor{codegreen}{rgb}{0,0.6,0}
\definecolor{codegray}{rgb}{0.5,0.5,0.5}
\definecolor{codepurple}{rgb}{0.58,0,0.82}
\definecolor{backcolour}{rgb}{0.95,0.95,0.92}
 
\lstdefinestyle{mystyle}{
    backgroundcolor=\color{backcolour},   
    commentstyle=\color{codegreen},
    keywordstyle=\color{magenta},
    numberstyle=\tiny\color{codegray},
    stringstyle=\color{codepurple},
    basicstyle=\footnotesize,
    breakatwhitespace=false,         
    breaklines=true,                 
    captionpos=b,                    
    keepspaces=true,                 
    numbers=left,                    
    numbersep=5pt,                  
    showspaces=false,                
    showstringspaces=false,
    showtabs=false,                  
    tabsize=2
}
 
\lstset{style=mystyle}

1.- Usando una computadora trace el diagrama de fase de estas dos ecuaciones diferenciales:

\begin{gather*}
 \begin{split}
\dot{x} = x - y - x(x^2+y^2), \\
\dot{y} = x + y - y(x^2+y^2).
 \end{split}
\end{gather*}


Para la siguiente ecuación diferencial, 

$$
\ddot{x} + \epsilon \dot{x} - x + x^3 = 0.
$$

Trace los campos vectoriales correspondientes (use la computadora). Haga un análisis de los 
puntos críticos y del comportamiento de las trayectorias en sus entornos.

\vspace{.3cm}

\underline{Solución:}

Para la solución a las primeras dos ecuaciones solo nos piden que tracemos el diagrama
de fases usando la computadora. Utilizamos Matplotlib\footnote{\href{http://matplotlib.org/}{http://matplotlib.org/}} la librería para hacer graficar 
por excelencia de Python. El código completo de la realización de esta figura se encuentra
en un repositorio libre de GitHub, hecho con Jupyter Notebooks. Puede encontrarse en el
siguiente link: 

El código que finalmente hace el gráfico de la figura () es el siguiente:

\begin{lstlisting}[language=Python]
Y, X = np.mgrid[-3:3:100j, -3:3:100j]
a = np.mgrid[-10:10:100j]
U = X - Y - X*(X**2+Y**2)
V = X + Y - Y*(X**2+Y**2)

plt.streamplot(X, Y, U, V,color=V,density=[1.2, 1.2], 
	      linewidth=2, cmap=plt.cm.winter)

plt.show()
\end{lstlisting}



\vspace{.3cm}

2.- ¿Podrían las funciones 

\begin{gather*}
 \begin{split}
 x(t) = at^2 + x_0, \\
 y(t) = ctx_0 + (1-t)y_0,
 \end{split}
\end{gather*}

ser la solución de una ecuación diferencial de la forma?

\begin{gather*}
 \begin{split}
 \dot{x} = f(x,y),
 \dot{y} = g(x,y)
 \end{split}
\end{gather*}

\vspace{.3cm}

\underline{Solución:}

\vspace{.3cm}

3.- Haga una clasificación de los puntos críticos regulares en un espacio de fase 
de tres dimensiones y trace un diagrama de fase para cada uno de ellos.

\vspace{.3cm}

\underline{Solución:}

\vspace{.3cm}

4.- Se han encontrado cuatro puntos críticos de una ecuación diferencial en la esfera
de dos dimensiones ($R^2$ cerrado al incluir el punto al infinito). Uno es un foco
estable y los otros tres son puntos silla. ¿Será posible que la ecuación no tenga
más puntos críticos? Explique su respuesta.


\vspace{.3cm}

\underline{Solución:}

\vspace{.3cm}

5.- Dé una demostración formal de la equivalencia matemática entre una ecuación
diferencial y un flujo.

\vspace{.3cm}

\underline{Solución:}

\vspace{.3cm}

\end{document}
