\documentclass[a4paper,10pt]{article}
\usepackage[utf8]{inputenc}
\usepackage[spanish]{babel}
\usepackage[affil-it]{authblk}
\usepackage{enumerate}
\usepackage{graphicx}
\usepackage{hyperref}
\usepackage{amsmath}
\usepackage{amssymb}
\usepackage{cancel}
\usepackage{tikz}
\usetikzlibrary{calc}

%Boxes
\newcommand*{\boxcolor}{blue}
\makeatletter
\renewcommand{\boxed}[1]{\textcolor{\boxcolor}{%
\tikz[baseline={([yshift=-1ex]current bounding box.center)}] \node [rectangle, minimum width=1ex,rounded corners,draw] {\normalcolor\m@th$\displaystyle#1$};}}
 \makeatother

%Constantes
\newcommand{\euler}{\mathrm{e}}
\newcommand{\im}{\mathrm{i}}

\title{Mecánica Clásica Tarea \# 2}
\author{Favio Vázquez\thanks{Correo: favio.vazquezp@gmail.com}}\affil{Instituto de Física. Universidad Nacional Autónoma de México}
\date{}

\begin{document}

\makeatletter
\def\@maketitle{%
  \newpage
  \null
  \vskip 2em%
  \begin{center}%
  \let \footnote \thanks
    {\Large\bfseries \@title \par}%
    \vskip 1.5em%
    {\normalsize
      \lineskip .5em%
      \begin{tabular}[t]{c}%
        \@author
      \end{tabular}\par}%
    \vskip 1em%
    {\normalsize \@date}%
  \end{center}%
  \par
  \vskip 1.5em}
\makeatother

\maketitle

1.- Usando una computadora trace el diagrama de fase de estas dos ecuaciones diferenciales:

\begin{gather*}
 \begin{split}
\dot{x} = x - y - x(x^2+y^2), \\
\dot{y} = x + y - y(x^2+y^2).
 \end{split}
\end{gather*}


Para la siguiente ecuación diferencial, 

$$
\ddot{x} + \epsilon \dot{x} - x + x^3 = 0.
$$

Trace los campos vectoriales correspondientes (use la computadora). Haga un análisis de los 
puntos críticos y del comportamiento de las trayectorias en sus entornos.

\vspace{.3cm}

\underline{Solución:}

\vspace{.3cm}

2.- ¿Podrían las funciones 

\begin{gather*}
 \begin{split}
 x(t) = at^2 + x_0, \\
 y(t) = ctx_0 + (1-t)y_0,
 \end{split}
\end{gather*}

ser la solución de una ecuación diferencial de la forma?

\begin{gather*}
 \begin{split}
 \dot{x} = f(x,y),
 \dot{y} = g(x,y)
 \end{split}
\end{gather*}

\vspace{.3cm}

\underline{Solución:}

\vspace{.3cm}

3.- Haga una clasificaci


\end{document}
