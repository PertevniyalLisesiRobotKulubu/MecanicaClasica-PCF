\documentclass[a4paper,10pt]{article}
\usepackage[utf8]{inputenc}
\usepackage[spanish]{babel}
\usepackage[affil-it]{authblk}
\usepackage{enumerate}
\usepackage{graphicx}
\usepackage{hyperref}
\usepackage{amsmath}
\usepackage{amssymb}
\usepackage{cancel}
\usepackage{tikz}
\usetikzlibrary{calc}
\usepackage{listings}

%Boxes
\newcommand*{\boxcolor}{blue}
\makeatletter
\renewcommand{\boxed}[1]{\textcolor{\boxcolor}{%
\tikz[baseline={([yshift=-1ex]current bounding box.center)}] \node [rectangle, minimum width=1ex,rounded corners,draw] {\normalcolor\m@th$\displaystyle#1$};}}
 \makeatother

%Constants
\newcommand{\euler}{\mathrm{e}}
\newcommand{\im}{\mathrm{i}}

\title{Mecánica Clásica Tarea \# 2}
\author{Favio Vázquez\thanks{Correo: favio.vazquezp@gmail.com}}\affil{Instituto de Física. Universidad Nacional Autónoma de México}
\date{}

\begin{document}

\makeatletter
\def\@maketitle{%
  \newpage
  \null
  \vskip 2em%
  \begin{center}%
  \let \footnote \thanks
    {\Large\bfseries \@title \par}%
    \vskip 1.5em%
    {\normalsize
      \lineskip .5em%
      \begin{tabular}[t]{c}%
        \@author
      \end{tabular}\par}%
    \vskip 1em%
    {\normalsize \@date}%
  \end{center}%
  \par
  \vskip 1.5em}
\makeatother

\maketitle

%Styling for code
\definecolor{codegreen}{rgb}{0,0.6,0}
\definecolor{codegray}{rgb}{0.5,0.5,0.5}
\definecolor{codepurple}{rgb}{0.58,0,0.82}
\definecolor{backcolour}{rgb}{0.95,0.95,0.92}
 
\lstdefinestyle{mystyle}{
    backgroundcolor=\color{backcolour},   
    commentstyle=\color{codegreen},
    keywordstyle=\color{magenta},
    numberstyle=\tiny\color{codegray},
    stringstyle=\color{codepurple},
    basicstyle=\footnotesize,
    breakatwhitespace=false,         
    breaklines=true,                 
    captionpos=b,                    
    keepspaces=true,                 
    numbers=left,                    
    numbersep=5pt,                  
    showspaces=false,                
    showstringspaces=false,
    showtabs=false,                  
    tabsize=2
}
 
\lstset{style=mystyle}

1.- Usando una computadora trace el diagrama de fase de estas dos ecuaciones diferenciales:

\begin{gather*}
 \begin{split}
\dot{x} = x - y - x(x^2+y^2), \\
\dot{y} = x + y - y(x^2+y^2).
 \end{split}
\end{gather*}


Para la siguiente ecuación diferencial, 

$$
\ddot{x} + \epsilon \dot{x} - x + x^3 = 0.
$$

Trace los campos vectoriales correspondientes (use la computadora). Haga un análisis de los 
puntos críticos y del comportamiento de las trayectorias en sus entornos.

\vspace{.3cm}

\underline{Solución:}

Para la solución a las primeras dos ecuaciones solo nos piden que tracemos el diagrama
de fases usando la computadora. Utilizamos Matplotlib\footnote{\href{http://matplotlib.org/}{http://matplotlib.org/}} la librería para hacer graficar 
por excelencia de Python. El código completo de la realización de esta figura se encuentra
en un repositorio libre de GitHub, hecho con Jupyter Notebooks. Puede encontrarse en el
siguiente link: \href{https://github.com/FavioVazquez/MecanicaClasica-PCF/blob/master/Tarea2/Tarea2\%20-\%20Problema1.ipynb}{GitHub-Repo}

\begin{figure}[ht]
 \centering
\includegraphics[scale=0.5]{problema1fig1}
\caption{Problema 1 parte 1}
\label{fig:problema1fig1}
\end{figure}


\vspace{.3cm}

Podemos ver que claramente hay un punto crítico en el origen y que coexisten 
en el espacio de fases un foco estable y uno inestable, donde obviamente debe haber
una separación entre ellos para que puedan convivir en el espacio, la cual se puede
observar claramente también.

\vspace{.3cm}

El código que finalmente hace el gráfico de la figura (\ref{fig:problema1fig1}) es el siguiente:

\begin{lstlisting}[language=Python]
Y, X = np.mgrid[-3:3:100j, -3:3:100j]
a = np.mgrid[-10:10:100j]
U = X - Y - X*(X**2+Y**2)
V = X + Y - Y*(X**2+Y**2)

plt.streamplot(X, Y, U, V,color=V,density=[1.2, 1.2], 
	      linewidth=2, cmap=plt.cm.winter)

plt.show()
\end{lstlisting}

Para la parte dos del problema nos piden que hagamos un análisis de los puntos críticos
y del comportamiento de las trayectorias en sus entornos. Para eso podemos reescribir
la ecuación diferencial de segundo orden como un sistema de ecuaciones diferenciales
lineales,


\begin{align}
 \begin{split}
  %
  \dot{x} &= y \\
  %
  \dot{y} &= - \epsilon y + x - x^3
 \end{split}
 \label{eq:ODEsystem1}
\end{align}

Haremos un análisis de estabilidad para estas ecuaciones de una forma muy conocida, utilizando
la matriz jacobiana y estudiando el comportamiento de los puntos críticos en la misma,
lo cual nos dará información sobre el comportamiento de las trayectorias en sus entornos.

\vspace{.3cm}

Para eso escribiremos (\ref{eq:ODEsystem1}) de la siguiente forma

\begin{align}
 \begin{split}
  %
  \dot{x} &= y = f(x,y)\\
  %
  \dot{y} &= - \epsilon y + x - x^3 = g(x,t)
 \end{split}
 \label{eq:ODEsystem2}
\end{align}

Para hallar los puntos críticos hacemos $0$ a $f(x,y)$ y a $g(x,y)$ en (\ref{eq:ODEsystem2})

\begin{align}
 \begin{split}
  %
  f(x,y) &= y = 0\\
  %
  g(x,t) &= -\epsilon y + x - x^3 = -\epsilon y + x(1 - x^2) = 0
  %
 \label{eq:ODEsystem3}
 \end{split}
\end{align}

Claramente vemos de (\ref{eq:ODEsystem3}) que $y=0$ será siempre parte de los puntos críticos,
y $g(x,y)$ vemos que será cero si $x=0$, $x=1$ o $x=1$. Por lo tanto lo puntos críticos son

\begin{align}
 \begin{split}
  %
  (0,0) \\
  %
  (1,0) \\
  %
  (-1,0) 
  %
 \label{eq:puntoscriticos1}
 \end{split}
\end{align}

Para analizar el comportamiento de las trayectorias al rededor de esos puntos, primero escribimos 
el la matriz jacobiana, que se construye como


\begin{align}
J = \begin{pmatrix}
     f_x & f_y \\
     g_x & g_y
\end{pmatrix} = \begin{pmatrix}
		0 & 1 \\
		1- x^2 & -\epsilon
		\end{pmatrix}
\label{eq:jacobiana1}
\end{align}

Ahora para analizar el comportamiento debemos evaluar la matriz jacobiana en los puntos
críticos y utilizar los conocidos criterios para los puntos críticos.

Para el punto crítico $(0,0)$

\begin{align}
J(0,0) = \begin{pmatrix}
     0 & 1 \\
     1 & -\epsilon
\end{pmatrix}
\label{eq:jacobiana2}
\end{align}

Para encontrar los eigenvalores de (\ref{eq:jacobiana2}) usamos la ecuación

\begin{equation}
 \lambda^2 - (\text{traza J}) \lambda + \det{J} = 0
 \label{eq:calcEigenvalores}
\end{equation}

entonces para $J(0,0)$

\begin{equation}
 \lambda^2 + \epsilon \lambda - 1 = 0
\end{equation}

Esta ecuación resulta en que los eigenvalores para $J(0,0)$ son

\begin{align}
 \begin{split}
  %
  \lambda_{1,2} = - \frac{\epsilon}{2} \pm \frac{\sqrt{\epsilon^2 + 4}}{2}
  %
 \end{split}
\end{align}

Por lo que vemos que tenemos que $\lambda_1 > 0$ y que $\lambda_2 < 0$ para todo valor
de $\epsilon$, y como ambos son reales, tendremos un punto silla. En este tipo de punto
crítico tendremos un comportamiento hiperbólico en el que los ejes coordenados son 
las asíntotas, una de las coordenadas tenderá a infinito, mientras que la otra a cero.

Para el punto crítico $(1,0)$,

\begin{align}
J(0,0) = \begin{pmatrix}
     0 & 1 \\
     1 & -\epsilon
\end{pmatrix}
\label{eq:jacobiana2}
\end{align}

Que tiene por eigenvalores utilizando la ecuación (\ref{eq:calcEigenvalores}),

\begin{align}
 \begin{split}
  %
  \lambda_{1,2} = \pm \sqrt{-\epsilon}
  %
 \end{split}
\end{align}

Donde vemos que si $\epsilon$ es positivo tendremos dos valores positivos por lo
que tendemos un foco inestable. En el caso de que $\epsilon$ sea negativo tenemos un
foco estable. Para el primero tenemos que las curvas tienden asintóticamente a infinito y para el segundo
tienden asintóticamente a cero. 

\vspace{.3cm}

Para el caso del punto crítico $(1,0)$ tendremos el mismo caso anterior. Para
el caso en que $\epsilon = 0$, tendremos dos centros. Entonces podemos construir 
los diagramas de fase y demostrar lo que hemos dicho (arriba).


\begin{figure}[h!]
 \centering
\includegraphics[scale=0.3]{problema1fig2}
\caption{Problema 1 parte 2. Centros}
\label{fig:problema1fig2}
\end{figure}

\begin{figure}[h!]
 \centering
\includegraphics[scale=0.3]{problema1fig3}
\caption{Problema 1 parte 2. Focos inestables}
\label{fig:problema1fig3}
\end{figure}

\begin{figure}[h!]
 \centering
\includegraphics[scale=0.3]{problema1fig4}
\caption{Problema 1 parte 2. Focos estables}
\label{fig:problema1fig4}
\end{figure}




\vspace{.3cm}

2.- ¿Podrían las funciones 

\begin{gather*}
 \begin{split}
 x(t) = at^2 + x_0, \\
 y(t) = ctx_0 + (1-t)y_0,
 \end{split}
\end{gather*}

ser la solución de una ecuación diferencial de la forma?

\begin{gather*}
 \begin{split}
 \dot{x} = f(x,y),
 \dot{y} = g(x,y)
 \end{split}
\end{gather*}

\vspace{.3cm}

\underline{Solución:}

\vspace{.3cm}

3.- Haga una clasificación de los puntos críticos regulares en un espacio de fase 
de tres dimensiones y trace un diagrama de fase para cada uno de ellos.

\vspace{.3cm}

\underline{Solución:}

\vspace{.3cm}

4.- Se han encontrado cuatro puntos críticos de una ecuación diferencial en la esfera
de dos dimensiones ($R^2$ cerrado al incluir el punto al infinito). Uno es un foco
estable y los otros tres son puntos silla. ¿Será posible que la ecuación no tenga
más puntos críticos? Explique su respuesta.


\vspace{.3cm}

\underline{Solución:}

\vspace{.3cm}

5.- Dé una demostración formal de la equivalencia matemática entre una ecuación
diferencial y un flujo.

\vspace{.3cm}

\underline{Solución:}

\vspace{.3cm}

\end{document}
