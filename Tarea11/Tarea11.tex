\documentclass[a4paper,10pt]{article}
\usepackage[utf8]{inputenc}
\usepackage[spanish]{babel}
\usepackage[affil-it]{authblk}
\usepackage{enumerate}
\usepackage{graphicx}
\usepackage{hyperref}
\usepackage{amsmath}
\usepackage{amssymb}
\usepackage{cancel}
\usepackage[usenames, dvipsnames]{color}
\usepackage{tikz}
\usepackage[labelfont=bf]{caption}
\usepackage{subcaption} %Multiple images
\usepackage{multicol} % Multiple columns
\usepackage{float}
\usepackage{cleveref}
 \usepackage{relsize} % bigger math symbols
\usepackage[margin=1.4in]{geometry}
\usepackage[titletoc,toc,title]{appendix}
\usepackage{enumitem}
\usepackage{etoolbox}
\usetikzlibrary{calc}
\numberwithin{equation}{section}

\graphicspath{Tarea10/}

% Circled words
\newcommand{\circled}[2][]{%
  \tikz[baseline=(char.base)]{%
    \node[shape = circle, draw, inner sep = 1pt]
    (char) {\phantom{\ifblank{#1}{#2}{#1}}};%
    \node at (char.center) {\makebox[0pt][c]{#2}};}}
\robustify{\circled}

%Appendices in spanish
\renewcommand{\appendixname}{Ap\'endices}
\renewcommand{\appendixtocname}{Ap\'endices}
\renewcommand{\appendixpagename}{Ap\'endices}

%Zero delimiter
\newcommand{\zerodel}{.\kern-\nulldelimiterspace}

%Columns separation
\setlength{\columnsep}{1cm}

%Indentation
\setlength{\parindent}{0ex}

%Multiple References

\crefrangelabelformat{equation}{(#3#1#4--#5\crefstripprefix{#1}{#2}#6)}

\usepackage{xparse}

%Boxes

\newcommand*{\boxcolor}{blue}
\makeatletter
\renewcommand{\boxed}[1]{\textcolor{\boxcolor}{%
\tikz[baseline={([yshift=-1ex]current bounding box.center)}] \node [rectangle, minimum width=1ex,rounded corners,draw] {\normalcolor\m@th$\displaystyle#1$};}}
 \makeatother

%Constantes
\newcommand{\euler}{\mathrm{e}}
\newcommand{\im}{i}

%Lemas, teoremas, definiciones y pruebas
\newcommand{\definicion}{\textbf{Definición: }}
\newcommand{\lema}{\textbf{Lema: }}
\newcommand{\teorema}{\textbf{Teorema: }}
\newcommand{\prueba}{\textbf{Prueba: }}
\newcommand{\proposicion}{\textbf{Proposición: }}
\newcommand{\corolario}{\textbf{Corolario: }}


%opening
\title{Mecánica Clásica Tarea \# 11}
\author{Favio Vázquez\thanks{Correo: favio.vazquezp@gmail.com}}\affil{Instituto de Ciencias Nucleares. Universidad Nacional Autónoma de México.}
\date{}

\begin{document}

\makeatletter
\def\@maketitle{%
  \newpage
  \null
  \vskip 2em%
  \begin{center}%
  \let \footnote \thanks
    {\Large\bfseries \@title \par}%
    \vskip 1.5em%
    {\normalsize
      \lineskip .5em%
      \begin{tabular}[t]{c}%
        \@author
      \end{tabular}\par}%
    \vskip 1em%
    {\normalsize \@date}%
  \end{center}%
  \par
  \vskip 1.5em}
\makeatother

\maketitle

\section{Problema 1}

Demuestre, con todo detalle, que un producto interno definido en cada espacio tangente 
establece una relación entre campos vectoriales y uno-formas diferenciales que se reduce 
a isomorfismos naturales entre cada pareja de espacios tangentes y cotangentes. Haga lo 
mismo para el caso de una forma simpléctica. 

\vspace{.3cm}

\underline{Solución:} \vspace{.3cm}

\section{Problema 2}

Demuestre, de manera detallada y rigurosa, que los productos externos 

$$
dq^i \wedge dq^j, \quad i < j
$$

forman una base local para el espacio de las dos-formas diferenciales.

\vspace{.3cm}

\underline{Solución:} \vspace{.3cm}

Una base para las uno-formas está dada por el conjunto $\{dq^i\}$, lo cual podemos 
ver fácilmente. Sea $\mathbf{\omega} \in TM_x^*$ una uno-forma arbitraria, y sea 
$\mathbf{\omega}[\mathbf{e}_i] \equiv \omega_i \in \mathbb{R}$. Entonces para un $\mathbf{v} \in 
TM_x$ arbitrario 
f
\begin{equation}
 \mathbf{\omega}[\mathbf{v}] = \mathbf{\omega}[v^i\mathbf{e}_i] =
 v^i\mathbf{\omega}[\mathbf{e}_i] = \omega_iv^i,
\end{equation}

y

\begin{equation}
 \omega_i dq^i[\mathbf{v}] = \omega_i dq^i[v^j e_j] = \omega_i v^j \delta^i_j = \omega_i v^i,
\end{equation}

donde hemos utilizado el hecho que $dq^i[\mathbf{e}_i] = \delta^i_j$. Vemos entonces que 
como habíamos establecido el conjunto $\{dq^i\}$ forma una base para las uno-formas 
en $TM_x^*$, esta base es la dual de la base $e_i$ de $TM_x$. 

\vspace{.3cm}

Lo que queremos probar ahora es que el producto exterior de estas bases forman una 
base para las dos-formas. Sea entonces $\gamma$ una dos-forma arbitraria y sea de nuevo 
$\{\mathbf{e}_i\}$ una base para los vectores en $TM_x$ correspondientes a la base 
$\{dq^i \}$ en $TM_x^*$. Sea $\gamma_{ij} \equiv \gamma[\mathbf{e}_i,\mathbf{e}_j] = 
- \gamma[\mathbf{e}_j,\mathbf{e}_i] = - \gamma_{ji}$. También sean $\mathbf{v} = v^i\mathbf{e}_i$ 
y $\mathbf{u} = u^i\mathbf{e}_i$ vectores arbitrarios en $TM_x$. Consideremos ahora la 
dos-forma $(1/2)\gamma_{ij}dq^i \wedge dq^j$ y mostremos que es idéntica a $\gamma$ 
probando que toma exactamente los mismos valores para $\mathbf{v}$ y $\mathbf{u}$ arbitrarios.

\begin{equation}
 \gamma[\mathbf{v},\mathbf{u}] = \gamma[v^i\mathbf{e}_i,u^j\mathbf{e}_j] = 
 v^iu^j\gamma[\mathbf{e}_i,\mathbf{e}_j] = v^iu^j\gamma_{ij}.
\end{equation}

\begin{equation}
 \frac{1}{2}\gamma_{ij} dq^i \wedge dq^j [\mathbf{v},\mathbf{u}] = 
 \frac{1}{2}\gamma_{ij}(dq^i[\mathbf{v}]dq^j[\mathbf{u}] - dq^i[\mathbf{u}]dq^j[\mathbf{v}],
\end{equation}

\begin{equation}
 \boxed{\therefore \frac{1}{2}\gamma_{ij} dq^i \wedge dq^j [\mathbf{v},\mathbf{u}] = 
 \frac{1}{2}\gamma_{ij}(v^iu^j - u^iv^j) = \gamma_{ij}v^iu^j}.
\end{equation}

Donde hemos usado el hecho de que $\gamma_{ij} = - \gamma_{ji}$. Claramente estas decisiones 
de comparación con $(1/2)\gamma_{ij}dq^i \wedge dq^j$ es para tomar en cuenta el hecho 
de que esto es verdad si $i < j$. Vemos entonces que las dos formas pueden escribirse 
en términos de las bases $dq^i \wedge dq^j$ para $i < j$ que era lo que queríamos probar.


\section{Problema 3}

Demuestre, de manera detallada y rigurosa, que al estar definido un producto interno 
en cada espacio tangente a una variedad diferencial que presenta un cambio continuo 
y diferencial, queda definida una métrica para la variedad diferencial.

\vspace{.3cm}

\underline{Solución:} \vspace{.3cm}

Necesitamos una serie de definiciones y teoremas para demostrar de manera detallada 
y rigurosa este enunciado. Comencemos hablando un poco sobre lo que es un producto 
interno en un espacio vectorial. La mayoría del tratamiento que haremos en este 
problema proviene de 

\vspace{.3cm}

\definicion Un producto interno sobre un espacio vectorial $V$ sobre un campo $F$ es 
una función bilineal de $V \times V$ a $F$, denotado por $\langle \bullet, \bullet \rangle$, que 
es simétrico

\begin{equation}
 \langle v, w \rangle = \langle w, v \rangle,
\end{equation}

y no degenerado: si $v\ne 0$, entonces $\exists$ algún $w \ne 0$ tal que

\begin{equation}
 \langle w, v \rangle \ne 0.
\end{equation}

Para el tratamiento que haremos en este problema el campo $F$ siempre será $\mathbb{R}$. Para 
cualquier $r$ en $0 \leq n \geq n$, podemos definir el producto interno 
$\langle \space , \space \rangle_r$ sobre $\mathbb{R}^n$ por 

\begin{equation}
 \langle a, b \rangle_r = \sum_{i=1}^r a^ib^i - \sum_{i=r+1}^n a^ib^i,
\end{equation}

que es no degenerado debido a que si $a \ne 0$, entonces $\sum_{i=1}^n (a^i)^2 > 0$. 
En particular si $r = n$ obtenemos el producto interno usual en $\mathbb{R}^n$,

\begin{equation}
 \langle a, b \rangle = \sum_{i=1}^n a^ib^i.
\end{equation}

Para este producto interno tenemos $\langle a, a \rangle > 0$ para cualquier 
$a \ne 0$. En general una función bilineal $\langle \space, \space \rangle$ es 
llamada positiva definida si 

\begin{equation}
 \langle v, v \rangle > 0 \quad \forall v \ne 0.
\end{equation}

Una función bilineal positiva definida $\langle \space, \space \rangle$
claramente es no degenerada y consecuentemente un producto interno. 

\vspace{.3cm}

Aunque hay mucho más que podríamos decir sobre los productos internos, 
pasemos a ver como éstos están relacionados a las métricas, y particularmente 
a las métricas riemannianas que son las que nos interesan para probar lo que
nos piden. 

\definicion Un haz vectorial $n$-dimensional (o un haz $n$-plano)
es una quíntupla 

\begin{equation}
 \xi = (E,\pi,B,\oplus,\odot),
\end{equation}

donde 

\begin{itemize}
 \item $E$ y $B$ son espacios (el ``espacio total'' y el ``espacio base'' de 
 $\xi$, respectivamente),
 \item $\pi: E \rightarrow B$ es un mapeo continuo hacia $B$,
 \item $\oplus$ y $\odot$ son mapeos
 
 \begin{equation}
  \oplus: \underset{p \in b}{\bigcup} \pi^{-1}(p) \times \pi^{-1}(p) \rightarrow E, \quad 
  \odot: R \times E \rightarrow E,
 \end{equation}
\end{itemize}

con $\oplus(\pi^{-1}(p) \times \pi^{-1}(p)) \subseteq \pi^{-1}(p)$, que hace 
cada fibra $\pi^{-1}(p)$ sea un espacio vectorial $n$-dimensional sobre 
$\mathbb{R}$.

\definicion Si $\xi = \pi: E \rightarrow B$ es un haz vectorial, definimos 
una métrica riemanniana sobre $\xi$ como una función $\langle \space, \space \rangle$, 
que asigna a cada $p \in B$ un producto interno positivo definido $\langle \space, \space \rangle_p$
en $\pi^{-1}(p)$ y que es continuo en el sentido que de que para dos secciones continuas 
$s_1,s_2: B \rightarrow E$, la función

\begin{equation}
 \langle s_1, s_2 \rangle = p \mapsto \langle s_1(p), s_2(p) \rangle_p
\end{equation}

es también continuo. Si $\xi$ es un haz vectorial $C^\infty$ sobre una variedad 
$C^\infty$ hablamos de métricas riemannianas $C^\infty$. Esto será de importancia pronto.

\vspace{.3cm}

Estamos en posición ahora de enunciar el teorema que nos servirá para 
demostrar lo que queremos.

\vspace{.3cm}

\teorema Sea $\xi = \pi: E \rightarrow M$ un $[C^\infty]$ haz $k$-plano
sobre una variedad $C^\infty$. Entonces $\exists$ una métrica riemanniana 
$[C^\infty]$ sobre $\xi$.

\vspace{.3cm}

\prueba Existe una cobertura abierta y localmente finita $\mathcal{O}$ sobre 
$M$ por los conjuntos $U$ para los cuales existen trivializaciones $[C^\infty]$

\begin{equation}
 t_U: \pi^{-1}(U) \rightarrow U \times \mathbb{R}^k.
\end{equation}

Sobre $U \times \mathbb{R}^k$, existe una obviamente una métrica riemanniana,

\begin{equation}
 \langle (p,a), (p,b)\rangle_p = \langle a, b \rangle. 
\end{equation}

Para $v,w \in \pi^{-1}(p)$, definimos

\begin{equation}
 \langle v, w \rangle_p^U = \langle t_U(v),t_U(w)\rangle_p.
\end{equation}

Entonces $\langle \space , \space \rangle^U$ es una métrica riemanniana
$[C^\infty]$ para $\xi|U$. Sea $\{\phi_U \}$ una partición de la unidad\footnote{ Una partición 
de la unidad de un espacio $X$ es un conjunto de funciones continuas de $X$ a $[0,1]$ tal que todo punto 
tenga un entorno donde todas las funciones sean cero excepto un número finito de ellas, y que la 
suma de todas las funciones sobre todo el espacio sea idénticamente 1.} 
subordinada a $\mathcal{O}$. Definimos $\langle \space, \space \rangle$ por 

\begin{equation}
 \langle v, w \rangle_p = \sum_{U \in \mathcal{O}} \phi_U(p)\langle v, w \rangle_p^U, 
 \qquad v,w \in \pi^{-1}(p).
\end{equation}

Entonces $\langle \space, \space \rangle$ es continua $[C^\infty]$ y 
cada $\langle \space, \space \rangle_p$ es una función simétrica y 
bilineal sobre $\pi^{-1}(p)$. Para mostrar que es positiva definida, 
notamos que 

\begin{equation}
 \langle v, v \rangle_p =  \sum_{U \in \mathcal{O}} \phi_U(p)\langle v, v \rangle_p^U,
\end{equation}

y cada $\phi_U(p)\langle v, w \rangle_p^U \geq 0$, para algún $U$ la igualdad 
se mantiene.

$\hspace{12cm} \square$

Este teorema cobra una especial significación, y le da solución a este
problema cuando el haz vectorial del teorema que probamos es el es haz 
tangente a una variedad diferenciable $C^\infty$ $M$. En este caso, 
el teorema (ver detalladamente la prueba) nos dice que definir un producto 
interno positivo definido en cada espacio tangente a $M$ define una 
métrica riemanniana $C^\infty$ en $M$. Para esclarecer un poco lo que hemos hecho,
si $(x,U)$ es un sistema de coordenadas en $M$, entonces sobre $U$ podemos 
escribir nuestra métrica riemanniana $\langle \space, \space \rangle$ como 

\begin{equation}
 \langle \space , \space \rangle = \sum_{i,j = 0}^n g_{ij}dx^i \otimes dx^j.
\end{equation}

donde las funciones $C^\infty$ $g_{ij}$ satisfacen $g_{ij} = g_{ji}$, debido 
a que $langle \space , \space \rangle$ es simétrico, y $\det{a} > 0$, debido a
que $langle \space , \space \rangle$ es positivo definido. Claramente la métrica 
riemanniana que hemos definido en $M$ es un tensor covariante de orden 2, en realidad 
son las componentes de un campo tensorial ya que está definido como un tensor 
en el espacio tangente en cada punto de la variedad, pero comúnmente 
solo se habla de él como un tensor.











\section{Problema 4}

En la base inducida por coordenadas, encuentre las componentes de un campo vectorial 
hamiltoniano cuando las coordenadas en el espacio fase, $(x^i,\dots,x^{2n})$, son 
arbitrarias.

\vspace{.3cm}

\underline{Solución:} \vspace{.3cm}

\section{Problema 5}

Dado un sistema canónico de coordenadas $(q,p)$ para un espacio de fase donde las $q$ 
son coordenadas de la variedad de configuración, demuestre que la diferencial exterior 
de la uno-forma diferencial 

$$
\omega = \sum_i p_idq^i
$$

es una dos-forma diferencial NO DEGENERADA y por tanto simpléctica.

\vspace{.3cm}

\underline{Solución:} \vspace{.3cm}

\end{document}