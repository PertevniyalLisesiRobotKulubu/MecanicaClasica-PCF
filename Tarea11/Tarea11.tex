\documentclass[a4paper,10pt]{article}
\usepackage[utf8]{inputenc}
\usepackage[spanish]{babel}
\usepackage[affil-it]{authblk}
\usepackage{enumerate}
\usepackage{graphicx}
\usepackage{hyperref}
\usepackage{amsmath}
\usepackage{amssymb}
\usepackage{cancel}
\usepackage[usenames, dvipsnames]{color}
\usepackage{tikz}
\usepackage[labelfont=bf]{caption}
\usepackage{subcaption} %Multiple images
\usepackage{multicol} % Multiple columns
\usepackage{float}
\usepackage{cleveref}
 \usepackage{relsize} % bigger math symbols
\usepackage[margin=1.4in]{geometry}
\usepackage[titletoc,toc,title]{appendix}
\usepackage{enumitem}
\usepackage{etoolbox}
\usetikzlibrary{calc}
\numberwithin{equation}{section}

\graphicspath{Tarea10/}

% Circled words
\newcommand{\circled}[2][]{%
  \tikz[baseline=(char.base)]{%
    \node[shape = circle, draw, inner sep = 1pt]
    (char) {\phantom{\ifblank{#1}{#2}{#1}}};%
    \node at (char.center) {\makebox[0pt][c]{#2}};}}
\robustify{\circled}

%Appendices in spanish
\renewcommand{\appendixname}{Ap\'endices}
\renewcommand{\appendixtocname}{Ap\'endices}
\renewcommand{\appendixpagename}{Ap\'endices}

%Zero delimiter
\newcommand{\zerodel}{.\kern-\nulldelimiterspace}

%Columns separation
\setlength{\columnsep}{1cm}

%Indentation
\setlength{\parindent}{0ex}

%Multiple References

\crefrangelabelformat{equation}{(#3#1#4--#5\crefstripprefix{#1}{#2}#6)}

\usepackage{xparse}

%Boxes

\newcommand*{\boxcolor}{blue}
\makeatletter
\renewcommand{\boxed}[1]{\textcolor{\boxcolor}{%
\tikz[baseline={([yshift=-1ex]current bounding box.center)}] \node [rectangle, minimum width=1ex,rounded corners,draw] {\normalcolor\m@th$\displaystyle#1$};}}
 \makeatother

%Constantes
\newcommand{\euler}{\mathrm{e}}
\newcommand{\im}{i}

%Lemas, teoremas, definiciones y pruebas
\newcommand{\definicion}{\textbf{Definición: }}
\newcommand{\lema}{\textbf{Lema: }}
\newcommand{\teorema}{\textbf{Teorema: }}
\newcommand{\prueba}{\textbf{Prueba: }}
\newcommand{\proposicion}{\textbf{Proposición: }}
\newcommand{\corolario}{\textbf{Corolario: }}


%opening
\title{Mecánica Clásica Tarea \# 11}
\author{Favio Vázquez\thanks{Correo: favio.vazquezp@gmail.com}}\affil{Instituto de Ciencias Nucleares. Universidad Nacional Autónoma de México.}
\date{}

\begin{document}

\makeatletter
\def\@maketitle{%
  \newpage
  \null
  \vskip 2em%
  \begin{center}%
  \let \footnote \thanks
    {\Large\bfseries \@title \par}%
    \vskip 1.5em%
    {\normalsize
      \lineskip .5em%
      \begin{tabular}[t]{c}%
        \@author
      \end{tabular}\par}%
    \vskip 1em%
    {\normalsize \@date}%
  \end{center}%
  \par
  \vskip 1.5em}
\makeatother

\maketitle

\section{Problema 1}

Demuestre, con todo detalle, que un producto interno definido en cada espacio tangente 
establece una relación entre campos vectoriales y uno-formas diferenciales que se reduce 
a isomorfismos naturales entre cada pareja de espacios tangentes y cotangentes. Haga lo 
mismo para el caso de una forma simpléctica. 

\vspace{.3cm}

\underline{Solución:} \vspace{.3cm}

\section{Problema 2}

Demuestre, de manera detallada y rigurosa, que los productos internos 

$$
dq^i \wedge dq^j, \quad i < j
$$

forman una base local para el espacio de las dos-formas diferenciales.

\vspace{.3cm}

\underline{Solución:} \vspace{.3cm}

\section{Problema 3}

Demuestre, de manera detallada y rigurosa, que al estar definido un producto interno 
en cada espacio tangente a una variedad diferencial que presenta un cambio continuo 
y diferencial, queda definida una métrica para la variedad diferencial.

\vspace{.3cm}

\underline{Solución:} \vspace{.3cm}

\section{Problema 4}

En la base inducida por coordenadas, encuentre las componentes de un campo vectorial 
hamiltoniano cuando las coordenadasen el espacio fase, $(x^i,\dots,x^{2n}$, son 
arbitrarias.

\vspace{.3cm}

\underline{Solución:} \vspace{.3cm}

\section{Problema 5}

Dado un sistema canónico de coordenadas $(q,p)$ para un espacio de fase donde las $q$ 
son coordenadas de la variedad de configuración, demuestre que la diferencial exterior 
de la uno-forma diferencial 

$$
\omega = \sum_i p_idq^i
$$

es una dos-forma diferencial NO DEGENERADA y por tanto simpléctica.

\vspace{.3cm}

\underline{Solución:} \vspace{.3cm}

\end{document}