\documentclass[a4paper,10pt]{article}
\usepackage[utf8]{inputenc}
\usepackage[spanish]{babel}
\usepackage[affil-it]{authblk}
\usepackage{enumerate}
\usepackage{graphicx}
\usepackage{hyperref}
\usepackage{amsmath}
\usepackage{amssymb}
\usepackage{cancel}
\usepackage[usenames, dvipsnames]{color}
\usepackage{tikz}
\usepackage[labelfont=bf]{caption}
\usepackage{subcaption} %Multiple images
\usepackage{multicol} % Multiple columns
\usepackage{float}
\usepackage{cleveref}
 \usepackage{relsize} % bigger math symbols
\usepackage[margin=1.4in]{geometry}
\usepackage[titletoc,toc,title]{appendix}
\usepackage{enumitem}
\usepackage{etoolbox}
\usetikzlibrary{calc}
\numberwithin{equation}{section}

\graphicspath{Tarea10/}

% Circled words
\newcommand{\circled}[2][]{%
  \tikz[baseline=(char.base)]{%
    \node[shape = circle, draw, inner sep = 1pt]
    (char) {\phantom{\ifblank{#1}{#2}{#1}}};%
    \node at (char.center) {\makebox[0pt][c]{#2}};}}
\robustify{\circled}

%Appendices in spanish
\renewcommand{\appendixname}{Ap\'endices}
\renewcommand{\appendixtocname}{Ap\'endices}
\renewcommand{\appendixpagename}{Ap\'endices}

%Zero delimiter
\newcommand{\zerodel}{.\kern-\nulldelimiterspace}

%Columns separation
\setlength{\columnsep}{1cm}

%Indentation
\setlength{\parindent}{0ex}

%Multiple References

\crefrangelabelformat{equation}{(#3#1#4--#5\crefstripprefix{#1}{#2}#6)}

\usepackage{xparse}

%Boxes

\newcommand*{\boxcolor}{blue}
\makeatletter
\renewcommand{\boxed}[1]{\textcolor{\boxcolor}{%
\tikz[baseline={([yshift=-1ex]current bounding box.center)}] \node [rectangle, minimum width=1ex,rounded corners,draw] {\normalcolor\m@th$\displaystyle#1$};}}
 \makeatother

%Constantes
\newcommand{\euler}{\mathrm{e}}
\newcommand{\im}{i}

%Lemas, teoremas, definiciones y pruebas
\newcommand{\definicion}{\textbf{Definición: }}
\newcommand{\lema}{\textbf{Lema: }}
\newcommand{\teorema}{\textbf{Teorema: }}
\newcommand{\prueba}{\textbf{Prueba: }}
\newcommand{\proposicion}{\textbf{Proposición: }}
\newcommand{\corolario}{\textbf{Corolario: }}


%opening
\title{Mecánica Clásica Tarea \# 11}
\author{Favio Vázquez\thanks{Correo: favio.vazquezp@gmail.com}}\affil{Instituto de Ciencias Nucleares. Universidad Nacional Autónoma de México.}
\date{}

\begin{document}

\makeatletter
\def\@maketitle{%
  \newpage
  \null
  \vskip 2em%
  \begin{center}%
  \let \footnote \thanks
    {\Large\bfseries \@title \par}%
    \vskip 1.5em%
    {\normalsize
      \lineskip .5em%
      \begin{tabular}[t]{c}%
        \@author
      \end{tabular}\par}%
    \vskip 1em%
    {\normalsize \@date}%
  \end{center}%
  \par
  \vskip 1.5em}
\makeatother

\maketitle

\textbf{\underline{Nota}: Para simplificar las ecuaciones, durante toda la tarea 
utilizaremos el convenio de suma de Einstein, el cual nos dice 
que un índice que aparece dos veces en un término matemático, una vez como un 
superíndice y una vez como un subíndice, es sumado sobre el rango entero de ese 
índice.}

\section{Problema 1}

Demuestre, con todo detalle, que un producto interno definido en cada espacio tangente 
establece una relación entre campos vectoriales y uno-formas diferenciales que se reduce 
a isomorfismos naturales entre cada pareja de espacios tangentes y cotangentes. Haga lo 
mismo para el caso de una forma simpléctica. 

\vspace{.3cm}

\underline{Solución:} \vspace{.3cm}

Para la solución de este problema y para no repetir lo que se hace en el
problema 3, se asumirán que las definiciones y propiedades para los productos 
internos y métricas ya están establecidas. La mayoría del tratamiento 
que haremos en este problema fue sacado de \cite{rasband} y \cite{spivak}.

\vspace{.3cm}

Para probar la primera parte del problema debemos probar que cuando 
se define un producto interno, y por lo tanto un tensor métrico (ver problema 3) 
existe, no hay diferencia entre $TM_x$ y $TM_x^*$, y que por lo tanto 
a cada vector le corresponde una forma lineal, y esto elevado a los haces 
tangentes y cotangentes establece un isomorfismo natural entre campos vectoriales 
y uno-formas diferenciales. Para poder hacer esta prueba, y que podamos decir 
que el isomorfismo es natural, la misma no debe depender de cartas, o lo mismo 
que no dependa de coordenadas, o como lo haremos aquí, que se cumpla para 
cualquier vector arbitrario.

\vspace{.3cm}

Sea $\mathbf{u} \in TM_x$ un vector arbitrario, definimos el producto interno 
$\langle \mathbf{u} , \mathbf{v} \rangle$ para todo $\mathbf{v} \in TM_x$. Si vemos 
a $\langle \mathbf{u} , \bullet \rangle$ como un operador sobre $TM_x$, vemos 
por lo demostrado en el problema 3 que es un operador real, lineal y por 
lo tanto una forma lineal; además sabemos que este producto interno induce 
una métrica en $M$ y podemos pensarlo como un mapeo lineal con dos argumentos 
sobre $TM_x$, podemos escribir entonces 

\begin{equation}
 \langle \mathbf{u} , \mathbf{v} \rangle = g_{ij}u^iv^i \in \mathbb{R}.
\end{equation}

De esta ecuación vemos que el mapeo $\langle \mathbf{u} , \bullet \rangle: TM_x 
\rightarrow \mathbb{R}$ es un mapeo real y lineal sobre $TM_x$ y por lo tanto 
es un elemento de $TM_x^*$. Este elemento es la imagen isomorfa de $\mathbf{u}$. 
Si $\mathbf{u} \in TM_x$ tiene componentes $u^i$ en una base natural, entonces 
la imagen isomorfa tiene componentes $u_i = g_{ij}u^j$. Con lo cual vemos que 
como esto fue hecho para vectores cualesquiera, y definido punto a punto, entonces 
vemos exactamente lo pedido, con el sencillo hecho de definir un producto interno 
en cada espacio tangente pudimos establecer un isomorfismo natural entre 
cada pareja de espacios tangentes y cotangentes, con lo cual tenemos inmediatamente 
una relación entre campos vectoriales y uno-formas diferenciales.

\vspace{.3cm}

Pero esta relación puede no parecer evidente así que trabajemos un poco sobre ella,
ya que también nos permitirá hacer lo mismo para una forma simpléctica. Sea $F(q,p)$ 
una función arbitraria del espacio de fase tal que $F: TM^* \rightarrow \mathbb{R}$. Un
flujo hamiltoniano está definido como el flujo 

\begin{equation}
 \frac{dx}{ds} = V_F(x),
 \label{eq:83}
\end{equation}

donde $x=(q,p)$ y donde las componentes de $V_F$ están dadas por 

\begin{equation}
 \left(\frac{dp_i}{ds}, \frac{dq^i}{ds}\right) = 
 \left(- \frac{\partial F}{\partial q^i}, \frac{\partial F}{\partial p_i}\right)
 \label{eq:82}
\end{equation}

Si denotamos los vectores bases como $\{\mathbf{e}_{p_i},\mathbf{e}_{q^i}\}$ 
correspondientes a las coordenadas $(p_i,q^i)$. Y asumimos que $V_F = \mathbf{e}_{q^i}$ 
en \eqref{eq:83}, encontramos que $dq^i/ds = 1$; todas las otras derivadas se cancelan. 
Entonces de \eqref{eq:82} encontramos que $\partial F / \partial p_i = 1$ y todas las 
otras derivadas parciales de $F$ se hacen cero. Entonces $F = p_i$ genera el flujo 
$V_F = \mathbf{e}_{q^i}$ y de forma idéntica $F = -q^i$ genera el flujo 
$\mathbf{e}_{p_{i}}$. Ahora, sabemos que podemos interpretar estos flujos como 
operadores \cite{rasband}, y entonces para

\begin{equation}
 V_F = \frac{\partial F}{\partial p_i}\mathbf{e}_{q^i} - \frac{\partial F}{\partial q^i}\mathbf{e}_{p_{i}},
\label{eq:88}
\end{equation}

definimos el operador

\begin{equation}
 \frac{\partial}{\partial V_F} = \frac{\partial F}{\partial p_i}\frac{\partial}{\partial q^i}
 - \frac{\partial F}{\partial q^i}\frac{\partial}{\partial p_i}.
\end{equation}

Pero la función $F$ también tiene la expresión usual correspondiente a una uno-forma

\begin{equation}
 dF = \frac{\partial F}{\partial p_i}dp_i + \frac{\partial F}{\partial q^i}dq^i.
 \label{eq:810}
\end{equation}

Comparando ahora \eqref{eq:88} con \eqref{eq:810} podemos ahora ver lo que queríamos 
mostrar, que para cada uno-forma diferenciable $dF$ le corresponde un campo vectorial 
$V_F$. La correspondencia en los términos está dada por 

\begin{equation}
 \left(- \frac{\partial F}{\partial q^i}, \frac{\partial F}{\partial p_i}\right) \leftrightarrow
 \left( \frac{\partial F}{\partial p_i}, \frac{\partial F}{\partial q^i}\right).
 \label{eq:811}
\end{equation}

Este isomorfismo entre campos vectoriales y uno-formas diferenciales podemos denotarla 
por $\omega$ y escribirlo como 

\begin{equation}
 \omega(V_F) = dF,
  \label{eq:812}
\end{equation}

donde el isomorfismo está definido por \eqref{eq:811}. Vemos entonces que hemos demostrado 
completamente lo que se pedía en la primera parte, aunque de una forma un poco no ortodoxa, 
a la final se cumplió con el objetivo. Algo que será de importancia para la siguiente 
parte de la prueba es que usando el isomorfismo escrito como \eqref{eq:812} podemos 
escribir los paréntesis de Poisson como \cite{rasband}

\begin{equation}
 \{f,g\} = \omega(V_F)[V_G] = - \omega(V_G)[V_F].
\end{equation}

Ahora para hacerlo con una forma simpléctica, recordamos que podemos escribir esta 
en coordenadas canónicas como

\begin{equation}
 \gamma = dp_i \wedge dq^i.
\end{equation}

Lo que veremos ahora es que el isomorfismo dado en \eqref{eq:811} y \eqref{eq:812} puede 
ser expresado también utilizando $\gamma$. Para hacer esto tomemos un campo vectorial 
arbitrario $V$, y definamos la uno-forma correspondiente a $V$ por 

\begin{equation}
 \omega_V = \gamma[\bullet,V] = dq^i[V]dp_i - dp_i[V]dq^i.
\end{equation}

y debemos asegurarnos ahora que este es el mismo isomorfismo que el definido por \eqref{eq:812}. Para 
verificar esto hacemos que $V = V_F$, en este caso \eqref{eq:82} nos muestra que 
$dq^i[V] = \partial F/ \partial p_i$ y $dp_i[V] = - \partial F/\partial q^i$. Sustituyendo 
esto en la anterior ecuación resulta en 

\begin{equation}
 \omega_{V_F} = \frac{\partial F}{\partial p_i}dp_i + \frac{\partial F}{\partial q^i}dq^i.
\end{equation}

que claramente es la misma expresión para $dF$ que \eqref{eq:810}, con lo cual vemos que 
tenemos el mismo isomorfismo y por lo tanto se ha probado lo que se quería.


\section{Problema 2}

Demuestre, de manera detallada y rigurosa, que los productos externos 

$$
dq^i \wedge dq^j, \quad i < j
$$

forman una base local para el espacio de las dos-formas diferenciales.

\vspace{.3cm}

\underline{Solución:} \vspace{.3cm}

Una base para las uno-formas está dada por el conjunto $\{dq^i\}$, lo cual podemos 
ver fácilmente. Sea $\mathbf{\omega} \in TM_x^*$ una uno-forma arbitraria, y sea 
$\mathbf{\omega}[\mathbf{e}_i] \equiv \omega_i \in \mathbb{R}$. Entonces para un $\mathbf{v} \in 
TM_x$ arbitrario 
f
\begin{equation}
 \mathbf{\omega}[\mathbf{v}] = \mathbf{\omega}[v^i\mathbf{e}_i] =
 v^i\mathbf{\omega}[\mathbf{e}_i] = \omega_iv^i,
\end{equation}

y

\begin{equation}
 \omega_i dq^i[\mathbf{v}] = \omega_i dq^i[v^j e_j] = \omega_i v^j \delta^i_j = \omega_i v^i,
\end{equation}

donde hemos utilizado el hecho que $dq^i[\mathbf{e}_i] = \delta^i_j$. Vemos entonces que 
como habíamos establecido el conjunto $\{dq^i\}$ forma una base para las uno-formas 
en $TM_x^*$, esta base es la dual de la base $e_i$ de $TM_x$. 

\vspace{.3cm}

Lo que queremos probar ahora es que el producto exterior de estas bases forman una 
base para las dos-formas. Sea entonces $\gamma$ una dos-forma arbitraria y sea de nuevo 
$\{\mathbf{e}_i\}$ una base para los vectores en $TM_x$ correspondientes a la base 
$\{dq^i \}$ en $TM_x^*$. Sea $\gamma_{ij} \equiv \gamma[\mathbf{e}_i,\mathbf{e}_j] = 
- \gamma[\mathbf{e}_j,\mathbf{e}_i] = - \gamma_{ji}$. También sean $\mathbf{v} = v^i\mathbf{e}_i$ 
y $\mathbf{u} = u^i\mathbf{e}_i$ vectores arbitrarios en $TM_x$. Consideremos ahora la 
dos-forma $(1/2)\gamma_{ij}dq^i \wedge dq^j$ y mostremos que es idéntica a $\gamma$ 
probando que toma exactamente los mismos valores para $\mathbf{v}$ y $\mathbf{u}$ arbitrarios.

\begin{equation}
 \gamma[\mathbf{v},\mathbf{u}] = \gamma[v^i\mathbf{e}_i,u^j\mathbf{e}_j] = 
 v^iu^j\gamma[\mathbf{e}_i,\mathbf{e}_j] = v^iu^j\gamma_{ij}.
\end{equation}

\begin{equation}
 \frac{1}{2}\gamma_{ij} dq^i \wedge dq^j [\mathbf{v},\mathbf{u}] = 
 \frac{1}{2}\gamma_{ij}(dq^i[\mathbf{v}]dq^j[\mathbf{u}] - dq^i[\mathbf{u}]dq^j[\mathbf{v}],
\end{equation}

\begin{equation}
 \boxed{\therefore \frac{1}{2}\gamma_{ij} dq^i \wedge dq^j [\mathbf{v},\mathbf{u}] = 
 \frac{1}{2}\gamma_{ij}(v^iu^j - u^iv^j) = \gamma_{ij}v^iu^j}.
\end{equation}

Donde hemos usado el hecho de que $\gamma_{ij} = - \gamma_{ji}$. Claramente estas decisiones 
de comparación con $(1/2)\gamma_{ij}dq^i \wedge dq^j$ es para tomar en cuenta el hecho 
de que esto es verdad si $i < j$. Vemos entonces que las dos formas pueden escribirse 
en términos de las bases $dq^i \wedge dq^j$ para $i < j$ que era lo que queríamos probar.


\section{Problema 3}

Demuestre, de manera detallada y rigurosa, que al estar definido un producto interno 
en cada espacio tangente a una variedad diferencial que presenta un cambio continuo 
y diferencial, queda definida una métrica para la variedad diferencial.

\vspace{.3cm}

\underline{Solución:} \vspace{.3cm}

Necesitamos una serie de definiciones y teoremas para demostrar de manera detallada 
y rigurosa este enunciado. Comencemos hablando un poco sobre lo que es un producto 
interno en un espacio vectorial. La mayoría del tratamiento que haremos en este 
problema proviene de \cite{spivak}.

\vspace{.3cm}

\definicion Un producto interno sobre un espacio vectorial $V$ sobre un campo $F$ es 
una función bilineal de $V \times V$ a $F$, denotado por $\langle \bullet, \bullet \rangle$, que 
es simétrico

\begin{equation}
 \langle v, w \rangle = \langle w, v \rangle,
\end{equation}

y no degenerado: si $v\ne 0$, entonces $\exists$ algún $w \ne 0$ tal que

\begin{equation}
 \langle w, v \rangle \ne 0.
\end{equation}

Para el tratamiento que haremos en este problema el campo $F$ siempre será $\mathbb{R}$. Para 
cualquier $r$ en $0 \leq r \leq n$, podemos definir el producto interno 
$\langle \hspace{.2cm} , \hspace{.1cm} \rangle_r$ sobre $\mathbb{R}^n$ por 

\begin{equation}
 \langle a, b \rangle_r = \sum_{i=1}^r a^ib^i - \sum_{i=r+1}^n a^ib^i,
\end{equation}

que es no degenerado debido a que si $a \ne 0$, entonces $\sum_{i=1}^n (a^i)^2 > 0$. 
En particular si $r = n$ obtenemos el producto interno usual en $\mathbb{R}^n$,

\begin{equation}
 \langle a, b \rangle = \sum_{i=1}^n a^ib^i.
\end{equation}

Para este producto interno tenemos $\langle a, a \rangle > 0$ para cualquier 
$a \ne 0$. En general una función bilineal $\langle \hspace{.1cm}, \hspace{.1cm} \rangle$ es 
llamada positiva definida si 

\begin{equation}
 \langle v, v \rangle > 0 \quad \forall v \ne 0.
\end{equation}

Una función bilineal positiva definida $\langle \hspace{.1cm}, \hspace{.1cm} \rangle$
claramente es no degenerada y consecuentemente un producto interno. 

\vspace{.3cm}

Aunque hay mucho más que podríamos decir sobre los productos internos, 
pasemos a ver como éstos están relacionados a las métricas, y particularmente 
a las métricas riemannianas que son las que nos interesan para probar lo que
nos piden. 

\vspace{.3cm}

\definicion Un haz vectorial $n$-dimensional (o un haz $n$-plano)
es una quíntupla 

\begin{equation}
 \xi = (E,\pi,B,\oplus,\odot),
\end{equation}

donde 

\begin{itemize}
 \item $E$ y $B$ son espacios (el ``espacio total'' y el ``espacio base'' de 
 $\xi$, respectivamente),
 \item $\pi: E \rightarrow B$ es un mapeo continuo hacia $B$,
 \item $\oplus$ y $\odot$ son mapeos
 
 \begin{equation}
  \oplus: \underset{p \in b}{\bigcup} \pi^{-1}(p) \times \pi^{-1}(p) \rightarrow E, \quad 
  \odot: R \times E \rightarrow E,
 \end{equation}
\end{itemize}

con $\oplus(\pi^{-1}(p) \times \pi^{-1}(p)) \subseteq \pi^{-1}(p)$, que hace 
cada fibra $\pi^{-1}(p)$ sea un espacio vectorial $n$-dimensional sobre 
$\mathbb{R}$.

\vspace{.3cm}

\definicion Si $\xi = \pi: E \rightarrow B$ es un haz vectorial, definimos 
una métrica riemanniana sobre $\xi$ como una función $\langle \hspace{.1cm}, \hspace{.1cm} \rangle$
que asigna a cada $p \in B$ un producto interno positivo definido $\langle \hspace{.1cm}, \hspace{.1cm} \rangle_p$
en $\pi^{-1}(p)$ y que es continuo en el sentido que de que para dos secciones continuas 
$s_1,s_2: B \rightarrow E$, la función

\begin{equation}
 \langle s_1, s_2 \rangle = p \mapsto \langle s_1(p), s_2(p) \rangle_p
\end{equation}

es también continuo. Si $\xi$ es un haz vectorial $C^\infty$ sobre una variedad 
$C^\infty$ hablamos de métricas riemannianas $C^\infty$. Esto será de importancia pronto.

\vspace{.3cm}

Estamos en posición ahora de enunciar el teorema que nos servirá para 
demostrar lo que queremos.

\vspace{.3cm}

\teorema Sea $\xi = \pi: E \rightarrow M$ un $[C^\infty]$ haz $k$-plano
sobre una variedad $C^\infty$. Entonces $\exists$ una métrica riemanniana 
$[C^\infty]$ sobre $\xi$.

\vspace{.3cm}

\prueba Existe una cobertura abierta y localmente finita $\mathcal{O}$ sobre 
$M$ por los conjuntos $U$ para los cuales existen trivializaciones $[C^\infty]$

\begin{equation}
 t_U: \pi^{-1}(U) \rightarrow U \times \mathbb{R}^k.
\end{equation}

Sobre $U \times \mathbb{R}^k$, existe una obviamente una métrica riemanniana,

\begin{equation}
 \langle (p,a), (p,b)\rangle_p = \langle a, b \rangle. 
\end{equation}

Para $v,w \in \pi^{-1}(p)$, definimos

\begin{equation}
 \langle v, w \rangle_p^U = \langle t_U(v),t_U(w)\rangle_p.
\end{equation}

Entonces $\langle \hspace{.1cm} , \hspace{.1cm} \rangle^U$ es una métrica riemanniana
$[C^\infty]$ para $\xi|U$. Sea $\{\phi_U \}$ una partición de la unidad\footnote{ Una partición 
de la unidad de un espacio $X$ es un conjunto de funciones continuas de $X$ a $[0,1]$ tal que todo punto 
tenga un entorno donde todas las funciones sean cero excepto un número finito de ellas, y que la 
suma de todas las funciones sobre todo el espacio sea idénticamente 1.} 
subordinada a $\mathcal{O}$. Definimos $\langle \hspace{.1cm}, \hspace{.1cm} \rangle$ por 

\begin{equation}
 \langle v, w \rangle_p = \sum_{U \in \mathcal{O}} \phi_U(p)\langle v, w \rangle_p^U, 
 \qquad v,w \in \pi^{-1}(p).
\end{equation}

Entonces $\langle \hspace{.1cm}, \hspace{.1cm} \rangle$ es continua $[C^\infty]$ y 
cada $\langle \hspace{.1cm}, \hspace{.1cm} \rangle_p$ es una función simétrica y 
bilineal sobre $\pi^{-1}(p)$. Para mostrar que es positiva definida, 
notamos que 

\begin{equation}
 \langle v, v \rangle_p =  \sum_{U \in \mathcal{O}} \phi_U(p)\langle v, v \rangle_p^U,
\end{equation}

y cada $\phi_U(p)\langle v, w \rangle_p^U \geq 0$, para algún $U$ la igualdad 
se mantiene.

$\hspace{12cm} \square$

Este teorema cobra una especial significación, y le da solución a este
problema cuando el haz vectorial del teorema que probamos es el es haz 
tangente a una variedad diferenciable $C^\infty$ $M$. En este caso, 
el teorema (ver detalladamente la prueba) nos dice que definir un producto 
interno positivo definido en cada espacio tangente a $M$ define una 
métrica riemanniana $C^\infty$ en $M$. Para esclarecer un poco lo que hemos hecho,
si $(x,U)$ es un sistema de coordenadas en $M$, entonces sobre $U$ podemos 
escribir nuestra métrica riemanniana $\langle \hspace{.1cm}, \hspace{.1cm} \rangle$ como 

\begin{equation}
 \langle \hspace{.1cm} , \hspace{.1cm} \rangle = \sum_{i,j = 0}^n g_{ij}dx^i \otimes dx^j.
\end{equation}

donde las funciones $C^\infty$ $g_{ij}$ satisfacen $g_{ij} = g_{ji}$, debido 
a que $\langle \hspace{.1cm} , \hspace{.1cm} \rangle$ es simétrico, y $\det{a} > 0$, debido a
que $\langle \hspace{.1cm} , \hspace{.1cm} \rangle$ es positivo definido. Claramente la métrica 
riemanniana que hemos definido en $M$ es un tensor covariante de orden 2, en realidad 
son las componentes de un campo tensorial ya que está definido como un tensor 
en el espacio tangente en cada punto de la variedad, pero comúnmente 
solo se habla de él como un tensor.


\section{Problema 4}

En la base inducida por coordenadas, encuentre las componentes de un campo vectorial 
hamiltoniano cuando las coordenadas en el espacio fase, $(x^i,\dots,x^{2n})$, son 
arbitrarias.

\vspace{.3cm}

\underline{Solución:} \vspace{.3cm}

Recordemos que para coordenadas canónicas la forma simpléctica está expresada 
simplemente por 

\begin{equation}
 \gamma = dp_i \wedge dq^i,
\end{equation}

y que al expresar la hamiltoniana del sistema también en coordenadas canónicas 
encontramos que las coordenadas del campo vectorial hamiltoniano $V = V_H$ son 

\begin{equation}
 (V^i,V_i) = \left(\frac{\partial H}{\partial p_i}, - \frac{\partial H}{\partial q^i}\right).
\end{equation}

Con lo cual vemos que el campo vectorial hamiltoniano se identifica con las ecuaciones de 
Hamilton. Como veremos más adelante, pasando a unas coordenadas no canónicas, al expresar 
la forma simpléctica y la hamiltoniana en esas coordenadas, la forma de las ecuaciones de 
movimiento no quedará hamiltoniana, la forma simpléctica no queda invariante ante 
el cambio de coordenadas y por lo tanto no podremos hallar esta identificación de las 
coordenadas del campo vectorial hamiltoniano con las ecuaciones de Hamilton.

\vspace{.3cm}

Para ver esto consideremos unas coordenadas arbitrarias, y hagamos una transformación 
de coordenadas partiendo de las coordenadas canónicas originales $(q,p)$. En el enunciado 
se expresan por $(x^i,\dots,x^{2n})$, pero para simplificar un poco el entendimiento 
y comparación dividimos estas coordenadas, las primeras $n$ coordenadas las denotaremos 
$x^i$ y las segunda $n$ coordenadas las denotaremos $y_i$, donde hemos mantenido la notación 
de índices abajo y arriba aunque debemos recordar que cuando tenemos un sistema 
de coordenadas cualquiera en el espacio de fases, pierde sentido tener índices arriba 
o abajo ya que no son coordenadas locales de la variedad de configuración unas 
y componentes de una forma lineal las otras, pero las mantendremos ya que hace más fácil 
la escritura de las ecuaciones y mantiene el convenio de suma de Einstein, pero 
hay que recordar que ha dejado de tener sentido hablar de covariancia o contravariancia 
ante transformaciones de coordenadas en la variedad de configuración. Tenemos entonces 
que 

\begin{align}
 x^i = x^i(q^i,\dots,q^n,p_1,\dots,p_n), \\
 y_i = y_i(q^i,\dots,q^n,p_1,\dots,p_n),
\end{align}

y 

\begin{align}
 q^i = q^i(x^1,\dots,x^n,y_1,\dots,y_n), \\
 p_i = p_i(x^1,\dots,x^n,y_1,\dots,y_n).
\end{align}

Veamos ahora como se expresa la forma simpléctica ante este cambio de coordenadas

\begin{equation}
 \gamma = dp_i \wedge dq^i = \left(\frac{\partial p_i}{\partial y_j}dy_j + 
 \frac{\partial p_i}{\partial x^j}dx^j\right) \wedge 
 \left(\frac{\partial q^i}{\partial y_k}dy_k + 
 \frac{\partial q^i}{\partial x^k}dx^k\right)
\end{equation}

\begin{align}
\begin{split}
 \gamma &= \left(\frac{\partial p_i}{\partial x^j}\frac{\partial q^i}{\partial x^k} 
 - \frac{\partial p_i}{\partial x^k}\frac{\partial q^i}{\partial x^j}\right)dx^j \wedge dx^k \\
 &+ \left(\frac{\partial p_i}{\partial y_j}\frac{\partial q^i}{\partial y_k} 
 - \frac{\partial p_i}{\partial y_k}\frac{\partial q^i}{\partial y_j}\right)dy_j \wedge dy_k \\
 &+ \left(\frac{\partial p_i}{\partial y_j}\frac{\partial q^i}{\partial x^k} 
 - \frac{\partial p_i}{\partial x^k}\frac{\partial q^i}{\partial y_j}\right)dy_j \wedge dx^k,
\end{split}
\end{align}

donde $j < k$ y donde recordamos que el único modo $\gamma$ en que se reduzca 
a $dy_i \wedge dx^i$ es que la transformación sea canónica, que no es nuestro caso. Estudiemos 
ahora que ocurre al aplicar $\gamma$ a un par de campos vectoriales arbitrarios utilizando 
las coordenadas $(x,y)$, y donde por lo tanto los expresamos como 

\begin{align}
 \begin{split}
  A = A^i\frac{\partial}{\partial x^i} + A_i \frac{\partial}{\partial y_i}, \\
  B = B^i\frac{\partial}{\partial x^i} + B_i \frac{\partial}{\partial y_i}.
 \end{split}
\end{align}

Tenemos entonces 

\begin{align}
\begin{split}
 \gamma[A,B] &= \left[\left(\frac{\partial p_i}{\partial x^j}\frac{\partial q^i}{\partial x^k} 
 - \frac{\partial p_i}{\partial x^k}\frac{\partial q^i}{\partial x^j}\right)dx^j \wedge dx^k\right\zerodel \\
 &+ \left(\frac{\partial p_i}{\partial y_j}\frac{\partial q^i}{\partial y_k} 
 - \frac{\partial p_i}{\partial y_k}\frac{\partial q^i}{\partial y_j}\right)dy_j \wedge dy_k \\
 &+ \left\zerodel\left(\frac{\partial p_i}{\partial y_j}\frac{\partial q^i}{\partial x^k} 
 - \frac{\partial p_i}{\partial x^k}\frac{\partial q^i}{\partial y_j}\right)dy_j \wedge dx^k\right]
 [A,B]
\end{split}
\end{align}

\begin{align}
\begin{split}
 \gamma[A,B] &= \left(\frac{\partial p_i}{\partial x^j}\frac{\partial q^i}{\partial x^k} 
 - \frac{\partial p_i}{\partial x^k}\frac{\partial q^i}{\partial x^j}\right)dx^j \wedge dx^k[A,B] \\
 &+ \left(\frac{\partial p_i}{\partial y_j}\frac{\partial q^i}{\partial y_k} 
 - \frac{\partial p_i}{\partial y_k}\frac{\partial q^i}{\partial y_j}\right)dy_j \wedge dy_k[A,B] \\
 &+ \left(\frac{\partial p_i}{\partial y_j}\frac{\partial q^i}{\partial x^k} 
 - \frac{\partial p_i}{\partial x^k}\frac{\partial q^i}{\partial y_j}\right)dy_j \wedge dx^k[A,B]
\end{split}
\end{align}


\begin{align}
\begin{split}
 \gamma[A,B] &= \left(\frac{\partial p_i}{\partial x^j}\frac{\partial q^i}{\partial x^k} 
 - \frac{\partial p_i}{\partial x^k}\frac{\partial q^i}{\partial x^j}\right)\left|\begin{matrix}
                                                                             dx^j[A] & dx^j[B] \\
                                                                             dx^k[A] & dx^k[B]
                                                                            \end{matrix}\right| \\
 &+ \left(\frac{\partial p_i}{\partial y_j}\frac{\partial q^i}{\partial y_k} 
 - \frac{\partial p_i}{\partial y_k}\frac{\partial q^i}{\partial y_j}\right)\left|\begin{matrix}
                                                                             dy_j[A] & dy_j[B] \\
                                                                             dy_k[A] & dy_k[B]
                                                                            \end{matrix}\right| \\
 &+ \left(\frac{\partial p_i}{\partial y_j}\frac{\partial q^i}{\partial x^k} 
 - \frac{\partial p_i}{\partial x^k}\frac{\partial q^i}{\partial y_j}\right)\left|\begin{matrix}
                                                                             dy_j[A] & dy_j[B] \\
                                                                             dx^k[A] & dx^k[B]
                                                                            \end{matrix}\right|
\end{split}
\end{align}

\begin{align}
\begin{split}
 \gamma[A,B] &= \left(\frac{\partial p_i}{\partial x^j}\frac{\partial q^i}{\partial x^k} 
 - \frac{\partial p_i}{\partial x^k}\frac{\partial q^i}{\partial x^j}\right)\left|\begin{matrix}
                                                                             A^j & B^j \\
                                                                             A^k & B^k
                                                                            \end{matrix}\right| \\
 &+ \left(\frac{\partial p_i}{\partial y_j}\frac{\partial q^i}{\partial y_k} 
 - \frac{\partial p_i}{\partial y_k}\frac{\partial q^i}{\partial y_j}\right)\left|\begin{matrix}
                                                                             A_j & B_j \\
                                                                             A_k & B_k
                                                                            \end{matrix}\right| \\
 &+ \left(\frac{\partial p_i}{\partial y_j}\frac{\partial q^i}{\partial x^k} 
 - \frac{\partial p_i}{\partial x^k}\frac{\partial q^i}{\partial y_j}\right)\left|\begin{matrix}
                                                                             A_j & B_j \\
                                                                             A^k & B^k
                                                                            \end{matrix}\right|
\end{split}
\end{align}

\begin{align}
\begin{split}
 \gamma[A,B] &= \left(\frac{\partial p_i}{\partial x^j}\frac{\partial q^i}{\partial x^k} 
 - \frac{\partial p_i}{\partial x^k}\frac{\partial q^i}{\partial x^j}\right)(A^jB^k - A^kB^j) \\
 &+ \left(\frac{\partial p_i}{\partial y_j}\frac{\partial q^i}{\partial y_k} 
 - \frac{\partial p_i}{\partial y_k}\frac{\partial q^i}{\partial y_j}\right)(A_jB_k - A_kB_j) \\
 &+  \left(\frac{\partial p_i}{\partial y_j}\frac{\partial q^i}{\partial x^k} 
 - \frac{\partial p_i}{\partial x^k}\frac{\partial q^i}{\partial y_j}\right)(A_jB^k - A^kB_j)
\end{split}
\end{align}

\begin{align}
\begin{split}
 \gamma[A,B] &= \frac{\partial p_i}{\partial x^j}\frac{\partial q^i}{\partial x^k}A^jB^k 
 - \frac{\partial p_i}{\partial x^k}\frac{\partial q^i}{\partial x^j}A^jB^k
 - \frac{\partial p_i}{\partial x^j}\frac{\partial q^i}{\partial x^k}A^kB^j 
 + \frac{\partial p_i}{\partial x^k}\frac{\partial q^i}{\partial x^j}A^kB^j\\
 &+ \frac{\partial p_i}{\partial y_j}\frac{\partial q^i}{\partial y_k}A_jB_k 
 - \frac{\partial p_i}{\partial y_k}\frac{\partial q^i}{\partial y_j}A_jB_k
 - \frac{\partial p_i}{\partial y_j}\frac{\partial q^i}{\partial y_k}A_kB_j 
 + \frac{\partial p_i}{\partial y_k}\frac{\partial q^i}{\partial y_j}A_kB_j\\
 &+  \frac{\partial p_i}{\partial y_j}\frac{\partial q^i}{\partial x^k}A_jB^k
 - \frac{\partial p_i}{\partial x^k}\frac{\partial q^i}{\partial y_j}A_jB^k
 -\frac{\partial p_i}{\partial y_j}\frac{\partial q^i}{\partial x^k}A^kB_j 
 + \frac{\partial p_i}{\partial x^k}\frac{\partial q^i}{\partial y_j}A^kB_j
\end{split}
\end{align}

\begin{align}
\begin{split}
 \gamma[A,B] &= \frac{\partial p_i}{\partial x^j}\frac{\partial q^i}{\partial x^k}A^jB^k 
 - \frac{\partial p_i}{\partial x^k}\frac{\partial q^i}{\partial x^j}A^jB^k
 - \frac{\partial p_i}{\partial x^k}\frac{\partial q^i}{\partial x^j}A^jB^k 
 + \frac{\partial p_i}{\partial x^j}\frac{\partial q^i}{\partial x^k}A^jB^k\\
 &+ \frac{\partial p_i}{\partial y_j}\frac{\partial q^i}{\partial y_k}A_jB_k 
 - \frac{\partial p_i}{\partial y_k}\frac{\partial q^i}{\partial y_j}A_jB_k
 - \frac{\partial p_i}{\partial y_k}\frac{\partial q^i}{\partial y_j}A_jB_k 
 + \frac{\partial p_i}{\partial y_j}\frac{\partial q^i}{\partial y_k}A_jB_k\\
 &+  \frac{\partial p_i}{\partial y_j}\frac{\partial q^i}{\partial x^k}A_jB^k
 - \frac{\partial p_i}{\partial x^k}\frac{\partial q^i}{\partial y_j}A_jB^k
 -\frac{\partial p_i}{\partial y_k}\frac{\partial q^i}{\partial x^j}A^jB_k 
 + \frac{\partial p_i}{\partial x^j}\frac{\partial q^i}{\partial y_k}A^jB_k
\end{split}
\end{align}

\begin{align}
\begin{split}
 \gamma[A,B] &= A^j\left(\frac{\partial p_i}{\partial x^j}\frac{\partial q^i}{\partial x^k}B^k
 - \frac{\partial p_i}{\partial x^k}\frac{\partial q^i}{\partial x^j}B^k
 - \frac{\partial p_i}{\partial x^k}\frac{\partial q^i}{\partial x^j}B^k
 + \frac{\partial p_i}{\partial x^j}\frac{\partial q^i}{\partial x^k}B^k\right\zerodel \\
 &\left\zerodel-\frac{\partial p_i}{\partial y_k}\frac{\partial q^i}{\partial x^j}B_k
 +\frac{\partial p_i}{\partial x^j}\frac{\partial q^i}{\partial y_k}B_k\right) + \\
 &A_j\left( \frac{\partial p_i}{\partial y_j}\frac{\partial q^i}{\partial y_k}B_k - 
  \frac{\partial p_i}{\partial y_k}\frac{\partial q^i}{\partial y_j}B_k
  - \frac{\partial p_i}{\partial y_k}\frac{\partial q^i}{\partial y_j}B_k 
  +\frac{\partial p_i}{\partial y_j}\frac{\partial q^i}{\partial y_k}B_k \right\zerodel \\
  &\left\zerodel + \frac{\partial p_i}{\partial y_j}\frac{\partial q^i}{\partial x^k}B^k
  - \frac{\partial p_i}{\partial x^k}\frac{\partial q^i}{\partial y_j}B^k\right)
  \label{eq:Hamilton1}
\end{split}
\end{align}

Apliquemos ahora la diferencial de la hamiltoniana al mismo campo vectorial $A$

% \begin{equation}
%  dH[A] = \left( \frac{\partial H}{\partial q^i}dq^i + \frac{\partial H}{\partial p_i}dp_i\right)
%  \left(A^i\frac{\partial}{\partial x^i} + A_i \frac{\partial}{\partial y_i}\right)
% \end{equation}
% 
% \begin{align}
%  dH[A] = \left[ \frac{\partial H}{\partial q^i}\left(\frac{\partial q^i}{\partial x^i}dx^i + \frac{\partial q^i}{\partial y_i}dy_i  \right)
%  + \frac{\partial H}{\partial p_i}\left(\frac{\partial p_i}{\partial x^i}dx^i + \frac{\partial p_i}{\partial y_i}dy_i  \right)\right]
%  \left[A^i\frac{\partial}{\partial x^i} + A_i \frac{\partial}{\partial y_i}\right]
% \end{align}
% 
% \begin{align*}
%  dH[A] = \left[ \frac{\partial H}{\partial q^i}\frac{\partial q^i}{\partial x^i}dx^i + \frac{\partial H}{\partial q^i}\frac{\partial q^i}{\partial y_i}dy_i
%  + \frac{\partial H}{\partial p_i}\frac{\partial p_i}{\partial x^i}dx^i + \frac{\partial H}{\partial p_i}\frac{\partial p_i}{\partial y_i}dy_i \right]
%  \left[A^i\frac{\partial}{\partial x^i} + A_i \frac{\partial}{\partial y_i}\right]
% \end{align*}

\begin{equation}
 dH[A] = \left( \frac{\partial H}{\partial q^i}dq^i + \frac{\partial H}{\partial p_i}dp_i\right)
 \left(A^j\frac{\partial}{\partial x^j} + A_j \frac{\partial}{\partial y_j}\right)
\end{equation}

\begin{align*}
 dH[A] = \left[ \frac{\partial H}{\partial q^i}\left(\frac{\partial q^i}{\partial x^k}dx^k + \frac{\partial q^i}{\partial y_k}dy_k  \right)
 + \frac{\partial H}{\partial p_i}\left(\frac{\partial p_i}{\partial x^k}dx^k + \frac{\partial p_i}{\partial y_k}dy_k  \right)\right]
 \left[A^j\frac{\partial}{\partial x^j} + A_j \frac{\partial}{\partial y_j}\right]
\end{align*}

\begin{align*}
 dH[A] = \left[ \frac{\partial H}{\partial q^i}\frac{\partial q^i}{\partial x^k}dx^k + \frac{\partial H}{\partial q^i}\frac{\partial q^i}{\partial y_k}dy_k
 + \frac{\partial H}{\partial p_i}\frac{\partial p_i}{\partial x^k}dx^k + \frac{\partial H}{\partial p_i}\frac{\partial p_i}{\partial y_k}dy_k \right]
 \left[A^j\frac{\partial}{\partial x^j} + A_j \frac{\partial}{\partial y_j}\right]
\end{align*}

\begin{align}
 dH[A] = \frac{\partial H}{\partial q^i}\frac{\partial q^i}{\partial x^j}A^j + \frac{\partial H}{\partial q^i}\frac{\partial q^i}{\partial y_j}A_j
 + \frac{\partial H}{\partial p_i}\frac{\partial p_i}{\partial x^j}A^j + \frac{\partial H}{\partial p_i}\frac{\partial p_i}{\partial y_j}A_j 
\end{align}

\begin{equation}
 dH[A] =  \frac{\partial H}{\partial x^j}A^j + \frac{\partial H}{\partial y_j}A_j
 + \frac{\partial H}{\partial x^j}A^j + \frac{\partial H}{\partial y_j}A_j
\end{equation}

\begin{equation}
 dH[A] =  2\frac{\partial H}{\partial x^j}A^j + 2\frac{\partial H}{\partial y_j}A_j
 \label{eq:Hamilton2}
\end{equation}

Comparando las ecuaciones \eqref{eq:Hamilton1} con \eqref{eq:Hamilton2} vemos entonces 
que las componentes del campo vectorial hamiltoniano en las coordenadas arbitrarias será 


\begin{align*}
 &\left(\frac{\partial p_i}{\partial x^j}\frac{\partial q^i}{\partial x^k}B^k
 - \frac{\partial p_i}{\partial x^k}\frac{\partial q^i}{\partial x^j}B^k
 - \frac{\partial p_i}{\partial x^k}\frac{\partial q^i}{\partial x^j}B^k
 + \frac{\partial p_i}{\partial x^j}\frac{\partial q^i}{\partial x^k}B^k\right\zerodel
 \left\zerodel-\frac{\partial p_i}{\partial y_k}\frac{\partial q^i}{\partial x^j}B_k
 +\frac{\partial p_i}{\partial x^j}\frac{\partial q^i}{\partial y_k}B_k,\right\zerodel \\
 &\left\zerodel \frac{\partial p_i}{\partial y_j}\frac{\partial q^i}{\partial y_k}B_k - 
  \frac{\partial p_i}{\partial y_k}\frac{\partial q^i}{\partial y_j}B_k
  - \frac{\partial p_i}{\partial y_k}\frac{\partial q^i}{\partial y_j}B_k 
  +\frac{\partial p_i}{\partial y_j}\frac{\partial q^i}{\partial y_k}B_k \right\zerodel 
  \left\zerodel + \frac{\partial p_i}{\partial y_j}\frac{\partial q^i}{\partial x^k}B^k
  - \frac{\partial p_i}{\partial x^k}\frac{\partial q^i}{\partial y_j}B^k\right) = \\ 
  &\hspace{5cm} (2\frac{\partial H}{\partial x^j}, 2\frac{\partial H}{\partial y_j})
\end{align*}

Con lo cual vemos el alto precio que se paga por no utilizar coordenadas canónicas.

\section{Problema 5}

Dado un sistema canónico de coordenadas $(q,p)$ para un espacio de fase donde las $q$ 
son coordenadas de la variedad de configuración, demuestre que la diferencial exterior 
de la uno-forma diferencial 

$$
\omega = \sum_i p_idq^i
$$

es una dos-forma diferencial NO DEGENERADA y por tanto simpléctica.

\vspace{.3cm}

\underline{Solución:} \vspace{.3cm}

Utilizando la definición de derivación exterior obtenemos que 

\begin{equation}
 \gamma = d\omega = d(p_i) \wedge dq^i = dp_i \wedge dq^i
\end{equation}

y sabemos que la derivación exterior de una uno-forma diferencial es una dos forma 
diferencial. Podemos notar que por definición $\gamma$ es cerrada, es decir que 
$d\gamma = 0$, debido a que la formamos tomando la diferencial de una uno-forma 
diferencial. Para probar que es no degenerada, debemos probar que dado cualquier 
campo vectorial distinto de cero, existe otro tal que, al aplicar $\gamma$ a esa pareja, 
el resultado es una función distinta de cero; de manera equivalente podemos decir 
que será no degenerada si el único modo de que dado un campo vectorial distinto de 
cero al aplicar $\gamma$ sea cero, es el que el otro campo al que se le aplica 
es cero; o de forma equivalente que el determinante que define la aplicación 
sea no singular. 

\vspace{.3cm}

Utilizando la definición de producto exterior entre dos uno-formas diferenciales, vemos 
que al aplicarla a dos campos vectoriales arbitrarios $X$ y $Y$ que expresados 
en coordenadas canónicas son 

\begin{align}
 X &= X^i\frac{\partial}{\partial q^i} + X_i\frac{\partial}{\partial p_i}, \\
 Y &= Y^i\frac{\partial}{\partial q^i} + Y_i\frac{\partial}{\partial p_i},
\end{align}

tenemos entonces que

\begin{equation}
 \gamma[X,Y] = \left|\begin{matrix}
                dp_i[X] & dp_i[Y] \\
                dq^i[X] & dq^i[Y]
               \end{matrix}\right|
\end{equation}

\begin{equation}
 \gamma[X,Y] = \left|\begin{matrix}
                X_i & Y_i \\
                X^i & Y^i
               \end{matrix}\right|
\end{equation}

\begin{equation}
 \gamma[X,Y] = X_i Y^i - X^iY_i
\end{equation}

Por la forma de esta ecuación debe ser evidente que si $X$ no es cero, entonces 
siempre existe una $Y$ tal que $\gamma[X,Y] \ne 0$. Con lo cual ya podríamos decir 
que la dos-forma es no degenerada, y como vimos al comienzo también es cerrada 
y por lo tanto simpléctica. 

\vspace{.3cm}

Para convencernos de este resultado, estudiemos el único caso en que $\gamma[X,Y] = 0$, 
que debería ser el caso en que $\gamma[X,0]$, entonces 

\begin{equation}
 \gamma[X,0] = \left|\begin{matrix}
                dp_i[X] & dp_i[0] \\
                dq^i[X] & dq^i[0]
               \end{matrix}\right|
\end{equation}

\begin{equation}
 \gamma[X,Y] = \left|\begin{matrix}
                X_i & 0 \\
                X^i & 0
               \end{matrix}\right|
\end{equation}

\begin{equation}
 \gamma[X,0] = 0.
\end{equation}

Y lo mismo se debe cumplir para $\gamma[0,X]$. Con lo cual demostramos también los otros 
criterios para la no degeneración. Estas pruebas completan la demostración que queríamos 
hacer, y sabemos que $\gamma$ es simpléctica debido a que por definición \cite{rasband,abraham,arnold},
una forma simpléctica es una dos-forma diferencial cerrada y no degenerada sobre 
una variedad diferencial de dimensión par (y vemos que esto es así ya que tenemos 
$n$ $q$'s y $n$ $p$'s), es decir 

\begin{equation}
 d\gamma = 0 \quad \text{y} \quad \forall \xi \ne 0 \quad \exists \eta: \gamma[\xi,\eta] \ne 0 
 \quad (\xi,\eta \in TM_x).
\end{equation}

que es exactamente lo que acabamos de demostrar.


\begin{thebibliography}{10}
\bibitem{rasband}
S. Rasband, \emph{Dynamics}, John Wiley \& Sons, 1983.
\bibitem{spivak}
M. Spivak, \emph{A comprehensive introduction to Differential Geometry}, vol. 1, 
3ra edición, Publish or Perish, 1999.
\bibitem{abraham}
 R. Abraham y J. Marsden, \emph{Foundations of Mechanics}, 2da edición, Addison-Wesley,
 1978.
 \bibitem{arnold}
V. Arnold, \emph{Mathematical Methods of Classical Mechanics}, 2da edición, Springer-Verlang, 
1989.
 
\end{thebibliography}


\end{document}