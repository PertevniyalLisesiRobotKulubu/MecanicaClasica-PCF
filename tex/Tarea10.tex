\documentclass[a4paper,10pt]{article}
\usepackage[utf8]{inputenc}
\usepackage[spanish]{babel}
\usepackage[affil-it]{authblk}
\usepackage{enumerate}
\usepackage{graphicx}
\usepackage{hyperref}
\usepackage{amsmath}
\usepackage{amssymb}
\usepackage{cancel}
\usepackage[usenames, dvipsnames]{color}
\usepackage{caption}
\usepackage{subcaption} %Multiple images
\usepackage{multicol} % Multiple columns
\usepackage{float}
\usepackage{cleveref}
 \usepackage{relsize} %bigger math symbols
\usepackage[margin=1.4in]{geometry}
\usepackage[labelfont=bf]{caption}
\usepackage[titletoc,toc,title]{appendix}
\usepackage{enumitem}
\usetikzlibrary{calc}
\numberwithin{equation}{section}

%Appendices in spanish
\renewcommand{\appendixname}{Ap\'endices}
\renewcommand{\appendixtocname}{Ap\'endices}
\renewcommand{\appendixpagename}{Ap\'endices}

%Zero delimiter
\newcommand{\zerodel}{.\kern-\nulldelimiterspace}

%Columns separation
\setlength{\columnsep}{1cm}

%Indentation
\setlength{\parindent}{0ex}

%Multiple References

\crefrangelabelformat{equation}{(#3#1#4--#5\crefstripprefix{#1}{#2}#6)}

\usepackage{xparse}
\ExplSyntaxOn
\NewDocumentCommand{\mref}{m}{\quinn_mref:n {#1}}
\seq_new:N \l_quinn_mref_seq
\cs_new:Npn \quinn_mref:n #1
 {
  \seq_set_split:Nnn \l_quinn_mref_seq { , } { #1 }
  \seq_pop_right:NN \l_quinn_mref_seq \l_tmpa_tl
  ( % print the left parenthesis
  \seq_map_inline:Nn \l_quinn_mref_seq
    { \ref{##1},\nobreakspace } % print the first references
  \exp_args:NV \ref \l_tmpa_tl 
  ) 
 }
\ExplSyntaxOff


%Boxes

\newcommand*{\boxcolor}{blue}
\makeatletter
\renewcommand{\boxed}[1]{\textcolor{\boxcolor}{%
\tikz[baseline={([yshift=-1ex]current bounding box.center)}] \node [rectangle, minimum width=1ex,rounded corners,draw] {\normalcolor\m@th$\displaystyle#1$};}}
 \makeatother

%Constantes
\newcommand{\euler}{\mathrm{e}}
\newcommand{\im}{i}

%Lemas, teoremas, definiciones y pruebas
\newcommand{\definicion}{\textbf{Definición: }}
\newcommand{\lema}{\textbf{Lema: }}
\newcommand{\teorema}{\textbf{Teorema: }}
\newcommand{\prueba}{\textbf{Prueba: }}
\newcommand{\proposicion}{\textbf{Proposición: }}
\newcommand{\corolario}{\textbf{Corolario: }}


%opening
\title{Mecánica Clásica Tarea \# 10}
\author{Favio Vázquez\thanks{Correo: favio.vazquezp@gmail.com}}\affil{Instituto de Ciencias Nucleares. Universidad Nacional Autónoma de México.}
\date{}

\begin{document}

\makeatletter
\def\@maketitle{%
  \newpage
  \null
  \vskip 2em%
  \begin{center}%
  \let \footnote \thanks
    {\Large\bfseries \@title \par}%
    \vskip 1.5em%
    {\normalsize
      \lineskip .5em%
      \begin{tabular}[t]{c}%
        \@author
      \end{tabular}\par}%
    \vskip 1em%
    {\normalsize \@date}%
  \end{center}%
  \par
  \vskip 1.5em}
\makeatother

\maketitle

\section{Problema 1}

Utilizando la formulación hamiltoniana de la mecánica, encuentre las ecuaciones de 
movimiento de un péndulo doble de masas y longitudes iguales.

\vspace{.3cm}

\underline{Solución:} \vspace{.3cm}

\section{Problema 2}

Una partícula de masa $m$ se mueve en una dimensión y la hamiltoniana del sistema 
es 

$$
H(q,p) = \frac{p^2}{2m}e^{-\frac{q}{a}}
$$

Encuentre las ecuaciones de movimiento y su solución. Considerando únicamente los casos 
en que $p>0$, ¿Cuál sería la fuerza que debería actuar sobre la partícula en una visión
newtoniana del este sistema? ¿Qué sentido le puede dar en este caso a las relaciones entre 
la energía total, la energía cinética y la hamiltoniana como integral de movimiento?

\vspace{.3cm}

\underline{Solución:} \vspace{.3cm}

\section{Problema 3}

Considere la hamiltoniana en dos grados de libertad 

$$
H = q^1p_1 - q^2p_2 - a(q^1)^2 + b(q^2)^2
$$

demuestre que las tres funciones 

$$
f_1 = (p_2 - bq^2)/q^1, \quad f_2 = q^1q^2, \quad f_3=q^1e^{-t},
$$

son constantes de movimiento. ¿Son todas integrales de movimiento? ¿Son independientes?, 
¿Cuáles son los paréntesis de Poisson entre ellas?, ¿Habrá más integrales de movimiento 
independientes?; de haber más, ¿podrán ser los paréntesis de Poisson entre todas 
las integrales de movimiento iguales a cero?

\vspace{.3cm}

Esta hamiltoniana es muy rara, encuentre esta rareza y descríbala.

\vspace{.3cm}

\underline{Solución:} \vspace{.3cm}

\section{Problema 4}

Demuestre que un sistema es hamiltoniano sí y solo sí

$$
\frac{d}{dt}\{f,g\} = \{\dot{f},g\} + \{f,\dot{g}\}
$$

\vspace{.3cm}

\underline{Solución:} \vspace{.3cm}

\section{Problema 5}

Demuestre la identidad de Jacobi para los paréntesis de Poisson.

\vspace{.3cm}

\underline{Solución:} \vspace{.3cm}

\end{document}
