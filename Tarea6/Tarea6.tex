\documentclass[a4paper,10pt]{article}
\usepackage[utf8]{inputenc}
\usepackage[spanish]{babel}
\usepackage[affil-it]{authblk}
\usepackage{enumerate}
\usepackage{graphicx}
\usepackage{hyperref}
\usepackage{amsmath}
\usepackage{amssymb}
\usepackage{cancel}
\usepackage[usenames, dvipsnames]{color}
\usepackage{tikz}
\usepackage{multimedia}
\usepackage{subcaption} %Multiple images
\usepackage{multicol} % Multiple columns
\usepackage{float}
\usepackage{cleveref}
\usepackage[margin=1.4in]{geometry}
\usepackage[labelfont=bf]{caption}
\usetikzlibrary{calc}
\numberwithin{equation}{section}

%Columns separation
\setlength{\columnsep}{1cm}

%Indentation
\setlength{\parindent}{0ex}

%Multiple References

\usepackage{xparse}
\ExplSyntaxOn
\NewDocumentCommand{\mref}{m}{\quinn_mref:n {#1}}
\seq_new:N \l_quinn_mref_seq
\cs_new:Npn \quinn_mref:n #1
 {
  \seq_set_split:Nnn \l_quinn_mref_seq { , } { #1 }
  \seq_pop_right:NN \l_quinn_mref_seq \l_tmpa_tl
  ( % print the left parenthesis
  \seq_map_inline:Nn \l_quinn_mref_seq
    { \ref{##1},\nobreakspace } % print the first references
  \exp_args:NV \ref \l_tmpa_tl 
  ) 
 }
\ExplSyntaxOff


%Boxes

\newcommand*{\boxcolor}{blue}
\makeatletter
\renewcommand{\boxed}[1]{\textcolor{\boxcolor}{%
\tikz[baseline={([yshift=-1ex]current bounding box.center)}] \node [rectangle, minimum width=1ex,rounded corners,draw] {\normalcolor\m@th$\displaystyle#1$};}}
 \makeatother

%Constantes
\newcommand{\euler}{\mathrm{e}}
\newcommand{\im}{i}

%Lemas, teoremas, definiciones y pruebas
\newcommand{\definicion}{\textbf{Definición: }}
\newcommand{\lema}{\textbf{Lema: }}
\newcommand{\teorema}{\textbf{Teorema: }}
\newcommand{\prueba}{\textbf{Prueba: }}


%opening
\title{Mecánica Clásica Tarea \# 6}
\author{Favio Vázquez\thanks{Correo: favio.vazquezp@gmail.com}}\affil{Instituto de Ciencias Nucleares. Universidad Nacional Autónoma de México.}
\date{}

\begin{document}

\makeatletter
\def\@maketitle{%
  \newpage
  \null
  \vskip 2em%
  \begin{center}%
  \let \footnote \thanks
    {\Large\bfseries \@title \par}%
    \vskip 1.5em%
    {\normalsize
      \lineskip .5em%
      \begin{tabular}[t]{c}%
        \@author
      \end{tabular}\par}%
    \vskip 1em%
    {\normalsize \@date}%
  \end{center}%
  \par
  \vskip 1.5em}
\makeatother

\maketitle

\section{Problema 1}

Una partícula de masa $m$ se mueve constreñida a la superficie de un paraboloide de 
revolución que tiene su abertura hacia arriba en presencia del campo de la gravedad. 
Calcule las fuerzas de constricción.

\vspace{.3cm}

\underline{Solución:} \vspace{.3cm}

\section{Problema 2}

Considere la lagrangiana de una partícula de masa $m$ totalmente libre en el 
espacio tridimensional desde la perspectiva de un sistema inercial y desde la 
perspectiva de un sistema que está en rotación respecto al inercial, y en 
traslación respecto a la partícula (estos movimientos no son necesariamente uniformes). 
Establezca las ecuaciones de Lagrange en ambos sistemas y, a partir de estas ecuaciones, 
encuentre las expresiones para las fuerzas ficticias; identifique en particular 
las fuerzas: centrífuga, de Coriolis, y de Euler.

\vspace{.3cm}

\underline{Solución:} \vspace{.3cm}

\section{Problema 3}

Un girocompás es un instrumento que consisten de un cuerpo rígido simétrico, 
$(I_1 = I_2 \ne I_3)$ cuyo eje de simetría está constreñido a permanecer sobre 
un plano horizontal en la superficie de la Tierra, la que, naturalmente está en 
rotación en torno a su eje con un período de 24 horas. Suponga que un girocompás 
se encuentra en un punto de la Tierra de latitud $\phi$ y se pone, inicialmente, 
en rotación en torno a su eje de simetría con una velocidad angular $\omega_3$ cuando 
dicho eje apunta en una dirección arbitraria. Calcule una lagrangiana para este 
sistema. Demuestre que la velocidad de rotación en torno al eje de simetría, $\omega_3$,
permanece constante. Demuestre que si $\omega_3 > (I_1,I_3)\omega_0\cos{\phi}$ entonces 
el eje de simetría oscilará, de manera estable, en torno a la dirección norte-sur 
($\omega_0$ es la velocidad angular de la Tierra).

\vspace{.3cm}

\underline{Solución:} \vspace{.3cm}


\end{document}
