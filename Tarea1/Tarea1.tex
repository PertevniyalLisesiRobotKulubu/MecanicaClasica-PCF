\documentclass[a4paper,10pt]{article}
\usepackage[utf8]{inputenc}
\usepackage[spanish]{babel}
\usepackage[affil-it]{authblk}
\usepackage{enumerate}
\usepackage{graphicx}
\usepackage{hyperref}
\usepackage{amsmath}
\usepackage{amssymb}
\usepackage{cancel}
\usepackage{tikz}
\usepackage{cleveref}
\usetikzlibrary{calc}
\numberwithin{equation}{section}

%Multiple References

\usepackage{xparse}
\ExplSyntaxOn
\NewDocumentCommand{\mref}{m}{\quinn_mref:n {#1}}
\seq_new:N \l_quinn_mref_seq
\cs_new:Npn \quinn_mref:n #1
 {
  \seq_set_split:Nnn \l_quinn_mref_seq { , } { #1 }
  \seq_pop_right:NN \l_quinn_mref_seq \l_tmpa_tl
  ( % print the left parenthesis
  \seq_map_inline:Nn \l_quinn_mref_seq
    { \ref{##1},\nobreakspace } % print the first references
  \exp_args:NV \ref \l_tmpa_tl 
  ) 
 }
\ExplSyntaxOff


%Boxes

\newcommand*{\boxcolor}{blue}
\makeatletter
\renewcommand{\boxed}[1]{\textcolor{\boxcolor}{%
\tikz[baseline={([yshift=-1ex]current bounding box.center)}] \node [rectangle, minimum width=1ex,rounded corners,draw] {\normalcolor\m@th$\displaystyle#1$};}}
 \makeatother

%Constantes
\newcommand{\euler}{\mathrm{e}}
\newcommand{\im}{i}

%opening
\title{Mecánica Clásica Tarea \# 1}
\author{Favio Vázquez\thanks{Correo: favio.vazquezp@gmail.com}}\affil{Instituto de Física. Universidad Nacional Autónoma de México}
\date{}

\begin{document}

\makeatletter
\def\@maketitle{%
  \newpage
  \null
  \vskip 2em%
  \begin{center}%
  \let \footnote \thanks
    {\Large\bfseries \@title \par}%
    \vskip 1.5em%
    {\normalsize
      \lineskip .5em%
      \begin{tabular}[t]{c}%
        \@author
      \end{tabular}\par}%
    \vskip 1em%
    {\normalsize \@date}%
  \end{center}%
  \par
  \vskip 1.5em}
\makeatother

\maketitle

\section{Problema 1}
Encontrar el movimiento de un oscilador armónico amortiguado
con un coeficiente de amortiguamiento de $\gamma = \omega_{0}/3$ 
($\omega_{0}/3$ frecuencia natural de l oscilador) si a $t=0$ está 
en reposo en su punto de equilibrio y a partir de ese instante se le
aplica una fuerza dada por\footnote{Hemos modificado el nombre de las contantes originales por conveniencia} 
$$F= G\sin{\omega_{0}t}+B\sin{3\omega_{0}t}$$

\underline{Solución:}

\vspace{.3cm}

Nos encontramos con el caso de un oscilador armónico amortiguado al cual se le
aplica una fuerza de conducción. Si asumimos que la fuerza de amortiguamiento es
una función lineal de la velocidad del oscilador, entonces la ecuación que debemos
resolver tiene la forma

\begin{equation}
 m\ddot{x} + b \dot{x} + kx = G \sen{\omega_0 t} + H \sen{3\omega_0 t}
 \label{eq:ODEinicialOscilador}
\end{equation}

sujeta a las condiciones iniciales

\begin{equation}
 x(0) = 0 \qquad \qquad \dot{x}(0) = 0
 \label{eq:condInicialesOscilador}
\end{equation}

Podemos escribir la ecuación (\ref{eq:ODEinicialOscilador}) de una forma más conveniente,

\begin{equation}
 \ddot{x} + 2\gamma \dot{x} + \omega_0^2 x = A \sen{\omega_0 t} + B \sen{3\omega_0 t}
 \label{eq:ODEmejoradaOscilador}
\end{equation}

Donde $A = G/m$, $B = H/m$ y hemos definido\footnote{Esta definición la hemos tomado de libros clásicos de mecánica \cite{marion,taylor}} 
a  $\gamma \equiv b/2m$, y representa el coeficiente de amortiguado del oscilador 
y $\omega_0^2 \equiv \sqrt{k/m}$ es la frecuencia natural del oscilador. 

\vspace{.3cm}

La solución a la ecuación (\ref{eq:ODEmejoradaOscilador}) consta de dos partes,
$x_h(t)$ a la que llamamos solución homogénea, la cual es solución de (\ref{eq:ODEmejoradaOscilador})
si hacemos el lado derecho de la misma cero, y una solución particular $x_p(t)$, que 
reproduce el lado derecho. 

\vspace{.3cm}

Para la solución homogénea tenemos 

\begin{equation}
 \ddot{x} + 2 \gamma \dot{x} + \omega_0^2 x = 0 
 \label{eq:ODEOsciladorHomogenea}
\end{equation}

Para resolver esta ecuación diferencial ordinaria de segundo orden seguiremos los pasos
establecidos en \cite{zill}, con algunos trucos utilizados por los libros de mecánica,
comenzamos por conjeturar que las soluciones de (\ref{eq:ODEOsciladorHomogenea}) tienen 
la forma 

$$
x(t) = \euler^{rt}
$$

entonces

\begin{align}
 \begin{split}
  %
  x(t) &= \euler^{rt} \\
  %
  \dot{x}(t) &= r\euler^{rt} \\
  %
  \ddot{x}(t) &= r^2\euler^{rt} 
  %
  \label{eq:solucionesExpHomogen}
 \end{split}
\end{align}

Sustituyendo (\ref{eq:solucionesExpHomogen}) en (\ref{eq:ODEOsciladorHomogenea}) obtenemos

\begin{equation}
 r^2 \euler^{rt} + 2r\gamma \euler^{rt} + \omega_0^2 \euler{rt} = 0
 \label{eq:CasiCaracteristica}
\end{equation}

$$
\therefore \left(r^2 + 2r\gamma + \omega_0^2\right) \euler^{rt} = 0
$$

Debido a que $\euler^{rt} \neq 0$ para cualquier $x \in \mathbb{R}$, entonces

\begin{equation}
  r^2 + 2r\gamma + \omega_0^2 = 0
 \label{eq:ecuacionCaracteristica1}
\end{equation}

Haciendo uso de la conocida relación para obtener las raíces de una ecuación
algebraica lineal de segundo orden

$$
r_{1,2} = \frac{-b \pm \sqrt{b^2-4ac}}{2a}
$$

tenemos que 

\begin{align}
\begin{split}
%
r_1 &= - \gamma + \sqrt{\gamma^2 - \omega_0^2} \\
%
r_2 &= - \gamma - \sqrt{\gamma^2 - \omega_0^2}
%
\end{split}
\end{align}

Debido a que $r_1$ y $r_2$ son linealmente independientes encontramos que la solución
a la ecuación homogénea es 

\begin{equation}
 x_h(t) = C_1 \euler^{r_1 t} + C_2 \euler^{r_2 t} = \euler^{-\gamma t} (C_1 \euler^{\sqrt{\gamma^2 - \omega_0^2}t}
 + C_2 \euler^{-\sqrt{\gamma^2 - \omega_0^2}t})
\end{equation}

Que podemos escribir como

\begin{equation}
 x_h(t) = C_h \euler^{-\gamma t} \sen{(\omega_i t + \phi)}
\end{equation}

Donde $\omega_i = \sqrt{\gamma^2 - \omega_0^2}$ y $\phi$ es la fase de la oscilación.


Para encontrar la solución particular utilizaremos el método de los coeficientes indeterminados, 
en el cual conjeturamos la forma de $x_p (t)$ motivados por los tipos de funciones linealmente
independientes que pueden construir el término no homogéneo de la ecuación, no demostraremos
el método, solo lo utilizaremos, puede encontrarse una demostración completa y comprobar
su validez en cualquier libro ecuaciones diferenciales modernas como \cite{zill}. Entonces
asumiendo que $x_p (t)$ tiene la forma

\begin{equation}
 x_p (t) = C \sen{\omega_0t} + D \cos{\omega_0t} + F \sen{3 \omega_0t} + E \cos{3 \omega_0t}
 \label{eq:particular1}
\end{equation}

Derivando (\ref{eq:particular1}) con respecto al tiempo y luego derivando ese resultado de nuevo
con respecto la tiempo obtenemos

\begin{align}
 \begin{split}
  % 
  \dot{x}_p (t) = C \omega_0 \cos{\omega_0t} - D\omega_0 \sen{\omega_0t} + 
  3F\omega_0 \cos{3 \omega_0t} - 3E\omega_0 \sen{3 \omega_0t} \\
  %
  \ddot{x}_p (t) = - C\omega_0^2 \sen{\omega_0t} - D\omega_0^2 \cos{\omega_0t} - 
  9F\omega_0^2 \sen{3 \omega_0t} - 9E\omega_0^2 \cos{3 \omega_0t} 
  %
  \label{eq:particular2}
 \end{split}
\end{align}

Sustituyendo (\ref{eq:particular1}) y (\ref{eq:particular2}) en (\ref{eq:ODEmejoradaOscilador}),
obtenemos

\begin{align*}
%
 - C\omega_0^2 \sen{\omega_0t} - D\omega_0^2 \cos{\omega_0t} - 
  9F\omega_0^2 \sen{3 \omega_0t} - 9E\omega_0^2 \cos{3 \omega_0t} - \\
  %
  2\gamma C \omega_0 \cos{\omega_0t} - 2\gamma D\omega_0 \sen{\omega_0t} + 
  6\gamma F\omega_0 \cos{3 \omega_0t} - 6\gamma E\omega_0 \sen{3 \omega_0t} + \\
  %
  C\omega_0^2 \sen{\omega_0t} + D\omega_0^2 \cos{\omega_0t} + F\omega_0^2 
  \sen{3 \omega_0t} + E\omega_0^2 \cos{3 \omega_0t} \\
  %
  = A \sen{\omega_0t} + B \sen{3\omega_0t}
\end{align*}

Pero según el enunciado del problema $\gamma = \omega_0/3$, entonces

\begin{align}
 \begin{split}
- C\omega_0^2 \sen{\omega_0t} - D\omega_0^2 \cos{\omega_0t} - 
  9F\omega_0^2 \sen{3 \omega_0t} - 9E\omega_0^2 \cos{3 \omega_0t} - \\
  %
  \frac{2}{3} C \omega_0^2 \cos{\omega_0t} - \frac{2}{3} \omega_0^2 D \sen{\omega_0t} + 
  2 \omega_0^2 F \cos{3 \omega_0t} - 2 \omega_0^2 E \sen{3 \omega_0t} + \\
  %
  C\omega_0^2 \sen{\omega_0t} + D\omega_0^2 \cos{\omega_0t} + F\omega_0^2 
  \sen{3 \omega_0t} + E\omega_0^2 \cos{3 \omega_0t} \\
  %
  = A \sen{\omega_0t} + B \sen{3\omega_0t}
  \end{split}
\end{align}

\begin{align}
 \begin{split}
  \sen{\omega_0t}\left[\cancel{- C\omega_0^2} - \frac{2}{3} \omega_0^2 D  + \cancel{C\omega_0^2}  \right] \\
  %
  \cos{\omega_0t}\left[\cancel{- D\omega_0^2} + \frac{2}{3} C \omega_0^2 + \cancel{D\omega_0^2} \right] \\
  %
  \sen{3\omega_0t} \left[ - 9F\omega_0^2 - 2 \omega_0^2 E + F\omega_0^2 \right] \\
  %
  \cos{3\omega_0t} \left[  - 9E\omega_0^2 + 2 \omega_0^2 + E\omega_0^2 \right] \\
  %
  = A \sen{\omega_0t} + B \sen{3\omega_0t} 
  %
  \end{split}
\end{align}

Entonces tenemos que, por igualación

\begin{align*}
 %
 A &= - \frac{2}{3} \omega_0^2 D \Rightarrow \boxed{D = - \frac{3A}{2\omega_0^2}} \\
 %
 0 &= - \frac{2}{3} \omega_0^2 C \Rightarrow \boxed{C = 0} \\
 %
 0 &= -8E\omega_0^2 + 2F\omega_0^2 \Rightarrow \boxed{F = 4E} \\
 %
 B &= -8F \omega_0^2 - 2E \omega_0^2 \Rightarrow \boxed{E = - \frac{B}{34\omega_0^2}}
\end{align*}

Entonces la solución particular es

\begin{equation}
 x_p(t) = - \frac{3A}{2\omega_0^2} \cos{\omega_0t} - \frac{2}{17} 
 \frac{B}{\omega_0^2} \sen{3\omega_0t} - \frac{B}{34\omega_0^2} \cos{3\omega_0t}
\end{equation}

La solución final a la ecuación es entonces, donde por simpleza asumiremos por ahora que $\phi = 0$

\begin{equation}
 x(t) = C_h \euler^{-\frac{\omega_0}{3} t} \sen{\omega_i t} - \frac{3A}{2\omega_0^2} \cos{\omega_0t} - \frac{2}{17} 
 \frac{B}{\omega_0^2} \sen{3\omega_0t} - \frac{B}{34\omega_0^2} \cos{3\omega_0t}
\end{equation}

Pero $x(0) = 0$, entonces

\begin{equation}
 0 = \frac{3A}{2\omega_0^2} - \frac{B}{34\omega_0^2} \Rightarrow \frac{3A}{2} = \frac{B}{34} \Rightarrow \boxed{A = \frac{B}{51}}
\end{equation}

Y $\dot{x}(0) = 0$, calculando la derivada de $x(t)$

\begin{align*}
 \dot{x}(t) =  &-\frac{\omega_0^2}{3} C_h \euler^{-\frac{\omega_0}{3} t} \sen{\omega_i t}
 + C_h \euler^{-\frac{\omega_0}{3} t} \cos{\omega_0t} + \frac{3A\omega_0}{2\omega_0^2} \sen{\omega_0t} \\ %
 &- \frac{6}{17}\frac{B\omega_0}{\omega_0^2} \cos{3\omega_0t} + \frac{3B\omega_0}{34\omega_0^2} \sen{3\omega_0t}
\end{align*}

\begin{align*}
 \dot{x}(t) = &- \frac{\omega_0^2}{3} C_h \euler^{-\frac{\omega_0}{3} t} \sen{\omega_i t}
 + C_h \euler^{-\frac{\omega_0}{3} t} \cos{\omega_0t} + \frac{3A}{2\omega_0} \sen{\omega_0t} \\ %
 &- \frac{6}{17}\frac{B}{\omega_0} \cos{3\omega_0t} + \frac{3B}{34\omega_0} \sen{3\omega_0t}
\end{align*}

Entonces,

\begin{align*}
0 &= C_h - \frac{6}{17}\frac{B}{\omega_0} \\
%
C_h &= \frac{6B}{17}
%
\end{align*}

Entonces la solución final al problema es, donde debido a que $\gamma < \omega_0$ hemos asumido que $\omega_i \approx \omega_0$

\begin{align*}
 \boxed{x(t) = \frac{6B}{17} \euler^{-\frac{\omega_0}{3} t} \sen{(\omega_0 t + \phi)}
 - \frac{B}{34\omega_0^2} \cos{\omega_0t} - \frac{2}{17}\frac{B}{\omega_0^2} \sen{3\omega_0t} - \frac{B}{34\omega_0^2} \cos{3\omega_0t}}
\end{align*}



\vspace{.3cm}

\section{Problema 2}

Considere un oscilador armónico con un pequeño amortiguamiento.
Muestre que el cambio en la energía durante un período es $2T/\tau$,
donde $T$ el período de oscilador sin amortiguar y $\tau$ es el tiempo
que tarda la amplitud en reducirse por un factor $1/e=1/2,718$.

\vspace{.3cm}

\underline{Solución:} \vspace{.3cm}

Nos encontramos de nuevo con el caso de un oscilador armónico con un pequeño amortiguamiento,
en este caso sin fuerzas de conducción, para este tipo de oscilador la ecuación diferencial
que lo gobierna tiene la forma 

\begin{equation}
 \ddot{x} + 2 \gamma \dot{x} + \omega_0^2 x = 0 
 \label{eq:ODEAmortiguado}
\end{equation}

Donde hemos definido el factor de amortiguamiento como $\gamma = b/2m$, y $\omega_0t$ 
es la frecuencia natural del oscilador. Esta ecuación diferencial lineal homogénea tiene
por solución, definiendo $\omega_i \equiv \sqrt{\omega_0^2 - \gamma^2}$

\begin{equation}
 x(t) = \euler^{-\gamma t} (A sin(\omega_i t + \delta)
 \label{eq:SolAmortiguado}
\end{equation}

Donde $\delta$ es la fase del oscilador. Podemos ver de (\ref{eq:SolAmortiguado}) que
la presencia del factor exponencial real $\euler^{-\gamma t}$ la amplitutud $A$ 
del oscilador amortiguado decrece con el tiempo. Podemos definir el período del oscilador
amortiguado como

\begin{equation}
 T_a = \frac{2\pi}{\omega_i} = \frac{2\pi}{\sqrt{\omega_0^2-\gamma^2}}
 \label{eq:PeriodoAmortiguado1}
\end{equation}

Estamos ya en posición de comenzar a demostrar lo que el problema nos pide, lo cual es
que el cambio en la energía durante un período es $2T/\tau$,
donde $T$ el período de oscilador sin amortiguar y $\tau$ es el tiempo
que tarda la amplitud en reducirse por un factor $1/e=1/2,718$. Comenzamos por 
escribir la energía total para oscilador amortiguado, la cual está dada por la suma
de la energía cinética y potencial

\begin{equation}
 E = \frac{1}{2} m \dot{x}^2 + \frac{1}{2} kx^2
 \label{eq:EnergiaAmorti1}
\end{equation}

Para un oscilador armónico esta energía es constante, pero en este caso veremos
que se cumple algo un poco diferente. Diferenciemos a $E$ con respecto al tiempo:

$$
\frac{dE}{dt} = m\dot{x}\ddot{x} + kx\dot{x} = (m\ddot{x}+kx)\dot{x}
$$

Pero por la ecuación (\ref{eq:ODEAmortiguado}) vemos que

\begin{equation}
\frac{dE}{dt} = - b \dot{x}^2
\label{eq:DerivadaEnergOscAmorti}
\end{equation}

La ecuación (\ref{eq:DerivadaEnergOscAmorti}) representa la tasa de cambio de la
energía para el oscilador armónico, que comúnmente es conocida cono la tasa de 
disipación de energía debido a que (\ref{eq:DerivadaEnergOscAmorti})
es siempre cero o negativa, entonces como la amplitud, la energía decrece
y en un punto se hace despreciable. El cambio de en la energía durante un período
$T_a$ es 

\begin{equation}
 \Delta E = \int_0^{T_i} \dot{E} dt
 \label{eq:CambioEnergPeriodAmorti}
\end{equation}

De la donde $\dot{E}$ está dada por (\ref{eq:DerivadaEnergOscAmorti}). Necesitamos
una expresión para $\dot{x}$, para eso derivamos (\ref{eq:SolAmortiguado}) y obtenemos

\begin{equation}
 \dot{x} = - A\gamma \euler^{-\gamma t} \sen{\omega_i t + \delta} - \omega_i \cos{(\omega_i t + \delta)}
 \label{eq:derivadaSolAmortiguado}
\end{equation}

Sustituyendo (\ref{eq:derivadaSolAmortiguado}) en (\ref{eq:CambioEnergPeriodAmorti}),

\begin{equation}
 \Delta E = \int_0^{T_i} [- A\gamma \euler^{-\gamma t} \sen{\omega_i t + \delta} - \omega_i \cos{(\omega_i t + \delta)}] dt
\end{equation}

Un procedimiento muy útil en este punto es el que sigue \cite{fowles} en la cual
se cambia la variable de integración por $\theta = \omega_i t + \delta$, con lo que
$dt = d\theta / \omega_i$ y la integral sobre el período $T_i$, se transforma debido
al cambio de variable\footnote{\\$
 t = 0 \Rightarrow \theta = \delta 
 $ \\
 $t = T_i \Rightarrow \theta = \omega_i T_i + \delta = \omega_i \frac{2\pi}{\omega_i} + \delta = 2\pi + \delta
 $} en una integral de $\delta$ a $\delta + 2\pi$, pero debido a que el valor de la 
integral sobre un ciclo completo no depende de la fase inicial del movimiento $\delta$,
lo podemos eliminar de los límites de integración, lo cual simplifica la evaluación
de la integral, entonces tenemos que

\begin{align}
 \begin{split}
  %
  \Delta E &= \frac{1}{\omega_i} \int_0^{2\pi} \dot{E} d\theta \\
  %
  &= - \frac{bA^2}{\omega_i}\euler^{-2\gamma t}\int_0^{2\pi} 
  (\gamma^2 \sen^2{\theta} - 2\gamma \omega_i \sen{\theta} \cos{\theta} + \omega_i^2 \cos^2{\theta}) d\theta
 \end{split}
\end{align}

La integral de $\sen^2{\theta}$ y $ \cos^2{\theta}$ de 0 a $2\pi$ es $\pi$, mientras
que la integral del producto de $\sen{\theta}\cos{\theta}$ se hace cero. Por lo tanto,

\begin{align}
 \begin{split}
  %
  \Delta E &= - \frac{bA^2}{\omega_i}\pi \euler^{-2\gamma t}(\gamma^2 + \omega_i^2) \\
  %
  &= - b A^2 \euler^{-2\gamma t}\omega_0^2\left(\frac{\pi}{\omega_i}\right) \\
  %
  &= - \gamma m \omega_0^2 A^2 \euler^{-2\gamma t}T_i
 \end{split}
\end{align}

Donde hemos utilizado las relaciones $\omega_0^2 = \omega_i^2 + \gamma^2$ y
$\gamma = b/2m$. Si ahora identificamos el valor de $\gamma$ con una
constante $\tau$, tal que $\gamma = (2\tau)^-1$, es decir el tiempo en que 
la amplitud decrece $1/\euler$, obtenemos que la magnitud de la pérdida de energía 
en un ciclo es

$$
\Delta E = \left(\frac{1}{2} mA^2\omega_0^2\euler^{-t/\tau}\right)\frac{T_i}{\tau}
$$

y definiendo a $E=\frac{1}{2}mA^2\omega_0^2\euler^{-t/\tau}$ como la energía
almacenada en el oscilador amortiguado, vemos que 

\begin{equation}
 \frac{\Delta E}{E} = \frac{T_i}{\tau}
\end{equation}

Podemos escribir esta ecuación en función del período de un oscilador sin amortiguamiento $T$
, expandiendo en series de Taylor y truncando a primer orden\footnote{Podemos hacer esto debido
a que el amortiguamiento es pequeño y por lo tanto $\gamma$ es pequeño} podemos 
lograr ese objetivo

\begin{align*}
 \begin{split}
  %
  T_i &= \frac{2\pi}{\sqrt{\omega_0^2-\gamma^2}} = \frac{2\pi}{\omega_0}\frac{1}{\sqrt{1-\frac{\gamma^2}{\omega_0^2}}} \\
  %
  T_i &\approx \frac{2\pi}{\omega_0} \left[1 + \cancel{\frac{1}{2}\frac{\gamma^2}{\omega_0^2}} \right]
  %
  \therefore T_i \approx T
 \end{split}
\end{align*}

Entonces tenemos que 

\begin{equation}
 \boxed{\frac{\Delta E}{E} = \frac{T}{\tau}}
 \label{eq:SolucionFinalProb3}
\end{equation}

La ecuación (\ref{eq:SolucionFinalProb3}) demuestra justo lo que se pide en el problema
el que cambio en la energía durante un período es $T/\tau$, expresado en términos del
período del oscilador sin amortiguar y $\tau$ es el tiempo que tarda la amplitud en reducirse
por un factor $1/\euler = 1 / 2.718$. Cabe destacar que la solución final al problema no
es exactamente lo que se pide demostrar, debido a que falta un 2 en el numerador, pero
luego de una intensa revisión de los cálculos realizados se llega a la conclusión de que
el valor obtenido es correcto, el único modo de que la solución de exactamente igual
es que se tome el valor de $T_i = 2T$.

\vspace{.3cm}

\section{Problema 3}
Un proyectil se dispara a partir del origen con velocidad inicial $v_{0}$.
Se desea impactar el punto de coordenadas $x=x_{0}$, $z=0$.

\begin{enumerate}[a)]
 \item ¿Cuál es el ángulo (o ángulos) de disparo que se requieren?
 \item Encuentre la corrección a primer orden que se debe dar a dicho ángulo si toma 
 en cuenta la resistencia del aire.
\end{enumerate}
\vspace{.3cm}

\underline{Solución:}

\vspace{.3cm}

Podemos esquematizar el problema como en la (figura \ref{fig:problema3}). 
Tenemos un proyectil que es disparado desde el origen a una velocidad 
inicial $v_0$, haciendo un ángulo $\theta$ con el eje x, y deseamos
alcanzar el punto $x=x_0$ y $z=0$. Primero consideremos el caso sin
tomar en cuenta la resistencia del aire.

\begin{figure}[ht]
 \centering
\includegraphics[scale=0.3]{problema3fig1}
\caption{Problema 3}
\label{fig:problema3}
\end{figure}

\vspace{.3cm}

De la (figura \ref{fig:problema3}) utilizando las relaciones trigonométricas en un 
triángulo rectángulo vemos que

\begin{gather*}
v_{0x} = v_0\cos(\theta) \\
v_{0z} = v_0\sen(\theta)
\end{gather*}


Si la partícula de masa $m$ está sujeta a la atracción gravitacional
por la tierra, entonces tenemos, utilizando la segunda ley de Newton,

\vspace{.3cm}

En la dirección de $x$,

\begin{equation}
 0 = m \ddot{x}
 \label{ec:1}
\end{equation}

y en la dirección de $z$,

\begin{equation}
 - mg = m \ddot{z}
  \label{ec:2}
\end{equation}

De la ecuación (\ref{ec:1}) tenemos que (utilizando las condiciones iniciales),

\begin{equation*}
 \ddot{x} = 0 \Rightarrow \frac{d\dot{x}}{dt} = 0 \Rightarrow 
 \int_{v_{0x}}^{\dot{x}}d\dot{x}' = 0 \Rightarrow \dot{x} - v_{0x} = 0 
 \end{equation*}

\begin{equation}
 \therefore \dot{x} = v_0 \cos(\theta)
 \label{eq:3}
\end{equation}

De la ecuación (\ref{eq:3}) podemos obtener $x(t)$, 

\begin{equation*}
 \frac{dx}{dt} = v_0 \cos(\theta) \Rightarrow dx = v_0 \cos(\theta) dt \Rightarrow
 \int_0^x dx' = v_0 \cos(\theta) \int_0^t dt'
\end{equation*}

\begin{equation}
 \therefore x = v_0 t \cos(\theta)
 \label{eq:4}
\end{equation}

Luego para $z$, utilizando la ecuación (\ref{ec:2}),

\begin{equation*}
 \ddot{z} = -g \Rightarrow \frac{d\dot{z}}{dt} = -g \Rightarrow 
 \int_{v_{0z}}^{\dot{z}}d\dot{z}' = -g \int_0^t dt' \Rightarrow \dot{z} - v_{0z} = -gt 
 \end{equation*}
 
\begin{equation}
 \therefore \dot{z} = -gt + v_0 \sen(\theta)
 \label{eq:5}
\end{equation}

De la ecuación (\ref{eq:5}) podemos obtener $z(t)$,

\begin{gather*}
 \frac{dz}{dt} = -gt + v_0 \sen(\theta) \Rightarrow dz = (-gt + v_0 \sen(\theta)) dt \\ \Rightarrow
 \int_0^z dz' = -g\int_0^t t' dt' + v_0 \cos(\theta) \int_0^t t'dt'
\end{gather*}

\begin{equation}
 \therefore z = - g \frac{t^2}{2} + v_0 t \sen(\theta)
 \label{eq:6}
\end{equation}

Podemos encontrar el momento en que el proyectil se encuentra en el punto $x_0$, 
es decir hasta que choque con el piso, lo cual ocurre al final de su recorrido cuando $z=0$.
Para dar respuesta a la parte $a)$ de la pregunta debemos determinar en qué momento
esto ocurre, es decir cuánto es el valor del tiempo para el punto en que 
$z=0$, ya que esto nos permitirá determinar los ángulos de lanzamiento
que requerimos para que el proyectil llegue a $x_0$ y $z=0$.

\vspace{.3cm}

Debido a que el proyectil sale disparado desde el origen, $z=0$ cuando
$t=0$, pero mediante la ecuación (\ref{eq:6}) podemos determinar el otro
momento en que $z=0$ que es al final de su recorrido, y a este tiempo
lo llamaremos $\tau$, reescribiendo (\ref{eq:6}) podemos ver esto claramente,

\begin{equation*}
 z = t(\frac{-gt}{2} + v_0 \sen(\theta)) = 0
\end{equation*}

Donde vemos que $z=0$ si $t=0$, pero también para un tiempo $\tau$,

\begin{equation*}
 \frac{-g\tau}{2} + v_0 \sen(\theta) = 0
\end{equation*}

Resolviendo para $\tau$,

\begin{equation}
 \tau = \frac{2v_0\sen(\theta)}{g}
 \label{eq:7}
\end{equation}

Entonces para obtener el rango del proyectil en $z=0$ cuando
$t=\tau$ y ha llegado al punto $x_0$, para eso sustituimos (\ref{eq:7}) en (\ref{eq:4})

\begin{equation*}
 x_0 = v_0 \tau \cos(\theta) \Rightarrow x = 2 \frac{(v_0)^2}{g} \sen(\theta)\cos(\theta)
\end{equation*}

Utilizando la identidad trigonométrica $2\sen(\theta)\cos(\theta) = \sen (2\theta)$,
obtenemos que

\begin{equation}
 x_0 = \frac{(v_0)^2}{g} \sen(2\theta)
 \label{eq:8}
\end{equation}

Despejando a $\theta$ de (\ref{eq:8}) podemos obtener uno de los ángulos que se 
requieren para que el proyectil llegue a $x_0$ y $z=0$, esto es

% \begin{equation}
%  \theta = \frac{n\pi}{4} \qquad \qquad n = 4\epsilon - 3, \quad  \epsilon = 1,2,3,4,5,\dots
%  \label{eq:9}
% \end{equation}

\begin{equation}
 \boxed{\theta = \frac{1}{2} \arcsen{\left(\frac{x_0 g}{v_0^2}\right)}}
 \label{eq:9}
\end{equation}

El otro ángulo puede obtenerse restando $\pi/2$ menos (\ref{eq:9}) lo 
cual completa la solución a la primera parte del problema.

\begin{equation}
 \boxed{\theta = \frac{1}{2}\left[\pi - \arcsen{\left(\frac{x_0 g}{v_0^2}\right)} \right]}
\end{equation}

\vspace{.3cm}

Para dar respuesta a la segunda parte del problema debemos considerar la 
resistencia del aire. Si suponemos que la fuerza aplicada por la resistencia 
del aire es proporcional a la velocidad del proyectil, tenemos entonces

En la dirección de $x$,

\begin{equation}
 m\ddot{x} = - k m \dot{x}
 \label{eq:10}
\end{equation}

y en la dirección de $z$,

\begin{equation}
 m \ddot{z} = - k m \dot{z} - mg
 \label{eq:11}
\end{equation}

Resolvemos primero la ecuación (\ref{eq:10}), la reescribimos como

\begin{align}
\begin{split}
 \ddot{x} &= - k \dot{x} \\
 %
 \frac{d \dot{x}}{dt} &= - k \dot{x} \\
 %
 \frac{d \dot{x}}{\dot{x}} &= - k dt
 %
 \end{split}
\end{align}

Y debido a que las condiciones iniciales son las mismas

\begin{align}
  \begin{split}
   %
  \int_{v_{ox}}^{\dot{x}} \frac{d \dot{x}'}{\dot{x}'} &= - k \int_0^t dt' \\
%   %  
  \therefore \dot{x} &= \euler^{-kt} v_0 \cos{\theta}
  \end{split}
  \label{eq:13}
 \end{align}

Podemos encontrar $x$ de la ecuación (\ref{eq:13}),

\begin{align}
  \begin{split}
   %
  \frac{dx}{dt} &= \euler^{-kt} v_0 \cos{\theta} \\
  %
  \int_0^x dx' &= v_0 \cos{\theta} \int_0^t \euler^{-kt'} \\
  %
  \therefore x &= \frac{v_0\cos{\theta}}{k}(1 - \euler^{-kt}) 
  \end{split}
  \label{eq:14}
 \end{align}

Resolvamos ahora la ecuación (\ref{eq:11}), reescribiendo como

\begin{align}
\begin{split}
 \ddot{z} &= - k \dot{z} - g \\
 %
 \frac{d \dot{z}}{dt} &= - k \dot{z} -g \\
 %
 \frac{d \dot{z}}{k\dot{x} +g} &= - dt \\
 %
 \int \frac{d \dot{z}}{k\dot{x} +g} &= \int dt
 \end{split}
 \label{eq:15}
\end{align}

Luego, como las condiciones iniciales son las mismas y haciendo el
cambio de variable $u=k\dot{z}+g$ podemos encontrar la solución a 
(\ref{eq:15}),

\begin{equation}
 \frac{1}{k} ln (k\dot{z}+g)|_{v_0z}^{\dot{z}} = -t
\end{equation}

Luego de hacer las sustituciones y la evaluación en los límites de integración
obtenemos que

\begin{equation}
 \dot{z} = - \frac{g}{k} + \frac{k v_0 \sen{\theta}+g\euler^{-kt}}{k}
 \label{eq:16}
\end{equation}

Podemos obtener el valor de $z$ de (\ref{eq:16}),

\begin{equation}
 \int_0^z dz' = \frac{g}{k} \int_0^t dt' + \frac{k v_0 \sen{\theta}+g}{k} \int_0^t \euler^{-kt'}dt'
\end{equation}

Por lo tanto

\begin{equation}
 z = - \frac{gt}{k}+ \frac{k v_0 \sen{\theta}+g}{k^2}(1-\euler^{-kt})
 \label{eq:17}
\end{equation}

Usaremos la misma metodología que en la primera parte del problema,
para encontrar ahora la corrección que se debe hacer para los ángulos
de disparo cuando se consideraba la resistencia del aire.

\vspace{.3cm}

Entonces, de (\ref{eq:17}) podemos encontrar el tiempo $\tau$ necesario
para que el proyectil alcance $z=0$,

\begin{equation}
 - \frac{gt}{k}+ \frac{k v_0 \sen{\theta}+g}{k}(1-\euler^{-kt}) = 0
\end{equation}

Despejando $\tau$, 

\begin{equation}
 \tau = \frac{k v_0 \sen{\theta}+g}{gk}(1-\euler^{-kt})
 \label{eq:18}
\end{equation}

Pero (\ref{eq:18}) es una ecuación trascendental para $\tau$, por lo 
que nos es imposible encontrar un valor analítico para $\tau$. Pero
podemos utilizar un método perturbativo para encontrar un valor aproximado.
Haciendo una expansión de $(1-\euler^{-kt})$, en series de potencia encontramos que

\begin{equation}
 \tau = \frac{k v_0 \sen{\theta}+g}{gk}\left(k\tau - \frac{1}{2}k^2\tau^2
 +\frac{1}{6}k^3\tau^3 - \dots \right)
\end{equation}

Aunque nos piden una corrección de primer orden nos será útil mantener
algunos órdenes superiores por el momento para obtener una ecuación de la 
forma

\begin{equation}
 \tau = \frac{tk^2 v_0 \sen{\theta}+gkt}{gk} - \frac{1}{2} \frac{t^2k^3 v_0 \sen{\theta}+gk^2t^2}{gk} + 
 \frac{1}{6} \frac{t^3k^4 v_0 \sen{\theta}+gk^3t^3}{gk}
\end{equation}

Si mantenemos los términos de la expansión hasta $k^3$ y luego de un largo
camino algebraico obtenemos que

\begin{equation}
 \tau = \frac{2 v_0 \sen{\theta} /g}{1+k v_0 \sen{\theta / g}} + \frac{1}{3} k \tau^2
 \label{eq:19}
\end{equation}

Expandiendo $\frac{1}{1+k v_0 \sen{\theta / g}}$ del primer término de
(\ref{eq:19}), y manteniendo términos hasta $k$ que es lo que nos dice
el problema, obtenemos 

\begin{equation}
  \tau = \frac{2 v_0 \sen{\theta}}{g}+\left(\frac{\tau^2}{3} - \frac{2 v_0^2 \sen^2{\theta}}{g^2} \right)
  \label{eq:20}
\end{equation}

Puede verificarse que fácilmente que si $k=0$, obtenemos el mismo resultado
para $\tau$ que para el caso sin resistencia del aire, lo cual es un indicio
de que la ecuación (\ref{eq:20}) es correcta. Por lo tanto, si $k$ es
pequeño, el tiempo $\tau$ será aproximadamente igual a l $\tau_0$ que
calculamos anteriormente, si usamos ese valor y sustituimos en el
lado derecho de (\ref{eq:20}), obtenemos un valor un poco más compacto
para el nuevo $\tau$,

\begin{equation}
 \tau \approx \frac{2 v_0 \sen{\theta}}{g}\left(1 - \frac{k v_0 \sen{\theta}}{3g} \right)
 \label{eq:21}
\end{equation}

Escribimos ahora la ecuación para $x$ en su forma expandida,

\begin{equation}
 x = \frac{v_0 \cos{\theta}}{k}\left(kt - \frac{1}{2}k^2t^2 + \frac{1}{6}k^3t^3  + \right)
\end{equation}

Pero $x = x_0$ cuando $t = \tau$, entonces 

\begin{equation}
 x_0 \approx v_0 \cos{\theta}\left(\tau - \frac{1}{2} k \tau^2 \right)
 \label{eq:22}
\end{equation}

Donde en (\ref{eq:22}) hemos considerado términos hasta el primer orden de $k$. 
Sustituyendo el valor de $\tau$ que hemos obtenido en (\ref{eq:21}), 

\begin{equation}
 x_0 = \frac{2 v_0^2 \cos{\theta} \sen{\theta} }{g}\left(1 - \frac{4k v_0 \sen{\theta} }{3g} \right)
\end{equation}

Que utilizando el hecho de que $2\sen{\theta}\cos{\theta} = \sen{2\theta}$, podemos
escribir como

\begin{equation}
 x_0 = \frac{v_0^2 \sen{2\theta}}{g}\left(1 - \frac{4k v_0 \sen{\theta} }{3g} \right)
 \label{eq:30}
\end{equation}

% Obtener el ángulo de lanzamiento de esta ecuación es un poco complicado
% pero se puede hacer luego de tres expansiones en series de Taylor, debajo
% están algunos pasos de muchos que deben hacerse para despejar $\theta$,
% 
% \begin{align}
%  \begin{split}
%   %
%   \frac{x_0}{1- \frac{4k}{3g}v_0 \sen{\theta}} &= \frac{v_0^2 \sen{2\theta}}{g} \\
%   %
%   x_0 - \frac{4k}{3g}v_0 x_0 \sen{\theta} &\approx \frac{v_0^2 \sen{2\theta}}{g} \\
%   %
%   - \frac{4k v_0 x_0}{3g} &\approx \frac{v_0^2 \cos{\theta}}{g} - x_0 \csc{\theta} \\
%   %
%   - \frac{4 v_0 x_0}{3g/k} &\approx \frac{v_0^2 \cos{\theta}}{g} - x_0 \left( \frac{cos\theta}{\sqrt{1-\cos^2{\theta}}}\right) \\
%   %
%   \cos{\theta} &\approx - \frac{4 v_0 x_0}{3k (2  v_0/g - x_0)}
%   %
%  \end{split}
% \end{align}
% 
% Por lo que 
% 
% \begin{equation}
%  \theta \approx \arccos{\frac{4k v_0 x_0}{3g (2 v_0 x_0/g - v_0 x_0)}}
% \end{equation}

Pero utilizando la ecuación 8 ecuación (\ref{eq:8}) podemos ver que el término de
corrección, al cual llamaremos $\alpha$ es\footnote{Hemos despreciado la contribución de 
$\theta$ de este término, debido a que como la ecuación (\ref{eq:30}) es trascendental para
$\theta$, solo podemos hallar valores aproximados para el mismo.}

\begin{equation}
  \alpha = \frac{4k v_0 }{3g}
\end{equation}

% Entonces
% 
% \begin{align*}
%  \frac{\cancel{x_0}}{1 - \frac{4k v_0 \sen{\theta} }{3g}} = \cancelto{x_0}{\frac{v_0^2 \sen{2\theta}}{g}}
% \end{align*}
% 
% \begin{equation}
%   \frac{1}{1- \frac{4k v_0 \sen{\theta} }{3g}} = 1
% \end{equation}
% 
% Expandiendo en series de Taylor,
% 
% \begin{equation}
%  1 - \frac{1}{3} \frac{4k v_0 \sen{\theta} }{3g}  = 1
% \end{equation}

\begin{align*}
 x_0 &\approx \frac{v_0^2 \sen{2\theta}}{g}(1 - \alpha) \\
%
 x_0 &\approx  \frac{v_0^2 \sen{2\theta}}{g} -  \frac{\alpha v_0^2 \sen{2\theta}}{g} \\
%
 x_0 &\approx \sen{2\theta} \left[ \frac{v_0^2}{g} - \frac{\alpha v_0^2}{g}\right] \\
 %
 x_0 &\approx \sen{2\theta} \left[ \frac{v_0^2 - \alpha v_0^2}{g}\right]
\end{align*}

Entonces 

\begin{equation}
 \boxed{\theta \approx \frac{1}{2} \arcsen{\frac{x_0 g}{v_0^2(1-\alpha)}}}
 \label{eq:33}
\end{equation}

De nuevo, el otro ángulo puede obtenerse restando $\pi/2$ menos (\ref{eq:33}) lo 
cual completa la solución a la segunda parte del problema.

\begin{equation}
 \boxed{\theta \approx \frac{1}{2}\left[\pi - \arcsen{\frac{x_0 g}{v_0^2(1-\alpha)}} \right]}
\end{equation}
























\vspace{.3cm}

\section{Problema 4}

Dos partículas de masa $m_1$ y $m_2$ están amarradas por una cuerda sin masa de 
longitud $l$ y descansan sobre una mesa sin fricción. En el instante t=0 la cuerda
está totalmente desplegada y las masas están en reposo; en ese instante se le aplica
un impulso a la partícula $m_2$ de tal suerte que ésta adquiere una velocidad $v_0$ 
perpendicular a la cuerda.

\begin{enumerate}[a)]
 \item Describa el movimiento después de haber aplicado el impulso
 \item ¿Cuál es la tensión de la cuerda durante el movimiento?
\end{enumerate}

\vspace{.3cm}

\underline{Solución:}\vspace{.3cm}

Para tener una mejor intuición del problema podemos representarlo como en la figura (\ref{fig:problema4})

\begin{figure}[ht]
 \centering
\includegraphics[scale=0.4]{problema4fig1}
\caption{Problema 4. Hemos hecho un pequeño cambio de notación en que $l=d$}
\label{fig:problema4}
\end{figure}

Donde consideramos que vemos de arriba de la mesa el sistema, como dice el problema
se le aplica una fuerza perpendicular a la cuerda a la masa 2 la cual se puede observar
en la figura. Consideraremos un caso general en el que las masas de las partículas
no necesariamente son las mismas, y trataremos el caso utilizando la formulaciónn
del centro de masa, que hemos ubicado en un punto arbirtrario entre las partículas,
y obviamente en la línea que las une que igual a la cuerda extendida.

\vspace{.3cm}

Debido a la simetría del problema, se puede ver que el movimiento que ocurrirá
luego de que se le aplica el impulso a la partícula $m_2$ será un movimiento
circular uniforme, con velocidad igual a la velocidad del centro de masa, que podemos
obtener fácilmente. El radio del centro de masa se puede obtener con la ecuación

\begin{equation}
 R_{CM} = \frac{m_1 r_1 + m_2 r_2}{m1+m2}
 \label{eq:RadioCM1}
\end{equation}

Escogeremos unas coordenadas que nos facilitarán el problema, en las cuales el eje
X, lo pondremos sobre la partícula $m_1$, por lo tanto $r_1 = 0$, luego

\begin{equation}
 R_{CM} = \frac{m_2 r_2}{m1+m2}
 \label{eq:RadioCM2}
\end{equation}

Entonces la velocidad del centro de masa será

\begin{equation}
 \boxed{V_{CM} = \frac{m_2 \dot{r_2}}{m_1+m_2} = \frac{m_2 v_0}{m_1 + m_2}}
\end{equation}

La tensión de la cuerda, debido al tratamiento de movimiento circular uniforme que 
le hemos dado al problema será igual a la fuerza centrípeta, la cual debería ser
la misma no importa si la calculemos utilizando la fuerza centrípeta que 
sentirá $m_1$ o $m_2$, demostremos esto. 

\vspace{.3cm}

La tensión de la cuerda en la partícula $m_2$ será

\begin{align*}
 T = F_c &= \frac{m_2 v_2'^2}{r_2'} = \frac{m_2 (v_0-V_{CM})^2}{d - \frac{m_2 d}{m1+m2}} = 
 \frac{m_2 (v_0 - \frac{m_2 v_0}{m_1 + m_2})^2}{d - \frac{m_2 d}{m1+m2}} \\
 %
 &= \frac{\frac{m_2 m_1^2 v_0^2}{(m_1+m_2)^2}}{\frac{d m_1}{m_1 + m2}} \\
%
 \therefore T &= \frac{m_1 m_2}{d(m_1 + m_2)} v_0^2
\end{align*}

Si definimos la masa reducida del sistema como $\mu = \frac{m_1 m_2}{m_1 + m_2}$, 
entonces tenemos que

\begin{equation}
 \boxed{T = \frac{\mu}{d} v_0^2}
 \label{eq:TensionM2}
\end{equation}

Nuestro tratamiento será correcto si encontramos que la tensión de la cuerda
en ambas masas es la misma, por lo tanto para $m_1$

\begin{align*}
\begin{split}
 T = \frac{m_1 v_1'^2}{r_1'} = \frac{m_1(0 - V_{CM})^2}{\frac{m_2}{m_1+m_2}} =
 \frac{m_1 (\frac{m_2 v_0}{m1+m2})^2}{\frac{m_2}{m_1+m_2}} =
 \frac{m_1 m_2 v_0^2 (m_1 + m2)}{d m_2 (m_1 + m_2)^2} \\
 %
 \end{split}
\end{align*}

\begin{equation}
 \boxed{\therefore T = \frac{m_1 m_2}{d(m_1 + m_2)}v_0^2 = \frac{\mu}{d} v_0^2} %
\end{equation}

Entonces vemos que la tensión será igual si definimos la masa reducida $\mu$ de nuevo,
lo cual completa la demostración y la solución al problema.





\vspace{.3cm}

\section{Problema 5}

Una partícula de masa $m_1$ tiene una energía cinética $T_1i$ y choca elásticamente 
con otra de masa $m_2$. La segunda partícula sale disparada en una dirección que hace un
ángulo $\theta$ con la dirección del movimiento inicial de la primera partícula. Encuentre
la energía cinética $T_2f$ con la que sale la segunda partícula. Muestre que esta energía
es máxima si el choque es de frente.

\vspace{.3cm}

\underline{Solución:}

\vspace{.3cm}

% Para la resolución de este problema se seguirán las convenciones y notación de \cite{marion} y algunos
% otros autores para el tratamiento de las colisiones elásticas. Como dicen los autores, 
% la descripción de muchos procesos físicos puede simplificarse considerablemente, si se 
% escoge un sistema de coordenadas en reposo con respecto al centro de masa del sistema. Según
% el enunciado del problema puede asumirse que la segunda partícula estaba en reposo, mientras
% que la primera va directo a chocarla. Las mediciones reales se hacen en el sistema de coordenadas
% del laboratorio en el cual el observador está en reposo, en este sistema una de las partículas
% está en movimiento, mientras que la otra está en reposo. No tenemos la misma imagen desde el
% sistema de coordenadas del centro de masa como veremos más adelante. Denominaremos al primero como el sistema \textbf{LAB}
% y al segundo como el sistema \textbf{CM}.


Para la resolución de este problema se seguirán las convenciones y notación de \cite{marion} y algunos
otros autores para el tratamiento de las colisiones elásticas. Según el enunciado del problema puede asumirse 
que la segunda partícula estaba en reposo, mientras que la primera va directo a chocarla. Esto simplifica 
un poco los cálculos y ecuaciones pero la extensión a un choque donde ambas partículas tienen una
velocidad inicial no es muy engorroso.

\vspace{.3cm}

Usaremos la siguiente notación:

 \begin{gather}
 \begin{split}
% 
  m_1 = \text{Masa de la partícula en movimiento} \\
  m_2 = \text{Masa de la partícula en reposo}
%  
  \label{eq:masas}
%  
 \end{split}
 \end{gather}

% Usaremos cantidades con primo (') para el sistema CM,
Las velocidades iniciales y finales de las partículas la denotaremos como

 \begin{gather}
 \begin{split}
%   u_1 = \text{Velocidad inicial de } m_1 \text{en el sistema LAB} \\
%   v_1 = \text{Velocidad final de } m_1 \text{en el sistema LAB} \\
%   
%   u_1' = \text{Velocidad inicial de } m_1 \text{en el sistema CM} \\
%   v_1' = \text{Velocidad final de } m_1 \text{en el sistema CM}
% 
  u_1 = \text{Velocidad inicial de } m_1 \\
  v_1 = \text{Velocidad final de } m_1 \\
%   
  u_2 = 0 \\
  v_2 = \textrm{Velocidad final de} m_2\\
 \label{eq:velMasas}
 \end{split}
 \end{gather}

%   \begin{gather}
%   \begin{split}
%    u_2 = 0 \\
%    v_2 = \textrm{Velocidad final de} m_2 \textrm{en el sistema LAB} \\
% %  
%  \vspace{.2cm}
% %  
%   u_2' = \textrm{Velocidad inicial de} m_2 \textrm{en el sistema CM} \\
%   v_2' = \textrm{Velocidad final de} m_2 \textrm{en el sistema CM}
% %  
%   \label{eq:velMasa2}
%   \end{split}
%  \end{gather}
 
Para las energías cinéticas,
 
 \begin{gather*}
  T_{0i} = \textrm{Energía cinética total inicial} \\
%
  T_{0f} = \textrm{Energía cinética total final} \\
%   T_0' = \textrm{Energía cinética total inicial el sistema CM} \\
%  
%  
   T_{1i} = \textrm{Energía cinética inicial de } m_1 \\
%    T_{1i}' = \textrm{Energía cinética inicial de } m_1 \textrm{en el sistema CM} \\
%   
   T_{2i} = \textrm{Energía cinética inicial de } m_2 \\
%    T_{2i}' = \textrm{Energía cinética inicial de } m_2 \textrm{en el sistema CM} \\
%  
  T_{1f} = \textrm{Energía cinética final de } m_1 \\
%   T_{1f}' = \textrm{Energía cinética final de } m_1 \textrm{en el sistema CM} \\
%  
  T_{2f} = \textrm{Energía cinética final de } m_2 \\
%   T_{2f}' = \textrm{Energía cinética final de } m_2 \textrm{en el sistema CM} \\
%
\label{eq:energiasCineticas}
  \end{gather*}
  
Y para los momenta,

 \begin{gather*}
  P_{0i} = \textrm{Momentum total inicial} \\
 %
  P_{0f} = \textrm{Momentum total final} \\
%  
   p_{1i} = \textrm{Momentum inicial de } m_1 \\
%   
   p_{2i} = \textrm{Momentum inicial de } m_2 \\
%  
  p_{1f} = \textrm{Momentum final de } m_1 \\
%  
  p_{2f} = \textrm{Momentum final de } m_2 \\
%
\label{eq:Momenta}
  \end{gather*}
  
En las colisiones elásticas el momentum y la energía cinética se conservan, lo cual nos 
será de mucha utilidad para los cálculos posteriores. Hagámos un breve análisis de lo 
que ocurrirá luego de la colisión. La masa $m1$ se moverá con una velocidad $v_1$,
haciéndo un ángulo $\psi$ con el eje $X$, y la masa $m_2$ se moverá con una 
velocidad $v_2$ haciéndo un ángulo $\theta$ con el eje $X$ (ver figura (\ref{fig:problema5}). La conservación 
del momentum lineal y la energía requieren que 

\begin{figure}[ht]
 \centering
\includegraphics[scale=0.5]{problema5fig1}
\caption{Problema 5}
\label{fig:problema5}
\end{figure}


\begin{equation}
 \mathbf{P}_{0i} = \mathbf{P}_{0f} \qquad \text{y} \qquad \mathbf{T}_{0i} = \mathbf{T}_{0f}
 \label{eq:ConservacionesColision1}
\end{equation}

Desglosando (\ref{eq:ConservacionesColision1}) tenemos que,

\begin{gather}
 \begin{split}
 %
  \mathbf{p}_{1i} +  \mathbf{p}_{2i} &=   \mathbf{p}_{1f} +  \mathbf{p}_{2f} \\
 %
  \mathbf{T}_{1i} +  \mathbf{T}_{2i} &=   \mathbf{T}_{1f} +  \mathbf{T}_{2f} \\
 \label{eq:ConservacionesColision2}
 \end{split}
\end{gather}

donde

\begin{gather*}
 \begin{split}
%
\mathbf{p}_{1i} = m_1 u_1, \qquad \mathbf{p}_{2i} = 0, \qquad \mathbf{p}_{1f} = m_1 v_1, \qquad \mathbf{p}_{2f} = m_2 v_2 \\
%
\mathbf{T}_{1i} = \frac{1}{2} m_1 u_1^2, \qquad \mathbf{T}_{2i} = 0, \qquad \mathbf{T}_{1f} = \frac{1}{2} m_1 v_1^2, \qquad \mathbf{T}_{2f} = \frac{1}{2} m_2 v_2^2 \\
%
 \end{split}
\end{gather*}

Utilizando estas expresiones podemos escribir \ref{eq:ConservacionesColision2} en forma de componentes
en el eje $X$ y $Y$, con la ayuda de la figura (\ref{fig:problema5}) como

\begin{equation}
%
m_1 u_1 = m_1 v_1 \cos{\phi} + m_2 v_2 \cos{\theta}
\label{eq:momentaAngulos1}
\end{equation}

\begin{equation}
0 = m_1 v_1 \sin{\phi} + m_2 v_2 \sin{\theta} 
\label{eq:momentaAngulos2}
\end{equation}

Y para las energías

\begin{equation}
 \frac{1}{2}m_1 u_1^2 = \frac{1}{2} m_1 v_1^2 + \frac{1}{2} m_2 v_2^2
 \label{eq:cineticas1}
\end{equation}

En la mayoría de las situaciones, $m_1$, $m_2$ y $u_1$ son conocidas, mientras que
$v_1$, $v_2$, $\theta$ y $\phi$ son cantidades desconocidas. Entonces tenemos tres
ecuaciones \mref{eq:momentaAngulos1, eq:momentaAngulos2, eq:cineticas1} y cuatro variables desconocidas. Pero
podemos eliminar una de esas cuatro incógnitas de manera simple, y encontrar relaciones
para las otras tres. Lo haremos para $\phi$, y encontraremos relaciones para $v_1$, 
$v_2$ y $\theta$. Comenzamos por escribir las ecuaciones \mref{eq:momentaAngulos1} y
\mref{eq:momentaAngulos2} como

\begin{align*}
 m_1 u_1 - m_1 v_1 \cos{\phi} &= m_2 v_2 \cos{\theta} \\
 m_1 v_1 \sen{\phi} &= m_2 v_2 \sen{\theta}
\end{align*}

Si elevamos al cuadrado estas ecuaciones y las sumamos obtenemos

\begin{align}
\begin{split}
  %
  m_1^2 u_1^2 - 2 m_1^2 u_1 sv_1 \cos{\phi} + m_1^2 v_1^1 \cos^2{\phi} + m_1^2 v_1^2 \sen^2{\phi}
  &= m_2^2 v_2^2 \cos{\theta} + m_2^2 v_2^2 \sen{\theta} \\
 %
  m_1^2 u_1^2 - 2 m_1^2 v_1 u_1 \cos{\phi} + m_1^2 v_1^2 &= m_2^2 v_2^2 
 %
\end{split}
\end{align}

Si dividimos por $m_1^2$ nos queda

\begin{equation}
 u_1^2 + v_1^2 - 2 u_1 v_1 \cos{\phi} = \left(\frac{m2}{m1} \right)^2 v_2^2 
 \label{eq:velocidadesCoseno}
\end{equation}

Y de la ecuación (\ref{eq:cineticas1}) obtenemos

\begin{equation}
 v_2^2 = \frac{m1}{m2}(u_1^2 - v_1^2)
 \label{eq:velocidadM2}
\end{equation}

Sustituyendo el valor de $v_2^2$ de (\ref{eq:velocidadM2}) en (\ref{eq:velocidadesCoseno}),
obtenemos

\begin{equation}
 u_1^2 + v_1^2 - 2 u_1 v_1 \cos{\phi} = \left(\frac{m2}{m1} \right)^2 \frac{m1}{m2} (u_1^2 - v_1^2)
\end{equation}

Y luego de un poco de trabajo algebraico descubrimos que esta es una ecuación cuadrática
en $v_1/u_1$ que al resolver nos da las siguientes raíces

\begin{equation}
 \frac{v_1}{u_1} = \frac{m_1}{m_1+m_2}\left[\cos{\theta} \pm \sqrt{\cos^2{\theta} - \left(\frac{m_1^2 - m_2^2}{m_1^2} \right)} \right]
\label{eq:expresionFinalVelM1}
 \end{equation}

Con esta expresión podemos entonces dar respuesta completa al problema. Primero para 
encontrar la energía cinética de $m_2$ solo debemos sustituir (\ref{eq:velocidadM2}) en la
ecuación para $T_{2f}$, entonces

\begin{align}
 T_{2f} = \frac{1}{2} m_2 v_2^2 = \frac{1}{2} m_2 \frac{m1}{m2}(u_1^2 - v_1^2) \\
 %
 \boxed{T_{2f} = m_1 (u_1^2 - v_1^2)}
 %
\end{align}

Esta energía es máxima claramente cuando $\cos{\phi}$ en (\ref{eq:expresionFinalVelM1}) es
igual a 1, lo que corresponde a una colisión de frente, que es justamente lo que nos piden
demostrar, en ese caso las velocidades finales de $m_1$ y $m_2$ son

\begin{align}
 \begin{split}
  %
  v_1 &= \frac{m_1 - m_2}{m_1 + m2} u_1 \\
  %
  v_2 &= \frac{2 m_1}{m_1 + m_2} u_1
 \end{split}
\end{align}

Y en ese caso la energía cinética de $m_2$ tendría la siguiente expresión

\begin{equation}
 T_{2f_{max}} = \frac{1}{2} m_2 v_2^2 = \frac{1}{2} m_2 \left(\frac{2 m_1 }{m_1 + m_2} u_1\right)^2
\end{equation}

Entonces,

\begin{equation}
 \boxed{T_{2f_{max}} = \frac{2 m_2 m_1^2}{(m_1 + m_2)^2} u_1^2}
\end{equation}

% Llamaremos $\textbf{V}$ a la velocidad del centro de masa en el sistema LAB, $\psi$ al 
% ángulo en que $m_1$ se desvía en el sistema LAB, $\xi$ al ángulo en que $m_2$ se desvía
% en el sistema LAB y $\theta$ al ángulo en el que $m_1$ y $m_2$ se desvían en el sistema CM.

% De acuerdo con la definición de centro de masa, tenemos que
% 
% \begin{equation}
%  m_1 \mathbf{r_1} + m_2 \mathbf{r_2} = M\mathbf{R}
%  %
%  \label{eq:CM1}
% \end{equation}
% 
% Diferenciando (\ref{eq:CM1}) con respecto al tiempo encontramos
% 
% \begin{equation}
%  m_1 \mathbf{u_1} + m_2 \mathbf{u_2} = M\mathbf{V}
%  \label{eq:CMVel1}
% \end{equation}
% 
% Pero debido a que $\mathbf{u_2}=0$ y que $M=m_1+m_2$, el centro de masa se mueve
% hacia $m_2$ con una velocidad,
% 
% \begin{equation}
% \mathbf{V} = \frac{m_1 \mathbf{u_1}}{m_1+m_2}
%  \label{eq:CMVel2}
% \end{equation}
% 
% Debido a que $m_2$ está en reposo al inicio, la velocidad inicial de $m_2$ en el sistema
% CM debe ser igual a $V$:
% 
% \begin{equation}
% u_2'= V = \frac{m_1 u_1}{m_1+m_2}
%  \label{eq:u2iniVel}
%  \end{equation}
%   
% Pero notando que el movimiento y las velocidades son opuestas en dirección, i.e. $\mathbf{u_2'=-\mathbf{V}}$.
% En el sistema CM, de la manera en que lo hemos escogido, el momentum lineal total es cero, por lo tanto
% antes de la colisión las partículas se mueven la una hacia la otra, y luego de la colisión
% se mueven exactamente en direcciones opuestas. Debido a que la colisión es elástica,
% las masas no cambian, y la conservación del momentum y la energía cinética son suficientes
% para decirnos que la velocidad en el sistema CM son iguales antes y después de la colisión,
% por lo que 
% 
% \begin{equation}
%  u_1'=v_1', \quad u_2'=v_2'
%  \label{eq:velIguales}
% \end{equation}
% 
% De (\ref{eq:u2iniVel}) vemos que 
% 
% \begin{equation}
% v_2'= \frac{m_1 u_1}{m_1+m_2}
%  \label{eq:u2finalVel} 
% \end{equation}
% 
% Recordando la forma expresar las velocidades en el sistema LAB con respecto al centro de masa,
% tomando como ejemplo una velocidad arbitraria $v_x$
% 
% $$
% v_{x,LAB} = v_{x,CM} + v_{CM,LAB}
% $$
% 
% Tenemos entonces que, utilizando nuestra notación y la ecuación (\ref{eq:u2iniVel}),
% 
% \begin{equation}
%  u_1 = u_1' + V = u_1' + u_2'
%  \label{eq:u1iniVel}
% \end{equation}
% 
% Utilizando las ecuaciones (\ref{eq:velIguales}) y (\ref{eq:u1iniVel}), podemos ver entonces que 
% 
%  \begin{gather}
%  \begin{split}
% % %
%    v_2' = \frac{m_1 u_1}{m_1 + m_2} \\
% %   %
%    v_1' = u_1 - u_2' = \frac{u_1 m_1 + \cancel{m_2 u_1} - \cancel{m_2 u_1}}{m_1 + m_2} =
%    \frac{m_2 u_1}{m_1 + m_2}\\
% %   %
%    \label{eq:velFinales}
% %   %
%  \end{split}
%  \end{gather}
% 
% Ya estamos en posición entonces para encontrar expresiones y relaciones que involucren
% la energía de las partículas. En el sistema LAB y debido a que la colisión es elástica,
% la energía cinética total inicial es igual a la energía cinética de la partícula $m_1$,
% es decir
% 
% \begin{equation}
%  T_0 = \frac{1}{2} m_1 u_1^2
%  \label{eq:energCinetTotIniLAB}
% \end{equation}
% 
% y en el sistema CM,
% 
% 
% \begin{equation}
%  T_0' = \frac{1}{2} (m_1 u_1'^2 + m_2 u_2'^2)
%  \label{eq:energCinetTotIniCM1}
% \end{equation}
% 
% pero utilizando las expresiones para las velocidades de (\ref{eq:velFinales}), 
% la ecuación (\ref{eq:energCinetTotIniCM1}) se convierte en 
% 
% \begin{align}
% \begin{split}
%  %
%  T_0' &= \frac{1}{2} \left[ \left(\frac{m_1 u_1}{m_1 + m_2}\right)^2 + m_2 \left(\frac{m_2 u_1}{m_1 + m2+}\right)^2\right] \\
%  %
%        &= \frac{1}{2} \left[ u_1^2 \frac{m_1 m_2^2}{(m_1+m_2)^2} + u_1^2 \frac{m_2 m_1^2}{(m_1+m_2)^2}\right] \\
% %  %
%        &= \frac{1}{2} \left[ m_1 m_2 u_1^2 \left( \frac{m1}{(m_1 + m_2)^2} + \frac{m_2}{(m_1 + m_2)^2}\right)\right] \\
% %  %
%        &= \frac{1}{2} \left[ m_1 m_2 u_1^2 \left( \frac{m_1+m_2}{(m_1+m_2)^2}\right) \right] \\
% %  %
%        &= \frac{1}{2} \left( \frac{m_1 m_2}{m_1+m_2}u_1^2 \right) \\
% %  %
% \therefore T_0' &= \frac{m_2}{m_1+m_2} T_0
%  %
% \end{split}
% \end{align}
% 
% El problema nos pide cual es la energía cinética $T_{2f}$ con la que sale la segunda partícula,
% ya con los resultados que hemos obtenido es fácil obtener la misma, calcularemos
% la energía cinética final de la partícula $m_1$ y $m_2$ tanto para el sistema
% LAB como para el sistema CM, para ello utilizando las expresiones para las 
% velocidades finales de la ecuación (\ref{eq:velFinales}),
% 
% \begin{align}
% \begin{split} 
% %
%  T_{1f}' &= \frac{1}{2} v_1'^2 \\
%  %
% 	 &= \frac{1}{2}m_1\left(\frac{m_2}{m_1+m_2}\right)^2 u_1^2 \\
% 	 %
% 	 &= \left(\frac{m_2}{m_1+m_2}\right)^2 T_0\\
% %
% \label{eq:erergCineFinalm1CM}
% \end{split}
% \end{align}
% 
% y
% 
% \begin{align}
% \begin{split} 
% %
%  T_{2f}' &= \frac{1}{2} v_2'^2 \\
% %
% 	 &= \frac{1}{2}m_2\left(\frac{m_2}{m_1+m_2}\right)^2 u_1^2 \\
% %
% 	 &= \frac{1}{2}m_1 u_1^2 \frac{m_1 m_2}{(m_1+m_2)^2} \\
% %
% 	 &= \frac{m_1 m_2}{(m_1+m_2)^2} T_0
% %
% \label{eq:erergCineFinalm2CM}
% \end{split}
% \end{align}






\section{Problema 6}

Un objeto de masa $m$ se encuentra en el ecuador de la Tierra. ¿Qué velocidad debe 
tener para que su peso y la fuerza de Coriolis sean iguales?

\vspace{.3cm}

\underline{Solución:}

\vspace{.3cm}

El efecto de la rotación de la Tierra sobre los experimentos de laboratorio es muy pequeño, pero sin embargo se puede
hacer un tratamiento del problema planteado considerandolo como un problema de movimiento con respecto
a unsistema de coordenadas en rotación. No demostraremos todas las ecuacioens debido a que se haría
muy larga la solución, pero podemos comenzar por escribir la siguiente ecuación,

\begin{equation}
 \mathbf{F'} = \mathbf{F} - 2m(\mathbf{\omega} \times \mathbf{v'}) - m\mathbf{\omega} \times (\mathbf{\omega} \times \mathbf{r'})
 \label{eq:FuerzasDeSistemaRotacion}
\end{equation}

Donde $\mathbf{F'} = m\mathbf{a'} = d\mathbf{p'}/dt$ es la fuerza que mediría un observador $S'$ situado
en un sistemas de coordenadas en rotación, $\mathbf{F}$ es la fuerza que mediría un observador
en reposo en un sistema inercial. Los dos términos resultantes son las conocidas \emph{fuerzas de inercia}:

\begin{equation}
 \mathbf{F_i'} = - 2m(\mathbf{\omega} \times \mathbf{v'}) - m\mathbf{\omega} \times (\mathbf{\omega} \times \mathbf{r'})
 \label{eq:FuerzasDeInercia}
\end{equation}

Donde recordamos que $\mathbf{\omega}$ es la velocidad angular del sistema en rotación. Es evidente que cuando $\mathbf{\omega} = 0$
entonces $S'$ no gira y ambos observadores deberían concordar en las fuerzas. Estas dos fuerzas tienen nombre, el segundo de los términos
recibe el nombre de fuerza centrífuga, aunque no diremos mucho más de ella, cabe resaltar que esta siempre dirigida hacia
afuera, es siempre perpendicular al eje de rotación y es proporcionala $\omega^2$ y a la deistancia al eje.

\vspace{.3cm}

La fuerza que nos interesa para resolver este problema es el primer término de (\ref{eq:FuerzasDeInercia}), $- 2m(\mathbf{\omega} \times \mathbf{v'})$,
llamada \emph{fuerza de Coriolis}, esta como se puede ver depende de la velocidad en $S'$, pero no de la posición. La fuerza de Coriolis 
es perpendicular a $\mathbf{\omega}$ y a $\mathbf{v'}$. En el problema se nos pregunta qé velocidad debe tener un objeto de masa $m$ para que su peso 
y la fuerza de Coriolis sean iguales. Como hemos dicho el efecto de la fuerza Coriolis sobre el movimiento de los cuerpos en la Tierra 
suele ser muy pequeño. Debido a la forma y estructura de la ecuación para la fuerza Coriolis podemos predecir que el objeto se deberá mover 
relativamente rápido para que su peso y la fuerza de Coriolis sean iguales, porque la velocidad angular de la tierra es de unos $2\pi$ 
radianes por día sideral (23 $h$, 56 $m$, 4.1 $s$) lo que da un valor aproximado de $7.3 \times 10^-5$ rad/s. El hecho de que el
cuerpo está en el ecuador facilita un poco el cálculo ya que no hay que considerar las correcciones para la fuerza Coriolis en el
hemisferio norte o sur de la Tierra.

\vspace{.3cm}

Entonces lo que requerimos es que para el objeto de masa $m$ se cumpla

\begin{equation}
 - 2m(\mathbf{\omega} \times \mathbf{v'}) = - m \mathbf{g}
 \label{eq:Coriolis1}
\end{equation}

De (\ref{eq:Coriolis1}), la velocidad que debe tener el cuerpo es de (asumiendo que se encuentra a nivel del mar
y una distribución homogénea del campo gravitacional que lo afecta)

\begin{equation}
 \boxed{ v' = \frac{g}{2\omega} = \frac{9.8 m/s^2}{7.3 \times 10^-5 rad/s} = 6.72 \times 10^5 m/s}
\end{equation}


\vspace{.3cm}

\section{Problema 7}

La hélice de un avión tiene un momento de inercia $I$ y el motor le imprime una torca 

$$N_m = N_0 (1+\alpha \cos{\omega t}) \qquad (\alpha \ll 1))$$

La resistencia del aire le imprime una torca

$$N_f = -b\dot{\theta}$$

¿Cuál es el estado estacionario del movimiento de la hélice?

\vspace{.3cm}

\underline{Solución:}

Nos encontramos con un caso en el que podemos asumir que la hélice del avión es un
cuerpo rígido en rotación, debido a que mediante esta suposición podremos utilizar
las ecuaciones clásicas para cuerpos rígidos en rotación y se simplifica mucho el
problema. Claramente esta es una descripción idealizada de la hélice porque no 
existe un cuerpo de tamaño físico que sea estríctamente rígido, debido a que se
deformará (así sea en cantidades muy pequeñas) bajo la acción de fuerzas aplicadas
en él. Sin embargo como veremos este tratamiento será muy útil para describir el
movimiento, y las desviaciones con respecto al sistema físico real no son tan 
significantes.

Recordemos que para un cuerpo rígido el momentum angular se peude expresar como

\begin{equation}
 L = I\omega = I\dot{\theta}
 \label{eq:MomemtumAngCuerpoRig}
\end{equation}

Donde I es conocido como el momento de inercia, $\omega$ es la velocidad angular
del cuerpo y, por consideraciones de simetría en el tratamiento del cuerpo
rígido podemos ver que $\dot{\theta} = \omega$\footnote{Este tratamiento es el que
se encuentra en la mayoría de los libros de mecánica clásica \cite{marion,arya}, y simplifica mucho
las ecuaciones diferenciales e interpretación física de la rotación del cuerpo}. Por
otra parte, la tasa de cambio del momentum angular de cualquier sistema, es igual
al torque total externo $\tau$. Entonces para el caso en que estamos tratando,

\begin{equation}
 \sum_i \tau_i = \frac{dL}{dt} = I \frac{d\omega}{dt} = I \ddot{\theta} 
 = N_0(1+\alpha \cos{\omega t}) - b\dot{\theta}
 \label{eq:torquesCuerpoRig1}
\end{equation}

Podemos reescribir (\ref{eq:torquesCuerpoRig1}) como

\begin{equation}
 I\ddot{\theta} - b\dot{\theta} = N_0(1+\alpha \cos{\omega t})
 \label{eq:torquesCuerpoRig2}
\end{equation}

Donde se puede ver que tenemos una ecuación diferencial lineal no homogénea
para $\theta$. Seguiremos el método clásico de solución en el cual se encuentra
una solución para la ecuación homogénea (haciendo cero el lado derecho de
(\ref{eq:torquesCuerpoRig2}) y una solución particular tomando en cuenta 
la forma de la parte que hace a la ecuación no homogénea, la solución final 
será una suma de la ecuación homogénea más la particular\footnote{Esto es posible
debido a que la ecuación diferencial que resolveremos es lineal}.

\vspace{.3cm}

La ecuación diferencial homogénea a resolver es,

\begin{equation}
 I\ddot{\theta} + b\dot{\theta} = 0
 \label{eq:helixHomog1}
\end{equation}

Si suponemos que la solución tiene la forma 

$$\theta = \euler^{rt}$$

y sus derivadas

\begin{align*}
 \begin{split}
%  
  \dot{\theta} &= r\euler^{rt} \\
  %
  \ddot{\theta} &= r^2\euler^{rt}  
%
 \end{split}
\end{align*}


entonces,

\begin{align*}
 \begin{split}
%  
  Ir^2\euler^{rt} + br\euler^{rt} &= 0 \\
  %
  (Ir^2+br)\euler^{rt} &= 0 \qquad \text{pero} \qquad \euler^{rt} \neq 0
%
 \end{split}
\end{align*}

$$
\therefore Ir^2 + br = 0 \Rightarrow r(Ir + b) = 0
$$

Para esta ecuación auxiliar, es fácil ver que las raíces son

\begin{align*}
 \begin{split}
  r_1 &= 0 \\
  %
  r_2 &= - \frac{I}{b}
  \label{eq:raicesHelixHomog}
 \end{split}
\end{align*}

Por lo tanto la solución homgénea es

\begin{equation}
 \theta_h = A \euler^{0t} + B \euler^{-\frac{I}{b}t} = A + B \euler^{-\frac{I}{b}t}
 \label{eq:solucHelixHomog}
\end{equation}

Para encontrar la solución partícular utilizaremos el método de los coeficientes
indeterminados, y supondremos que la solución tiene la forma

$$
\theta_p = C \cos{\omega t} + D \sen{\omega t} + E t
$$

\begin{align}
 \begin{split}
  \dot{\theta}_p = - C \omega \sen{\omega t} + D \omega \cos{\omega t} + E \\
  %
  \ddot{\theta}_p = - C \omega^2 \cos{\omega t} - D \omega^2 \sen{\omega t}
  \label{eq:helixPart1}
 \end{split}
\end{align}

Si sustituimos (\ref{eq:helixPart1}) en (\ref{eq:torquesCuerpoRig2}), nos queda

\begin{align*}
 \begin{split}
 %
 -IC\omega^2 \cos{\omega t} - ID\omega^2 \sen{\omega t} + bC\omega \sen{\omega t} 
 -bD\omega \cos{\omega t} - bE &= \\ N_0 + N_0\alpha \cos{\omega t} \\
 %
 \cos{\omega t}[-IC\omega^2 - bD\omega] + \sen{\omega t}[-ID\omega^2+bC\omega]
 -bE  &= \\ N_0 + N_0\alpha \cos{\omega t}
 %
 \end{split}
\end{align*}

Por lo tanto

\begin{align*}
 \begin{split}
  %
  -IC\omega^2 - bD\omega &= N_0 \alpha \\
  %
  -ID\omega^2+bC\omega &= 0 \\
  %
  - bE = N_0
  %
 \end{split}
\end{align*}

De donde se obtiene directamente que $E = -N_0/b$, y para $C$ y $D$, 

\begin{align*}
 \begin{split}
  %
  C &= \frac{ID\omega^2}{b\omega} \\
  %
  \therefore C &= \frac{ID\omega}{b} 
  %
 \end{split}
\end{align*}

luego podemos usar el valor de $C$ para obtener $D$

\begin{align*}
 \begin{split}
  %
  D &= - \frac{N_0 \alpha + IC\omega^2}{b\omega} \\
  %
  D &= - \frac{N_0 \alpha + I \frac{ID\omega}{b} \omega^2}{bw} \\
  %
  D &= - \frac{N_0 \alpha}{b\omega} + \frac{I^2 D \omega^3}{b^2\omega} \\
  %
  D &= - \frac{N_0 \alpha}{b\omega} + \frac{I^2D\omega}{b^2} \\
  %
  D - \frac{I^2D\omega}{b^2} &= - \frac{N_0 \alpha}{b\omega} \\
  %
  D  \left(1 - \frac{I^2\omega}{b^2} \right) &= - \frac{N_0 \alpha}{b\omega} \\
  %
  D \left(\frac{b^2-I^2\omega}{b^2} \right) &= - \frac{N_0 \alpha}{b\omega} \\
  %
  D &= - \frac{N_0\alpha}{b\omega} \div \left(\frac{b^2-I^2\omega}{b^2} \right) \\
  %
  \therefore D &= - \frac{N_0 \alpha b}{\omega(b^2-I\omega^2)}
  %
 \end{split}
\end{align*}

Luego,

$$
C = - \frac{I\omega}{b}  \frac{N_0 \alpha b}{\omega(b^2-I\omega^2)} 
\Rightarrow C = - \frac{IN_0 \alpha}{b^2-I\omega^2}
$$

Entonces la solución particular es igual 

\begin{equation}
 \theta_p = - \frac{IN_0 \alpha}{b^2-I\omega^2} \cos{\omega t} 
 - \frac{N_0 \alpha b}{\omega(b^2-I\omega^2)} \sen{\omega t} - \frac{N_0}{b}
\end{equation}


Por lo tanto la solución a la ecuación diferencial es

\begin{equation}
 \theta = A + B \euler^{-\frac{I}{b}t} - \frac{IN_0 \alpha}{b^2-I\omega^2} \cos{\omega t} 
 - \frac{N_0 \alpha b}{\omega(b^2-I\omega^2)} \sen{\omega t} - \frac{N_0}{b}
\end{equation}

El movimiento estacionario de la hélice es el que ocurrirá cuando el factor $\euler^{-\frac{I}{b}t}$, 
se haga casi cero, lo cual ocurrirá mientras el tiempo avance, debido a que es un factor de 
decaimiento exponencial, entonces cuando $\euler^{-\frac{I}{b}t} \approx 0$, el movimiento
de la hélice podrá ser descrito por el siguiente $\theta$, 

\begin{equation}
 \boxed{\theta = A - \frac{IN_0 \alpha}{b^2-I\omega^2} \cos{\omega t} 
 - \frac{N_0 \alpha b}{\omega(b^2-I\omega^2)} \sen{\omega t} - \frac{N_0}{b}}
\end{equation}

El cálculo de las constantes de integración podrá hacerse si se conocen las condiciones iniciales
para la ecuación diferencial (\ref{eq:torquesCuerpoRig2}). 

\vspace{.3cm}

\section{Problema 8}
8.- Dos cuerpos rígidos extensos (no puntuales) chocan elásticamente. ¿Es posible que 
después del choque ambos queden únicamente con energía cinética de rotación en torno 
a sus centros de masa?. De ser así de un ejemplo.
\vspace{.3cm}

\underline{Solución:}\vspace{.3cm}

Cuando chocan dos cuerpos rígidos, suelen intercambiar cantidad de movimiento y momentum
cinético. En dicho intercambio se conservan la cantidad de movimiento total y si es central
la fuerza de intercambio entre los cuerpos, el momentum cinético total. Debido a que somos
enfrentados en el problema con un choque elástico, se conservará también la energía 
cinética total del sistema.

\vspace{.3cm}

Para demostrar la posibilidad del enunciado consideremos el caso de dos barras moviéndose,
ambas de masa $m$ y longitud $d$, la una hacia la otra, ambas con velocidad del centro de masa
$V$, y chocan justo en el borde quedando unidas (tenían algún tipo de pegamento en los bordes,
o gancho), entonces debido a que debe conservarse el momentum lineal, y el momento cinético
, estos deben ser iguales antes y después del choque, pero como el choque es tal que después
que queden unidas la barras, se muevan la una en dirección opuesta a la otra y por las condiciones
iniciales de cantidad de movimiento para cada barra, la energía cinética de rotación será
tal que solo dependerá del centro de masa del sistema. Por lo tanto se demuestra en términos
cualitativos que si es posible el enunciado del problema. Debajo podemos ver una ilustración
del ejemplo planteado.

\begin{figure}[ht]
 \centering
\includegraphics[scale=0.4]{problema8fig1}
\caption{Problema 8}
\label{fig:problema8}
\end{figure}


\vspace{.3cm}

\section{Problema 9}

Un meteorito choca con la Tierra en el Polo Norte haciendo un ángulo $\theta$ con
el eje de la tierra. El meteorito tiene una masa igual a $10^-17$ de la masa de la 
Tierra y llega con la velocidad de escape. ¿En qué forma se verá perturbado el movimiento
de rotación de la Tierra? (Desprecie el cambio en el momento de inercia de la Tierra
debido a la incrustación del meteorito).
\vspace{.3cm}

\underline{Solución:}\vspace{.3cm}

El problema al cual nos vemos enfrentados a resolver podemos imaginarlo representado
por la figura (\ref{fig:problema9})


\begin{figure}[ht]
 \centering
\includegraphics[scale=0.3]{problema9fig1}
\caption{Problema 9}
\label{fig:problema9}
\end{figure}

Consideraremos a la Tierra como un cuerpo rígido ideal, lo cual simplificará grandemente
la solución del problema. Como hemos establecido anteriormente no existe un cuerpo rígido
con dimensiones físicas reales, y ciertamente la tierra no es uno de ellos, pero la simplificación
nos permitirá encontrar relaciones que dan una gran idea de cómo se comportará un choque
realista. De la teoría de cuerpos rígidos sabemos que en ausencia de un impulso exterior,
el momentum cinético y desde luego, la cantidad de movimiento permanecen constantes. Por 
lo que las variaciones de la cantidad de movimiento y del momentum cinético total se
deberán únicamente a un impulso exterior.

\vspace{.3cm}

Supongamos entonces que el choque del meteorito es tal que le aplica un impulso instantáneo
$\mathbf{J}$ a la tierra. La variación de la cantidad de movimiento $\Delta \mathbf{P} = \mathbf{J}$
es independiente del punto de aplicación, pero la variación del momentum cinético depende
del punto de aplicación. Si $\mathbf{R}$ es el vector de posición de la Tierra, que consideraremos
que es el vector de posición del centro de masa de la Tierra, y ella es quien recibe 
el impulso, entonces la variación del momentum cinético\footnote{Cuando hablamos de momentum cinético nos referimos al momentum angular, pero en la
notación clásica de cuerpos rígidos es más común hablar de momentum cinético.} total
de la tierra será

\begin{equation}
 \Delta \mathbf{L_t} = \mathbf{R_t} \times \mathbf{J}
 \label{eq:MomentumAngularTierra1}
\end{equation}

Pero debido a que $\Delta \mathbf{P} = \mathbf{J}$, entonces podemos reescribir 
(\ref{eq:MomentumAngularTierra1}) como

\begin{equation}
 \Delta \mathbf{L_t} = \mathbf{R_t} \times \Delta \mathbf{P}
 \label{eq:MomentumAngularTierra2} 
\end{equation}

Debido a que estamos tratando al impulso $\mathbf{J}$ como instantáneo y que consideramos
que la cantidad de movimiento de la tierra permanece constante luego de la aplicación
del mismo, entonces $\Delta \mathbf{P} = m_m v_m$, donde $m_m$ es la masa del meteorito
y $v_m$ es la velocidad del meteorito. Sustituyendo esto en (\ref{eq:MomentumAngularTierra2})
nos queda

\begin{equation}
 \boxed{\Delta \mathbf{L_t} = \mathbf{R_t} \times m_m \mathbf{v_m} = R_t m_m v_m \sen{\theta}}
 \label{eq:MomentumAngularTierra3}
\end{equation}

La ecuación (\ref{eq:MomentumAngularTierra3}) nos da una expresión de cómo se verá
perturbado el movimiento de rotación de la tierra luego del choque del meteorito, lo cual
es lo que pide el problema. Claramente llegar a esta ecuación fue sencillo, debido a que
el enunciado del problema indica que se puede despreciar el cambio de momento de inercia
de la Tierra debido a incrustación el meteorito. Podemos ponerle números a la ecuación
(\ref{eq:MomentumAngularTierra3}) y ver cuánto es modificado el momentum cinético de la 
Tierra, sustituyendo tenemos que

\begin{equation}
 \boxed{\Delta \mathbf{L_t} = (6,371 \times 10^{6} m) (5,972 \times 10^7 kg) (11,2 \times 10^{3} m s^{-1}) \sen{\theta}}
 \label{eq:MomentumAngularTierraFinal}
\end{equation}

La ecuación (\ref{eq:MomentumAngularTierraFinal}) es la solución final al problema,
donde solo nos queda como incógnita el valor del ángulo $\theta$ que hace el meteorito
al chocar, pero podemos ver claramente que si el choque es tal que $\sen{\theta} = 1$, entonces
será máximo el cambio, dando un cambio de aproximadamente $\boxed{4,26 \times 10^{18} Js}$, lo cual
sería el cambio en el momentum angular de la Tierra. Es un resultado algo significativo
debido a que la masa del meteorito era relativamente grande y llegó con una gran velocidad,
pero igual no es lo suficiente para alterar notablemente el movimiento de rotación de la 
Tierra\footnote{Esto puede comprobase fácilmente si se calcula el momentum de rotación de 
la Tierra.},cabe destacar que el análisis de unidades es correcto debido a que el momentum angular
se expresa en Joules por segundo, y es exactamente lo que hemos obtenido. El ángulo
que hace el impacto sea máximo es entonces de $90^\circ$, y el cambio será cero si 
el ángulo es de $0^\circ$.


\vspace{.3cm}

\section{Problema 10}

Uno de los postulados básicos de la mecánica clásica es el de la relatividad galileana,
nos dice que la mecánica clásica debe ser invariante ante las transformaciones de Galileo, 
esto es las leyes de la física básica deben ser las mismas en todos los sistemas inerciales
de referencia y la transformación entre un sistema inercial y otro es la de Galileo 
($\hat{r}' = \hat{r} - \hat{v}'t$ donde $\hat{v}'$ es la velocidad constante entre los dos 
sistemas de referencia).

\begin{itemize}
 \item En la formulación de Newton de 1687 se consideraba la existencia de un sistema 
 absoluto de referencia que permitía definir el estado de reposo de un cuerpo, Newton lo
 veía como un sistema anclado a las estrellas fijas o uno muy cercano a este. La visión 
 moderna se desarrolló en el siglo XIX, elimina la idea del reposo absoluto y hace 
 uso del postulado que mencionamos. Describa brevemente el cambio entre la visión 
 estrictamente newtoniana y la del siglo XIX.
 \item Muestre, a manera de ejemplo del cumplimiento del postulado anterior, que las 
 ecuaciones de movimiento del oscilador armónico tridimensional son invariantes ante
 las transformaciones de Galileo.
\end{itemize}

\vspace{.3cm}

\underline{Solución:}

\vspace{.3cm}

\begin{thebibliography}{10}
 \bibitem{marion}
 S. Thronton y J. Marion, \textit{Classical dynamics of particles and systems}, Thomson Brooks/Cole,
 5ta edición, 2004.
 \bibitem{taylor}
 J. Taylor, \textit{Classical mechanics}, University Science Books, 2005.
 \bibitem{zill}
 D. Zill y W. Wright, \textit{Differential equations with Boundary-Value Problems}, Brooks/Cole,
 8va edición, 2013.
 \bibitem{arya}
 A. Arya, \textit{Introduction to classical mechanics}, Pretince-Hall, 1era Edición,
 1990.
 \bibitem{fowles}
 G. Fowles y G. Cassiday, \emph{Analytical Mechanics}, Thomson Brooks/Cole, 
 7ma edición, 2005.
\end{thebibliography}


\end{document}
