\documentclass[a4paper,10pt]{article}
\usepackage[utf8]{inputenc}
\usepackage[spanish]{babel}
\usepackage[affil-it]{authblk}
\usepackage{enumerate}



%opening
\title{Mec\'anica Cl\'asica Tarea \# 1}
\author{Favio V\'azquez\thanks{Correo: favio.vazquezp@gmail.com}}\affil{Instituto de F\'isica. Universidad Nacional Aut\'onoma de M\'exico}
\date{}

\begin{document}

\makeatletter
\def\@maketitle{%
  \newpage
  \null
  \vskip 2em%
  \begin{center}%
  \let \footnote \thanks
    {\Large\bfseries \@title \par}%
    \vskip 1.5em%
    {\normalsize
      \lineskip .5em%
      \begin{tabular}[t]{c}%
        \@author
      \end{tabular}\par}%
    \vskip 1em%
    {\normalsize \@date}%
  \end{center}%
  \par
  \vskip 1.5em}
\makeatother

\maketitle

1.- Encontrar el movimiento de un oscilador arm\'onico amortiguado
con un coeficiente de amortiguamiento de $\gamma = \omega_{0}/3$ 
($\omega_{0}/3$ frecuencia natural de l oscilador) si a $t=0$ est\'a 
en reposo en su punto de equilibrio y a partir de ese instante se le
aplica una fuerza dada por $$F=A\sin{\omega_{0}t}+B\sin{3\omega_{0}t}$$

\vspace{.3cm}

\underline{Soluci\'on:}

\vspace{.3cm}

2.- Considere un oscilador arm\'onico con un peque~no amortiguamiento.
Muestre que el cambio en la energ\'ia durante un per\'iodo es $2T/\tau$,
donde $T$ el per\'iodo de oscilador sin amortiguar y $\tau$ es el tiempo
que tarda la amplitud en reducirse por un factor $1/e=1/2,718$.

\vspace{.3cm}

\underline{Soluci\'on:}

\vspace{.3cm}

3.- Un proyectil se dispara a partir del origen con velocidad inicial $v_{0}$.
Se desea impactar el punto de coordenadas $x=x_{0}$, $z=0$.

\begin{enumerate}[a)]
 \item ?`Cu\'al es el \'angulo (o \'angulos) de disparo que se requieren?
 \item Encuentre la correcci\'on a primer orden que se debe dar a dicho \'angulo si toma 
 en cuenta la resistencia del aire.
\end{enumerate}
\vspace{.3cm}

\underline{Soluci\'on:}

\vspace{.3cm}

4.- Dos part\'iculas de masa $m_1$ y $m_2$ est\'an amarradas por una cuerda sin masa de 
longitud $l$ y descansan sobre una mesa sin fricci\'on. En el instante t=0 la cuerda
está totalmente desplegada y las masas están en reposo; en ese instante se le aplica
un impulso a la partícula $m_2$ de tal suerte que ésta adquiere una velocidad $v_0$ 
perpendicular a la cuerda.

\end{document}
