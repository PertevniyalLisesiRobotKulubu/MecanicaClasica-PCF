
\documentclass[a4paper,10pt]{article}
\usepackage[utf8]{inputenc}
\usepackage[spanish]{babel}
\usepackage[affil-it]{authblk}
\usepackage{enumerate}
\usepackage{graphicx}
\usepackage{hyperref}
\usepackage{amsmath}
\usepackage{amssymb}
\usepackage{cancel}
\usepackage[usenames, dvipsnames]{color}
\usepackage{tikz}
\usepackage[labelfont=bf]{caption}
\usepackage{subcaption} %Multiple images
\usepackage{multicol} % Multiple columns
\usepackage{float}
\usepackage{cleveref}
 \usepackage{relsize} % bigger math symbols
\usepackage[margin=1.4in]{geometry}
\usepackage[titletoc,toc,title]{appendix}
\usepackage{enumitem}
\usepackage{etoolbox}
\usetikzlibrary{calc}
\numberwithin{equation}{section}

% Circled words
\newcommand{\circled}[2][]{%
  \tikz[baseline=(char.base)]{%
    \node[shape = circle, draw, inner sep = 1pt]
    (char) {\phantom{\ifblank{#1}{#2}{#1}}};%
    \node at (char.center) {\makebox[0pt][c]{#2}};}}
\robustify{\circled}

%Appendices in spanish
\renewcommand{\appendixname}{Ap\'endices}
\renewcommand{\appendixtocname}{Ap\'endices}
\renewcommand{\appendixpagename}{Ap\'endices}

%Zero delimiter
\newcommand{\zerodel}{.\kern-\nulldelimiterspace}

%Columns separation
\setlength{\columnsep}{1cm}

%Indentation
\setlength{\parindent}{0ex}

%Multiple References

\crefrangelabelformat{equation}{(#3#1#4--#5\crefstripprefix{#1}{#2}#6)}

\usepackage{xparse}

%Boxes

\newcommand*{\boxcolor}{blue}
\makeatletter
\renewcommand{\boxed}[1]{\textcolor{\boxcolor}{%
\tikz[baseline={([yshift=-1ex]current bounding box.center)}] \node [rectangle, minimum width=1ex,rounded corners,draw] {\normalcolor\m@th$\displaystyle#1$};}}
 \makeatother

%Constantes
\newcommand{\euler}{\mathrm{e}}
\newcommand{\im}{i}

%Lemas, teoremas, definiciones y pruebas
\newcommand{\definicion}{\textbf{Definición: }}
\newcommand{\lema}{\textbf{Lema: }}
\newcommand{\teorema}{\textbf{Teorema: }}
\newcommand{\prueba}{\textbf{Prueba: }}
\newcommand{\proposicion}{\textbf{Proposición: }}
\newcommand{\corolario}{\textbf{Corolario: }}

% Definición de las secciones y su numeración

\makeatletter
\def\@seccntformat#1{%
  \expandafter\ifx\csname c@#1\endcsname\c@section\else
  \csname the#1\endcsname\quad
  \fi}
\makeatother

%opening
\title{Mecánica Clásica Tarea \# 13}
\author{Favio Vázquez\thanks{Correo: favio.vazquezp@gmail.com}}\affil{Instituto de Ciencias Nucleares. Universidad Nacional Autónoma de México.}
\date{}

\begin{document}

\makeatletter
\def\@maketitle{%
  \newpage
  \null
  \vskip 2em%
  \begin{center}%
  \let \footnote \thanks
    {\Large\bfseries \@title \par}%
    \vskip 1.5em%
    {\normalsize
      \lineskip .5em%
      \begin{tabular}[t]{c}%
        \@author
      \end{tabular}\par}%
    \vskip 1em%
    {\normalsize \@date}%
  \end{center}%
  \par
  \vskip 1.5em}
\makeatother

\maketitle

\section{Problema 1}

Una partícula de masa $m$ se mueve sobre el eje de las $x$ sujeta a un potencial 

$$
V = a\sec^2{\left(\frac{x}{l}\right)},
$$

encuentre la trayectoria por el método de Hamilton-Jacobi.

\vspace{.3cm}

\underline{Solución:} \vspace{.3cm}

Tenemos una partícula que se mueve en sólo una dimensión en el eje $x$, por lo tanto 
podemos escribir la energía cinética como

\begin{equation}
 T = \frac{1}{2}m\dot{x}^2 = \frac{1}{2m}p_x^2,
\end{equation}

y considerando la expresión que tenemos para el potencial, podemos escribir la 
hamiltoniana del sistema como\footnote{Debido a que la lagrangiana $L = T - V$ del 
sistema es independiente del tiempo y la energía cinética es una función cuadrática 
de las velocidades, entonces la cantidad de Jacobi es la energía total del sistema, 
y al escribir a la cantidad de Jacobi en términos de $x$ y $p_x$ tenemos que también 
la hamiltoniana es la energía total del sistema, es decir $H = T+V$.}

\begin{equation}
 H = \frac{1}{2m}p_x^2 + a\sec^2{\left(\frac{x}{l}\right)}.
\end{equation}

Podemos ahora construir la ecuación de Hamilton-Jacobi, que queda expresada como 

\begin{equation}
 \frac{1}{2m}\left(\frac{\partial G}{\partial x}\right)^2 + 
 a\sec^2{\left(\frac{x}{l}\right)} = - \frac{\partial G}{\partial t}.
 \label{eq:HJsec1}
\end{equation}

Claramente esta ecuación se puede resolver por el método de separación de variables,
y proponemos entonces que

\begin{equation}
 G(x,t) = W(x)+ T(t),
\end{equation}

entonces la ecuación \eqref{eq:HJsec1} se convierte en 

\begin{equation}
 \frac{1}{2m}\left(\frac{\partial W}{\partial x}\right)^2 + 
 a\sec^2{\left(\frac{x}{l}\right)} = - \frac{\partial T}{\partial t}.
 \label{eq:HJsec2}
\end{equation}

Debido a que el lado izquierdo de \eqref{eq:HJsec2} depende únicamente de $x$ y el derecho de $t$, el único 
modo de que esta expresión se cumpla es que ambos términos sean iguales a una misma 
constante que llamaremos $\alpha_1$. Entonces,

\begin{equation}
  \frac{1}{2m}\left(\frac{\partial W}{\partial x}\right)^2 + 
 a\sec^2{\left(\frac{x}{l}\right)} = \alpha_1,
\end{equation}

\begin{equation}
 \frac{\partial T}{\partial t} = - \alpha_1.
\end{equation}

De la segunda ecuación vemos que 

\begin{equation}
 T(t) = - \alpha_1t,
\end{equation}

y de la primera ecuación

\begin{equation}
 p_x = \frac{\partial W}{\partial x} = \left\{2m\left[\alpha_1 - a\sec^2{\left(\frac{x}{l}\right)}\right]\right\}^{1/2}.
\end{equation}

y entonces 

\begin{equation}
 W(x) = \int \left\{2m\left[\alpha_1 - a\sec^2{\left(\frac{x}{l}\right)}\right]\right\}^{1/2}dx.
\end{equation}


Por lo tanto tenemos que 

\begin{equation}
 G(x,t) =  \int \left\{2m\left[\alpha_1 - a\sec^2{\left(\frac{x}{l}\right)}\right]\right\}^{1/2}dx
 - \alpha_1t.
\end{equation}

Podemos hallar ahora $\beta_1 = \frac{\partial G}{\partial \alpha_1}$, 

\begin{equation}
 \beta_1 = - t + \int \frac{mdx}{\left\{2m\left[\alpha_1 - a\sec^2{\left(\frac{x}{l}\right)}\right]\right\}^{1/2}}dx.
\end{equation}

De la última ecuación podemos encontrar una ecuación para la trayectoria, $x(t)$, pero 
resulta casi imposible despejar a $x$ de la misma, pero se ha encontrado en términos 
generales la trayectoria del sistema, ya que con encontrar estas expresiones, aunque 
quedan en términos de integrales, se considera resuelto el sistema, esto también 
se da ya que hemos encontrado una expresión para $G$ que contiene toda la información 
del sistema, y obviamente la trayectoria. Para ser completos se muestra debajo el 
resultado de integrar la anterior ecuación, si se desea una expresión 
concreta para $x$ se pueden utilizar algunas aproximaciones o series de potencia.

\begin{equation}
\frac{l\sec{\left(\frac{x}{l}\right)}\sqrt{\alpha_1\cos{\left(\frac{2x}{l}\right)}}
\sqrt{\frac{1}{\alpha_1 - a\sec^2{\left(\frac{x}{l}\right)}}}
\arctan{\left(\frac{\sqrt{2\alpha_1}\sen{\left(\frac{x}{l}\right)}}{\alpha_1\cos{\left(\frac{2x}{l}\right)+ \alpha_1 - 2a}}\right)}}{2\sqrt{\alpha_1}}
= - t - \beta_1
\end{equation}

\section{Problema 2}

Usando los ángulos de Euler como coordenadas, establezca la ecuación de Hamilton-Jacobi 
del trompo simétrico. ¿Se podrá resolver esta ecuación por separación de variables?; 
de ser esto posible encuentre la solución. Puede dejar integrales indicadas.

\vspace{.3cm}

\underline{Solución:} \vspace{.3cm}

\section{Problema 3}

Demuestra que la ecuación de Hamilton-Jacobi de una partícula atraída por dos centros
gravitatorios iguales que se encuentran a una distancia fija $l$ es separable en coordenadas 
elípticas confocales. 

\vspace{.3cm}

\underline{Solución:} \vspace{.3cm}

\section{Problema 4}

Establezca la ecuación de Hamilton-Jacobi de una partícula libre en dos dimensiones 
en coordenadas polares. Encuentre una solución completa de esta ecuación. Haga un 
análisis de las superficies de nivel de esta solución y de su relación con el movimiento. 
Establezca el significado de las constantes $\alpha$ y $\beta$.

\vspace{.3cm}

\underline{Solución:} \vspace{.3cm}

\section{Problema 5}

Utilizando el método de Hamilton-Jacobi reduzca a cuadraturas el péndulo simple. 

\vspace{.3cm}

\underline{Solución:} \vspace{.3cm}

\end{document}