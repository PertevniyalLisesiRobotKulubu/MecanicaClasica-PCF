\documentclass[a4paper,10pt]{article}
\usepackage[utf8]{inputenc}
\usepackage[spanish]{babel}
\usepackage[affil-it]{authblk}
\usepackage{enumerate}
\usepackage{graphicx}
\usepackage{hyperref}
\usepackage{amsmath}
\usepackage{amssymb}
\usepackage{cancel}
\usepackage[usenames, dvipsnames]{color}
\usepackage{tikz}
\usepackage{multimedia}
\usepackage{subcaption} %Multiple images
\usepackage{multicol} % Multiple columns
\usepackage{float}
\usepackage{cleveref}
\usepackage[margin=1.4in]{geometry}
\usepackage[labelfont=bf]{caption}
\usepackage[titletoc,toc,title]{appendix}
\usepackage{enumitem}
\usetikzlibrary{calc}
\numberwithin{equation}{section}

%Appendices in spanish
\renewcommand{\appendixname}{Ap\'endices}
\renewcommand{\appendixtocname}{Ap\'endices}
\renewcommand{\appendixpagename}{Ap\'endices}

%Zero delimiter
\newcommand{\zerodel}{.\kern-\nulldelimiterspace}

%Columns separation
\setlength{\columnsep}{1cm}

%Indentation
\setlength{\parindent}{0ex}

%Multiple References

\crefrangelabelformat{equation}{(#3#1#4--#5\crefstripprefix{#1}{#2}#6)}

\usepackage{xparse}
\ExplSyntaxOn
\NewDocumentCommand{\mref}{m}{\quinn_mref:n {#1}}
\seq_new:N \l_quinn_mref_seq
\cs_new:Npn \quinn_mref:n #1
 {
  \seq_set_split:Nnn \l_quinn_mref_seq { , } { #1 }
  \seq_pop_right:NN \l_quinn_mref_seq \l_tmpa_tl
  ( % print the left parenthesis
  \seq_map_inline:Nn \l_quinn_mref_seq
    { \ref{##1},\nobreakspace } % print the first references
  \exp_args:NV \ref \l_tmpa_tl 
  ) 
 }
\ExplSyntaxOff


%Boxes

\newcommand*{\boxcolor}{blue}
\makeatletter
\renewcommand{\boxed}[1]{\textcolor{\boxcolor}{%
\tikz[baseline={([yshift=-1ex]current bounding box.center)}] \node [rectangle, minimum width=1ex,rounded corners,draw] {\normalcolor\m@th$\displaystyle#1$};}}
 \makeatother

%Constantes
\newcommand{\euler}{\mathrm{e}}
\newcommand{\im}{i}

%Lemas, teoremas, definiciones y pruebas
\newcommand{\definicion}{\textbf{Definición: }}
\newcommand{\lema}{\textbf{Lema: }}
\newcommand{\teorema}{\textbf{Teorema: }}
\newcommand{\prueba}{\textbf{Prueba: }}


%opening
\title{Mecánica Clásica Tarea \# 7}
\author{Favio Vázquez\thanks{Correo: favio.vazquezp@gmail.com}}\affil{Instituto de Ciencias Nucleares. Universidad Nacional Autónoma de México.}
\date{}

\begin{document}

\makeatletter
\def\@maketitle{%
  \newpage
  \null
  \vskip 2em%
  \begin{center}%
  \let \footnote \thanks
    {\Large\bfseries \@title \par}%
    \vskip 1.5em%
    {\normalsize
      \lineskip .5em%
      \begin{tabular}[t]{c}%
        \@author
      \end{tabular}\par}%
    \vskip 1em%
    {\normalsize \@date}%
  \end{center}%
  \par
  \vskip 1.5em}
\makeatother

\maketitle

\section{Problema 1}

En presencia de la gravedad, dos discos uniformes de masa $m$ y radio $R$ unidos por 
un eje de longitud $l$ en torno al cual ambos giran libremente, descansan sobre un plano 
inclinado por un ángulo $\alpha$. Inicialmente se encuentran en reposo y el eje hace un ángulo 
$\beta$ con la dirección de máximo descenso. Si los discos ruedan sin resbalar determine 
las curvas que, sobre el plano, trazan los dos puntos de contacto entre los disco y el plano.

\vspace{.3cm}

\underline{Solución:} \vspace{.3cm}

Debido a la complejidad de este problema, y como en últimas la posición del eje 
está determinada por la posición de los discos, utilizaremos durante casi toda la solución 
las coordenadas del centro del eje, y solo luego al final haremos los cambios necesarios 
para poder expresar la solución en términos de los puntos de contacto entre los 
discos y el plano. Mucho del tratamiento que haremos será siguiendo a \cite{ginoux}. Además para nuestra formulación, aunque 
exista el ángulo $\beta$ que hace el eje con la dirección de máximo descenso, nos será 
más útil el ángulo que hace el eje con el eje horizontal que definiremos a continuación 
en el plano inclinado. Entonces, sean $XX'$ y $YY'$ un sistema de ejes perpendiculares 
sobre el plano inclinado, estando $YY'$ en dirección de máximo descenso. El centro 
$O$ del eje está caracterizado por las coordenadas $X$ e $Y$ en este sistema, con 
origen arbitrario $A$. Denotaremos por $\xi$ y $\xi'$ a los centros de los discos, y la 
distancia $l=\xi\xi'$. Debajo se encuentra una imagen ilustrativa del sistema físico,

\begin{figure}[H]
 \center
 \includegraphics[scale=0.4]{problema1fig1}
 \caption{Representación esquemática del sistema de dos discos que ruedan sin resbalar.}
 \label{fig:problema1fig1}
\end{figure}

Para los discos, el eje puede considerarse como un eje de simetría, y entonces los 
momentos de inercia en el plano de los discos serán iguales $I_1 = I_2 = I$ y a lo 
largo de el eje $\xi\xi'$ será $I_3$. El eje hace un ángulo $\theta$ con la horizontal 
$XX'$. Denotaremos por $\theta$ y $\theta'1$ los ángulos que marcan los puntos de 
referencia de los puntos en la circunferencia de los discos con respecto a la línea 
normal al plano inclinado. Entonces con esas suposiciones y geometría que hemos 
planteado proponemos que podemos describir el sistema en términos de 5 coordenadas 
generalizadas $(X,Y,\theta,\phi,\phi')$. Debajo mostramos el sistema visto de arriba 
para una más fácil visualización de las coordenadas y de los ángulos que hemos 
planteado.

\begin{figure}[H]
 \center
 \includegraphics[scale=0.4]{problema1fig2}
 \caption{Geometría del sistema de ejes propuesto sobre el plano inclinado visto desde 
 arriba.}
 \label{fig:problema1fig2}
\end{figure}

Recordamos que el sistema está definido en el plano inclinado, con origen $A$, el eje 
horizontal es $AX$, el eje $AY$ está en la dirección de máximo descenso y $AZ$ es normal 
al plano. Las coordenadas del centro del eje, $O$, las denotaremos por $X$ y $Y$. Sea 
$\mathbf{K}$ el vector unitario normal al plano, $\mathbf{u}$ el vector unitario 
a lo largo de $\mathbf{O\xi}$ y $\mathbf{v}$ el vector unitario del plano perpendicular 
a $\mathbf{O\xi}$. Por esta construcción es fácil ver que $\mathbf{O\xi} = (l/2)\mathbf{u}$.
Debido a que también por construcción $\mathbf{A\xi} = \mathbf{AO} + \mathbf{O\xi}$ la 
velocidad del punto $\xi$ será $\mathbf{V}_\xi = \mathbf{V}_O + (l/2)\dot{\theta}\mathbf{v}$. 
Para obtener las cantidades relativas a $\xi'$ solo bastará con cambiar $l \rightarrow -l$ y 
$\phi \rightarrow \phi'$ a la cantidad relativa a $\xi$ lo cual simplificará 
bastante el trabajo.

\vspace{.3cm}

Debido a que el eje no tiene masa, solo los discos contribuyen a la energía 
cinética. Es fácil ver que el vector de rotación para el disco $\xi$ estará 
dado por $\mathbf{\omega} = \dot{\theta}\mathbf{K} + \dot{\phi}\mathbf{u}$. La 
energía cinética estará dada por la energía del centro de masa de la disco 
más la energía cinética de rotación,

\begin{equation}
 T_\xi = \frac{1}{2}\mathbf{V}_\xi^2 + T_\xi^{(rot)}.
\end{equation}

La energía rotacional se calcula simplemente utilizando $\mathbf{\omega}$ y
los momentos de inercia de los discos, $(T_\xi^{(rot)})_i = I_i(\mathbf{\omega}\cdot \mathbf{\omega})$, 
y ahora usando la expresión para $\mathbf{V}_\xi$ tenemos que 

\begin{equation}
 T_\xi = \frac{1}{2}m\left[\mathbf{V}_O^2 + (l^2/4)\dot{\theta}^2 +
 l\dot{\theta}\mathbf{v}\cdot\mathbf{V}_O \right]
 + \frac{1}{2}\left[ I\dot{\theta}^2 + I_3\dot{\phi}^2\right],
\end{equation}

y para $\xi'$

\begin{equation}
 T_{\xi'} = \frac{1}{2}m\left[\mathbf{V}_O^2 + (l^2/4)\dot{\theta}^2 
 - l\dot{\theta}\mathbf{v}\cdot\mathbf{V}_O \right]
 + \frac{1}{2}\left[ I\dot{\theta}^2 + I_3\dot{\phi}'^2\right].
\end{equation}

Ahora debido a que $\mathbf{V}_O^2 = \dot{X}^2 + \dot{Y}^2$ por la 
construcción que hemos hecho, y sumando las expresiones de las energías 
cinéticas para ambos discos tenemos que,

\begin{align*}
 T &= \frac{1}{2}m\left[(\dot{X}^2 + \dot{Y}^2) + (l^2/4)\dot{\theta}^2 +
 l\dot{\theta}\mathbf{v}\cdot\mathbf{V}_O \right]
 + \frac{1}{2}\left[ I\dot{\theta}^2 + I_3\dot{\phi}^2\right] \\
 %
 &+ \frac{1}{2}m\left[(\dot{X}^2 + \dot{Y}^2) + (l^2/4)\dot{\theta}^2 
 - l\dot{\theta}\mathbf{v}\cdot\mathbf{V}_O \right]
 + \frac{1}{2}\left[ I\dot{\theta}^2 + I_3\dot{\phi}'^2\right], \\
 %
   &= \frac{1}{2}m(\dot{X}^2 + \dot{Y}^2) + \frac{1}{2}m(l^2/4)\dot{\theta}^2 +
      \cancel{\frac{1}{2}ml\dot{\theta}\mathbf{v}\cdot\mathbf{V}_O}
      + \frac{1}{2}I\dot{\theta}^2 + \frac{1}{2}I_3\dot{\phi}^2 \\
  %
  &+ \frac{1}{2}m(\dot{X}^2 + \dot{Y}^2) + \frac{1}{2}m(l^2/4)\dot{\theta}^2 -
      \cancel{\frac{1}{2}ml\dot{\theta}\mathbf{v}\cdot\mathbf{V}_O}
      + \frac{1}{2}I\dot{\theta}^2 + \frac{1}{2}I_3\dot{\phi}'^2,
  %
\end{align*}

\begin{equation}
 \therefore T = m\left(\dot{X}^2 + \dot{Y}^2\right)  
 +\left( I + \frac{1}{4}ml^2 \right) \dot{\theta}^2 
 + \frac{1}{2}I_3 \left(\dot{\phi}^2 + \dot{\phi}'^2\right).
\end{equation}

Veremos que escribir las aceleraciones generalizadas nos será muy útil
para resolver este problema ya que lo haremos utilizando el método de 
los multiplicadores indeterminados de Lagrange. Recordando la expresión
para las aceleraciones generalizadas 

\begin{equation}
 A_i(q,\dot{q},\ddot{q},t) = \frac{d}{dt}\frac{\partial T(q,\dot{q},t)}{\partial\dot{q^i}}
 + \frac{\partial T(q,\dot{q},t)}{\partial q^i},
\end{equation}

tenemos entonces que,

\begin{align}
 A_X &= 2m\ddot{X}, \\
 A_Y &= 2m\ddot{Y}, \\
 A_\theta &= 2 \left( I + \frac{1}{4}ml^2 \right) \ddot{\theta}, \\
 A_\phi &= I_3\ddot{\phi}, \\
 A_{\phi'} &= I_3\ddot{\phi}''.
\end{align}

Necesitamos ahora expresar las condiciones para rodar sin resbalar. Sea 
$\zeta$ el punto de contacto del disco $\xi$con el plano; la condición que 
requerimos es que $\mathbf{V}_\zeta = 0$. Pero sabemos que para $\xi$, 
el vector de rotación instantánea en $\zeta$ puede escribirse como 

\begin{equation}
 \mathbf{V}_\zeta = \mathbf{V}_\xi + \mathbf{\omega}\times \mathbf{\xi\zeta} = 0.
\end{equation}

Esta condición nos dará dos condiciones escalares, lo mismo para 
el disco $\xi'$ que en el punto $\zeta'$ debe cumplirse que 

\begin{equation}
 \mathbf{V}_\zeta' = \mathbf{V}_\xi' + \mathbf{\omega}\times \mathbf{\xi'\zeta'} = 0.
\end{equation}

De estas cuatro condiciones (utilizando la figura \mref{fig:problema1fig2} y la ecuación para $\mathbf{V}_\xi$), 
dos son idénticas, y tenemos entonces tres ecuaciones de constricción,

\begin{align}
\label{eq:discosConstri01}
 \dot{X}\cos{\theta} + \dot{Y}\sen{\theta} = 0, \\
\label{eq:discosConstri02}
 \dot{Y}\cos{\theta} - \dot{X}\sen{\theta} + \frac{1}{2}l\dot{\theta} + R\dot{\phi} = 0, \\
\label{eq:discosConstri03}
 \dot{Y}\cos{\theta} - \dot{X}\sen{\theta} - \frac{1}{2}l\dot{\theta} + R\dot{\phi}' = 0,
\end{align}

Podemos escribir estas ecuaciones de una forma más informativa, haciendo 
un poco de álgebra, como

\begin{align}
\label{eq:discosConstri1}
 2\dot{X} - R (\dot{\phi} + \dot{\phi}')\sen{\theta} = 0, \\
\label{eq:discosConstri2}
 2\dot{Y} + R (\dot{\phi} + \dot{\phi}')\cos{\theta} = 0, \\
\label{eq:discosConstri3}
 l\dot{\theta} + R (\dot{\phi} - \dot{\phi}') = 0.
\end{align}

De estas constricciones, solo \mref{eq:discosConstri3} es holonómica, 
en cambio \mref{eq:discosConstri1} y \mref{eq:discosConstri2} son no holonómicas. 
Estas construcciones representan las constricciones requeridas para rodar 
sin resbalar. 

\vspace{.3cm}

Debido a que los puntos de aplicación para las fuerzas de reacción 
debidos al plano no se desplazan durante un desplazamiento virtual, 
por la condición de rodar sin resbalar, estas fuerzas de reacción no 
implican fuerzas generalizadas. El único trabajo virtual que no 
se hace cero es debido al peso. Podemos calcularla con un desplazamiento 
virtual del centro de masa $O$. Es fácil ver que 

\begin{equation}
 \delta W = - 2mg \delta z_O = -2mg\sen{\alpha}\delta Y,
\end{equation}

entonces la única fuerza generalizada que no se hace cero es 

\begin{equation}
 Q_Y = -2mg\sen{\alpha}.
\end{equation}

Ahora recordando la expresión para el formalismo de los multiplicadores 
indeterminados de Lagrange tenemos que, utilizando las expresiones 
para las aceleraciones generalizadas,

\begin{align}
 \label{eq:1lagrange1}
 m\ddot{X} = \lambda_1, \\
 \label{eq:1lagrange2}
 m\ddot{Y} = - mg\sen{\alpha} + \lambda_2, \\
 \label{eq:1lagrange3}
 2\left( I + \frac{1}{4}ml^2\right) \ddot{\theta} = l\lambda_3, \\
 \label{eq:1lagrange4}
 I_3\ddot{\phi} = -\lambda_1 R\sen{\theta} + \lambda_2 R\cos{\theta} + R\lambda_3, \\
 \label{eq:1lagrange5}
 I_3\ddot{\phi}' = -\lambda_1 R\sen{\theta} + \lambda_2 R\cos{\theta} - R\lambda_3.
\end{align}

Para resolver las ocho ecuaciones (las cinco ecuaciones de Lagrange más las tres 
ecuaciones de constricciones) hagamos el siguiente cambio de variable 
$\sigma = (\phi + \phi')/2$ y $\delta = (\phi - \phi')$. Con este cambio de variable 
la ecuación \mref{eq:discosConstri3} se convierte en 

\begin{equation}
 l\dot{\theta} + R\dot{\delta} = 0,
\end{equation}

y de las ecuaciones \mref{eq:1lagrange4,eq:1lagrange5} obtenemos que 

\begin{equation}
 \lambda_3 = \left(\frac{I_3}{2R}\right)\ddot{\delta}.
\end{equation}

Usando esta ecuación e introduciéndola en \mref{eq:1lagrange3} llegamos a que 

\begin{equation}
 \left[2I + \frac{1}{2}ml^2 + \left(\frac{I_3l^2}{2R^2}\right) \right]\ddot{\theta} = 0,
\end{equation}

de la cual obtenemos que $\ddot{\theta} = 0$, por lo que $\ddot{\theta}=0$. Lo cual 
quiere decir que el eje rota con una velocidad angular constante $\omega$, que con 
una selección conveniente del tiempo de origen podemos escribir $\theta = \omega t$. Entonces 
$\delta - \delta_0 = - (l\omega/R)t$ y $\lambda_3 = 0$. 

\vspace{.3cm}

Derivemos ahora las constricciones no holonómicas \mref{eq:discosConstri02} y \mref{eq:discosConstri03}
y sumémoslas, con lo cual luego de unas pocas líneas de álgebra encontramos que

\begin{equation}
 2\ddot{Y}\cos{\omega t} - 2\dot{Y}\omega\sen{\omega t} - 2\ddot{X}\sen{\omega t} 
 - 2\dot{X}\omega\cos{\omega t}  + 2R\ddot{\sigma} = 0
\end{equation}

pero por \mref{eq:discosConstri1} y \mref{eq:discosConstri2},

\begin{align}
\label{eq:discosConstri4}
 2\dot{X} = 2R\dot{\sigma}\sen{\omega t}, \\
\label{eq:discosConstri5}
 2\dot{Y} = -2R\dot{\sigma}\cos{\omega t}. 
\end{align}

y tomando en cuenta que $I_3 = mR^2/2$ y $I = mR^2/4$, haciendo la sustitución en 
\mref{eq:1lagrange1}, \mref{eq:1lagrange2}, \mref{eq:1lagrange4} y \mref{eq:1lagrange5},
tenemos que 

\begin{align}
 \label{eq:1lagrange6}
 \ddot{X} = \frac{\lambda_1}{m}, \\
 \label{eq:1lagrange7}
 \ddot{Y} = - g\sen{\alpha} + \frac{\lambda_2}{m}, \\
 \label{eq:1lagrange8}
 R\ddot{\phi} = \frac{2}{m}(-\lambda_1 \sen{\omega t} + \lambda_2 \cos{\omega t}), \\
 \label{eq:1lagrange9}
 R\ddot{\phi}' = \frac{2}{m}(-\lambda_1 \sen{\omega t} + \lambda_2 \cos{\omega t}).
\end{align}

Entonces usando \mref{eq:discosConstri4}, \mref{eq:discosConstri5}, \mref{eq:1lagrange6} y 
\mref{eq:1lagrange7}, tenemos que 

\begin{align*}
 2 \left(- g\sen{\alpha} + \frac{\lambda_2}{m}\right)\cos{\omega t} 
 +\cancel{2R\omega\dot{\sigma}\cos{\omega t}\sen{\omega t}} \\
 - 2\frac{\lambda_1}{m}\sen{\omega t} 
 - \cancel{2R\omega\dot{\sigma}\sen{\omega t}\cos{\omega t}}  + 2R\ddot{\sigma} = 0,
\end{align*}

\begin{equation}
 \therefore R\ddot{\sigma} = \frac{\lambda_1}{m}\sen{\omega t} + g\sen{\alpha}\cos{\omega t}  - \frac{\lambda_2}{m}\cos{\omega t}. 
\label{eq:1Rsigma1}
\end{equation}

Ahora si sumamos \mref{eq:1lagrange8} con \mref{eq:1lagrange9} obtenemos 

\begin{equation}
4 R \ddot{\sigma} = \frac{4}{m}(-\lambda_1 \sen{\omega t} + \lambda_2 \cos{\omega t}),
\end{equation}

o

\begin{equation}
 R\ddot{\sigma} = -\frac{\lambda_1}{m} \sen{\omega t} + \frac{\lambda_2}{m} \cos{\omega t}.
\label{eq:1Rsigma2}
\end{equation}

Si ahora sumamos \mref{eq:1Rsigma1} con \mref{eq:1Rsigma2}, obtenemos que 

\begin{equation}
 2R\ddot{\sigma} =  g\sen{\alpha}\cos{\omega t} 
\end{equation}

entonces

\begin{equation}
 \ddot{\sigma} = \frac{g\sen{\alpha}\cos{\omega t} }{2R}.
\end{equation}

De esta ecuación podemos encontrar fácilmente una expresión para $\dot{\sigma}$, tomando 
en cuanta de que como los discos inicialmente se encuentran reposo las condiciones 
iniciales serán $\dot{\sigma}(0) = 0$, $\dot{\phi}(0) = 0$ y $\dot{\phi'}(0) = 0$,

\begin{equation}
 \boxed{\dot{\sigma} = \frac{g\sen{\alpha}\sen{\omega t}}{2\omega R}.}
 \label{eq:1sigma1}
\end{equation}

Utilizando la ecuación \mref{eq:1sigma1} podemos encontrar una expresión para $X(t)$ utilizando  
\mref{eq:discosConstri4},

\begin{equation}
 \dot{X} = \frac{g\sen{\alpha}\sen^2{\omega t}}{2\omega},
\end{equation}

\begin{equation}
 \boxed{X = \frac{g\sen{\alpha}}{4\omega}\left(t - \frac{\sen{2\omega t}}{2\omega}\right) + X(0) .}
\end{equation}

Podemos hacer lo mismo para $Y(t)$ utilizando \mref{eq:discosConstri5},

\begin{equation}
 \dot{Y} = - \frac{g\sen{\alpha}\cos{\omega t}\sen{\omega t}}{2\omega},
\end{equation}

\begin{equation}
 \boxed{Y = \frac{g\sen{\alpha}\cos^2{\omega t}}{4\omega^2} + Y(0).}
\end{equation}

Ahora podemos encontrar las expresiones para los puntos de apoyo de los discos 
$\xi$ y $\xi'$, tenemos entonces que 

\begin{equation}
 \xi = X + Y + \frac{l}{2}(\cos{\omega t} + \sen{\omega t})
\end{equation}

\begin{equation}
 \boxed{\xi = \frac{g\sen{\alpha}}{4\omega}\left(t - \frac{\sen{2\omega t}}{2\omega} \right) + X(0) 
 + \frac{g\sen{\alpha}\cos^2{\omega t}}{4\omega^2} + Y(0) + 
 \frac{l}{2}(\cos{\omega t} + \sen{\omega t})}
\end{equation}

y para el disco $\xi'$ 

\begin{equation}
  \xi' = X + Y - \frac{l}{2}(\cos{\omega t} + \sen{\omega t})
\end{equation}

\begin{equation}
 \boxed{\xi = \frac{g\sen{\alpha}}{4\omega}\left(t - \frac{\sen{2\omega t}}{2\omega} \right) + X(0) 
 + \frac{g\sen{\alpha}\cos^2{\omega t}}{4\omega^2} + Y(0) - 
 \frac{l}{2}(\cos{\omega t} + \sen{\omega t})}
\end{equation}

Debajo se muestran varias gráficas que representan lo solicitado, unas son de $\xi$ vs. $t$ con 
$\xi'$ vs. $t$ y otras son para $\xi$ vs. $\xi'$, para distintos intervalos de tiempo y 
configuraciones iniciales, i.e. longitud de la barra, ángulo de inclinación, etc. Todos 
los códigos con los que se hicieron estas gráficas se encuentran en un repositorio público,
en un NoteBook de IPython, usando las librerías \texttt{mathplotlib} y \texttt{numpy}. Se 
puede acceder al NoteBook por \href{https://github.com/FavioVazquez/MecanicaClasica-PCF/blob/master/Tarea7/Problema1.ipynb}{\color{blue}.::este link::.}.
Vemos interesantes patrones oscilatorios para el gráfico $\xi$ vs. $\xi'$ lo cual es 
congruente con el sistema físico que nos planteamos.

\begin{figure}[H]
 \center 
 \includegraphics[scale=0.4]{problema1fig3}
 \caption{$X(0) = Y(0) = 5$, $\alpha = 30^\circ$, $t = [0,5]$, $l = 10$}
 \label{problema1fig3}
\end{figure}

\begin{figure}[H]
 \center 
 \includegraphics[scale=0.5]{problema1fig4}
 \caption{$X(0) = 10$, $Y(0) = 5$, $\alpha = 10^\circ$, $t = [0,5]$, $l = 10$}
 \label{problema1fig4}
\end{figure}

\begin{figure}[H]
 \center 
 \includegraphics[scale=0.5]{problema1fig5}
 \caption{$X(0) = Y(0) = 5$, $\alpha = 45^\circ$, $t = [0,5]$, $l = 50$}
 \label{problema1fig5}
\end{figure}

\begin{figure}[H]
 \center 
 \includegraphics[scale=0.5]{problema1fig6}
 \caption{$X(0) = Y(0) = 5$, $\alpha = 30^\circ$, $t = [0,10]$, $l = 10$}
 \label{problema1fig6}
\end{figure}

\begin{figure}[H]
 \center 
 \includegraphics[scale=0.5]{problema1fig7}
 \caption{$X(0) = Y(0) = 5$, $\alpha = 30^\circ$, $t = [0,10]$, $l = 10$}
 \label{problema1fig7}
\end{figure}

\begin{figure}[H]
 \center 
 \includegraphics[scale=0.5]{problema1fig8}
 \caption{$X(0) = Y(0) = 5$, $\alpha = 60^\circ$, $t = [0,10]$, $l = 50$}
 \label{problema1fig8}
\end{figure}

\begin{figure}[H]
 \center 
 \includegraphics[scale=0.5]{problema1fig9}
 \caption{$X(0) = Y(0) = 5$, $\alpha = 60^\circ$, $t = [0,10]$, $l = 50$}
 \label{problema1fig9}
\end{figure}

\begin{figure}[H]
 \center 
 \includegraphics[scale=0.5]{problema1fig10}
 \caption{$X(0) = Y(0) = 5$, $\alpha = 45^\circ$, $t = [0,10]$, $l = 2$}
 \label{problema1fig10}
\end{figure}







\section{Problema 2}

Un sistema consiste de una partícula de masa $m$ en el espacio físico y tiene una 
lagrangiana

$$
L = \frac{1}{2}m(\mathbf{\dot{r}} \cdot \mathbf{\dot{r}}) - V(\mathbf{r}).
$$

La partícula puede describir una trayectoria entre dos puntos $A$ y $B$ en un tiempo 
$t_{AB}$. Demuestre que una partícula de masa $M$ sujeta al mismo potencial podrá 
describir la misma trayectoria entre los puntos $A$ y $B$, pero que el tiempo que 
tardaría sería distinto. Calcule ese tiempo.

\vspace{.3cm}

\underline{Solución:} \vspace{.3cm}

Debido a que la lagrangiana no depende del tiempo y la energía potencial 
es una función homogénea cuadrática de las velocidades, se conserva la cantidad 
de Jacobi y esta es igual a la energía total del sistema, por lo tanto 
podemos escribir

\begin{equation}
 E = T + V = \frac{1}{2}m\dot{r}^2 + V,
\end{equation}

de esta ecuación podemos obtener el tiempo $t_{AB}$ que le toma a la primera 
partícula describir la trayectoria entre $AB$ como

\begin{equation}
 t_{AB} = \frac{\sqrt{m}}{\sqrt{2}} \int_A^B \frac{dr}{\sqrt{E - V}}.
 \label{eq:tiempo1}
\end{equation}

Por otra parte para la segunda partícula tendremos que

\begin{equation}
 E = \frac{1}{2}M\dot{r}^2 + V,
\end{equation}

por lo que el tiempo $\tau_{AB}$ que le toma a la segunda 
partícula describir la trayectoria entre $AB$ lo podemos escribir como

\begin{equation}
 t_{AB} = \frac{\sqrt{M}}{\sqrt{2}} \int_A^B \frac{dr}{\sqrt{E - V}}.
 \label{eq:tiempo2}
\end{equation}

Ahora dividiendo \mref{eq:tiempo1} entre \mref{eq:tiempo2} obtenemos que 

\begin{equation}
 \frac{t_{AB}}{\tau_{AB}} = \frac{\sqrt{m}/\sqrt{2}}{\sqrt{M}/\sqrt{2}} 
 \frac{\int_A^B \frac{dr}{\sqrt{E - V}}}{\int_A^B \frac{dr}{\sqrt{E - V}}}
\end{equation}

\begin{equation}
 \frac{t_{AB}}{\tau_{AB}} = \left(\frac{m}{M}\right)^{\frac{1}{2}}
\end{equation}

\begin{equation}
 \boxed{\therefore t_{AB} = \left(\frac{m}{M}\right)^{\frac{1}{2}}\tau_{AB}}
\end{equation}

Por lo tanto hemos demostrado que  partícula de masa M sujeta al mismo potencial 
describirá la misma trayectoria entre los puntos A y B, pero que el tiempo que tardaría sería distinto. Claramente cuando 
$m = M$ el tiempo será el mismo.





\section{Problema 3}

Una funcional $f(y)$ $(y(t) \in \mathbb{R}^n)$ se define por 

$$
f(y) = \int_{t_{i}}^{t_{f}} F(y,\dot{y},\ddot{y},y)dt.
$$

Encuentre las condiciones o ecuaciones que debe cumplir la curva $y$ para que $f$ sea
extremal cuando fijamos $y(t_i) = y_i$, $y(t_f)=y_f$, $\dot{y}(t_i) = v_i$ y
$\dot{y}(t_f) = v_f$.

\vspace{.3cm}

\underline{Solución:} \vspace{.3cm}

Comencemos calculando la diferencial de la funcional dada, claramente esta debe ser 
diferenciable para poder cumplirse lo que demostraremos en este problema. Usando 
el hecho de 

\begin{equation}
 F(c+h) - F(c) = DF_c[h]+O(|h|^2),
 \label{eq:diferencialFuncional1}
\end{equation}

donde $c$ y $h$ son curvas, y $Df_c$ es una funcional lineal, a la cual se le llama 
la diferencial de la funcional $f$ valuada en la curva $c$, tenemos entonces que, 
donde recordamos que $y = y(t)$, $\dot{y} = \dot{y}(t)$,$\ddot{y} = \ddot{y}(t)$ y $h=h(t)$, 
$\dot{h}=\dot{h}(t)$, $\ddot{h}=\ddot{h}(t)$

\begin{align}
\begin{split}
 f(y+h) - f(y) &= \int_{t_{i}}^{t_{f}} F(y+h,\dot{y}+ \dot{h}, \ddot{y} + \ddot{h}, t)dt -
 \int_{t_{i}}^{t_{f}} F(y,\dot{y},\ddot{y},y)dt, \\
	       &= \int_{t_{i}}^{t_{f}} \left[ F(y+h,\dot{y}+ \dot{h}, \ddot{y} + \ddot{h}, t)
	       -F(y,\dot{y},\ddot{y},y) \right]dt.
 \end{split}
\end{align}

Ahora, como $F$ es una función diferenciable de $\mathbb{R}^{n+n+n+1}$ en $R$ podemos 
escribir, utilizando \mref{eq:diferencialFuncional1}

\begin{equation}
 f(y+h) - f(y) =  \int_{t_{i}}^{t_{f}} \left\{DF_y[h,\dot{h},\ddot{h},t] + O(|h|^2) \right\}dt
\end{equation}

Luego,

\begin{align*}
 f(y+h) - f(y) = \int_{t_{i}}^{t_{f}} \left[ \left\zerodel \sum_{i=1}^n  \left( \frac{\partial F}{\partial y_i}h_i + 
 \frac{\partial F}{\partial \dot{y}_i}\dot{h}_i + \frac{\partial F}{\partial \ddot{y}_i}\ddot{h}_i\right)\right|_{y} 
 + O(|h|^2) \right]dt, \\
%  
	      = \sum_{i=1}^n \left[\int_{t_{i}}^{t_{f}}\left\zerodel\frac{\partial F}{\partial y_i}\right|_y h_i dt  
	      + \int_{t_{i}}^{t_{f}} \left\zerodel\frac{\partial F}{\partial \dot{y}_i}\right|_y \dot{h}_i dt
	       + \int_{t_{i}}^{t_{f}} \left\zerodel\frac{\partial F}{\partial \ddot{y}_i}\right|_y \ddot{h}_i dt\right] 
	       +  \int_{t_{i}}^{t_{f}} O(|h|^2)dt
 %
\end{align*}

Integrando por parte el segundo y el primer término tenemos que

\begin{equation}
  \int_{t_{i}}^{t_{f}} \left\zerodel\frac{\partial F}{\partial \dot{y}_i}\right|_y \dot{h}_i dt = 
  - \int_{t_{i}}^{t_{f}} \frac{d}{dt}\left(\left\zerodel\frac{\partial F}{\partial \dot{y}}\right)\right|_y h_i dt + 
  \left\zerodel\frac{\partial F}{\partial \dot{y}}\right|_y \left\zerodel h_i\right|_{t_{i}}^{t_{f}},
\end{equation}

y

\begin{equation}
  \int_{t_{i}}^{t_{f}} \left\zerodel\frac{\partial F}{\partial \ddot{y}_i}\right|_y \ddot{h}_i dt = 
  - \int_{t_{i}}^{t_{f}} \frac{d}{dt}\left(\left\zerodel\frac{\partial F}{\partial \ddot{y}}\right)\right|_y \dot{h}_i dt + 
  \left\zerodel\frac{\partial F}{\partial \ddot{y}}\right|_y \left\zerodel \dot{h}_i\right|_{t_{i}}^{t_{f}},
\end{equation}

Entonces tenemos que,

\begin{align}
 \begin{split}
  f(y+h) - f(y) =  &\sum_{i=1}^n \left[\int_{t_{i}}^{t_{f}}\left\zerodel\frac{\partial F}{\partial y_i}\right|_y h_i dt 
   - \int_{t_{i}}^{t_{f}} \frac{d}{dt}\left(\left\zerodel\frac{\partial F}{\partial \dot{y}}\right)\right|_y h_i dt + 
  \left\zerodel\frac{\partial F}{\partial \dot{y}}\right|_y \left\zerodel h_i\right|_{t_{i}}^{t_{f}}\right\zerodel \\
  %
  &- \left\zerodel\int_{t_{i}}^{t_{f}} \frac{d}{dt}\left(\left\zerodel\frac{\partial F}{\partial \ddot{y}}\right)\right|_y \dot{h}_i dt + 
  \left\zerodel\frac{\partial F}{\partial \ddot{y}}\right|_y \left\zerodel \dot{h}_i\right|_{t_{i}}^{t_{f}}\right]
  +  \int_{t_{i}}^{t_{f}} O(|h|^2)dt \\
  %
  &= \sum_{i=1}^n\left[\int_{t_{i}}^{t_{f}}\left\zerodel\frac{\partial F}{\partial y_i}\right|_y h_i dt 
   - \int_{t_{i}}^{t_{f}} \frac{d}{dt}\left(\left\zerodel\frac{\partial F}{\partial \dot{y}}\right)\right|_y h_i dt + 
  \left\zerodel\frac{\partial F}{\partial \dot{y}}\right|_y \left\zerodel h_i\right|_{t_{i}}^{t_{f}}\right\zerodel \\
  %
  &-\left\zerodel \int_{t_{i}}^{t_{f}} \frac{d^2}{dt^2}\left(\left\zerodel\frac{\partial F}{\partial \ddot{y}}\right)\right|_y h_i dt + 
  \left\zerodel\frac{\partial F}{\partial \ddot{y}}\right|_y \left\zerodel \dot{h}_i\right|_{t_{i}}^{t_{f}}\right]
  +  \int_{t_{i}}^{t_{f}} O(|h|^2)dt
  %
 \end{split}
\end{align}

Tenemos entonces que 

\begin{align*}
 f(y+h) - f(y) &= \sum_{i=1}^n \int_{t_{i}}^{t_{f}} \left[\frac{\partial F}{\partial y_i}  - 
 \frac{d}{dt}\frac{\partial F}{\partial \dot{y}}  
 - \frac{d^2}{dt^2}\frac{\partial F}{\partial \ddot{y}} \right]_y h_i dt \\
 &+\sum_{i=1}^n \left\zerodel \left\zerodel\frac{\partial F}{\partial \dot{y}}\right|_y h_i\right|_{t_{i}}^{t_{f}}
 + \sum_{i=1}^n \left\zerodel \left\zerodel\frac{\partial F}{\partial \ddot{y}}\right|_y\dot{h}_i\right|_{t_{i}}^{t_{f}}
 + \int_{t_{i}}^{t_{f}} O(|h|^2)dt
\end{align*}

Tenemos entonces que la diferencial está dada por

\begin{align}
\begin{split}
 DF_c[h] &=  \sum_{i=1}^n \int_{t_{i}}^{t_{f}} \left[\frac{\partial F}{\partial y_i}  - 
 \frac{d}{dt}\frac{\partial F}{\partial \dot{y}}  
 - \frac{d^2}{dt^2}\frac{\partial F}{\partial \ddot{y}} \right]_y h_i dt \\
 &+\sum_{i=1}^n \left\zerodel \left\zerodel\frac{\partial F}{\partial \dot{y}}\right|_y h_i\right|_{t_{i}}^{t_{f}}
 + \sum_{i=1}^n \left\zerodel \left\zerodel\frac{\partial F}{\partial \ddot{y}}\right|_y\dot{h}_i\right|_{t_{i}}^{t_{f}}
 \end{split}
 \label{eq:diferencialFuncional2}
\end{align}


Para resolver este problema, limitamos las posibles curvas a aquellas que 
para pasan por puntos fijos, y que cumplen con las condiciones del enunciado, con lo 
cual vemos que la variación de $h_i$ y $\dot{h}_i$ es cero evaluada entre $t_i$ y $t_f$, 
haciendo cero a los dos últimos términos de \mref{eq:diferencialFuncional2}, y entonces 
como el resto de las cantidades dentro del primer término de \mref{eq:diferencialFuncional2}
son independientes de $h(t)$, para que se cumpla que la curva $y$ para que $f$ sea 
extremal debe cumplirse que 

\begin{equation}
\boxed{\frac{\partial F}{\partial y_i}  - 
 \frac{d}{dt}\frac{\partial F}{\partial \dot{y}}  
 - \frac{d^2}{dt^2}\frac{\partial F}{\partial \ddot{y}} = 0.}
 \label{eq:condicionExtremal}
\end{equation}

Decimos entonces que la funcional $f$, y dentro del espacio de las curvas que pasan 
por un punto específico para $t=t_0$ y por otro para $t=t_1$, las curvas para que la 
funcional tiene una extremal están dadas por las ecuaciones \mref{eq:condicionExtremal}.



\section{Problema 4}

Demuestre que la formulación lagrangiana de la mecánica es invariante ante el grupo 
de Galileo.

\vspace{.3cm}

\underline{Solución:} \vspace{.3cm}

\textbf{Nota}: Luego de una charla con el profesor nos comentó que solo teníamos que probar 
la invariancia ante los ``boosts'', es decir las transformaciones cinemáticas..

\noindent\rule[0.5ex]{\linewidth}{1pt}

El marco inercial con el cual podemos describir un sistema mecánico no es único, de hecho 
existe un número infinito de marcos inerciales. Sea $S$ un marco inercial. Puede probarse 
\cite{tong} que hay $10$ transformaciones independientes $S\rightarrow S$ tales que 
$S'$ también sea un marco inercial. Estas son:

\begin{itemize}
 \item 3 Rotaciones: $\mathbf{r}' = O\mathbf{r}$ donde $O$ es una matriz ortogonal $3\times 3$.
 \item 3 Traslaciones: $\mathbf{r}' = \mathbf{r} + \mathbf{c}$ para un vector constante $\mathbf{c}$.
 \item 3 ``boosts'' (impulsos): $\mathbf{r}' = \mathbf{r} + \mathbf{v}t$ para una velocidad constante 
 $\mathbf{v}$.
 \item 1 Traslación temporal: $t' = t+c$ para una constante real $c$.
\end{itemize}

Estas transformaciones conforman el grupo de Galileo, el cual es un grupo de 10 parámetros. 
Como hemos dicho solo probaremos la invariancia ante los ``boosts''. Recordemos que podemos 
escribir la lagrangiana de la forma

\begin{equation}
 L = T-V = \frac{1}{2}m(\mathbf{v}\cdot \mathbf{v}) - V,
 \label{eq:lagrangianaGalileo1}
\end{equation}

y sea una transformación de la forma

\begin{equation}
 \mathbf{r}' = \mathbf{r} + \mathbf{v_0}t, \qquad \mathbf{v}' = \mathbf{v} + \mathbf{v_0}.
\label{eq:boost1}
\end{equation}

Entonces tenemos que la lagrangiana para el sistema $S'$ estará dada por

\begin{equation}
 L' = \frac{1}{2}m(\mathbf{v}'\cdot \mathbf{v}') - V,
\end{equation}

pero por \mref{eq:boost1} 

\begin{align*}
 L'&= \frac{1}{2}m[(\mathbf{v} + \mathbf{v_0})\cdot(\mathbf{v} + \mathbf{v_0})] - V, \\
   &= \frac{1}{2}m(\mathbf{v}\cdot \mathbf{v} + 2\mathbf{v}\cdot \mathbf{v}_0 + \mathbf{v}_0\cdot \mathbf{v}_0) - V, \\
   &= \frac{1}{2}m(\mathbf{v}\cdot \mathbf{v}) - V + \frac{1}{2}m(2\mathbf{v}\cdot \mathbf{v}_0) + \frac{1}{2}m(\mathbf{v}_0\cdot \mathbf{v}_0),
\end{align*}

pero por \mref{eq:lagrangianaGalileo1} tenemos que 

\begin{equation}
 L' = L + \frac{1}{2}m\mathbf{v}_0(2\mathbf{v}+\mathbf{v}_0)
 \label{eq:lagrangianaGalileo2}
\end{equation}

Ahora debido a que\footnote{Recordar que $\mathbf{v}_0$ es constante.}

\begin{equation}
 \frac{1}{2}m\mathbf{v}_0(2\mathbf{v}+\mathbf{v}_0) = \frac{d}{dt}\left[\frac{1}{2}m\mathbf{v}_0(2\mathbf{r}+\mathbf{v}_0t)\right]
 = \frac{d}{dt}\phi,
\end{equation}

donde $\phi \equiv \frac{1}{2}m\mathbf{v}_0(2\mathbf{r}+\mathbf{v}_0t)$. Entonces 
\mref{eq:lagrangianaGalileo2} se transforma en 

\begin{equation}
 \boxed{L' = L + \frac{d}{dt}\phi}
\end{equation}

Vemos entonces que debido a que como ya se ha demostrado en tareas anteriores, dos lagrangianas 
diferentes resultarán en ecuaciones de movimiento idénticas si difieren solamente 
por una función de las coordenadas y el tiempo. Con lo cual hemos demostrado que 
la formulación lagrangiana es invariante ante el grupo de Galileo, particularmente 
queríamos probar que era invariante ante transformaciones del tipo ``boost''.

\section{Problema 5}

Demuestre, con toda formalidad y usando el teorema de Noether, que la lagrangiana 
de una partícula libre en tres dimensiones es invariante ante el grupo de rotaciones 
$SO(3)$ y que por ello se conserva el vector de impulso angular.

\vspace{.3cm}

\underline{Solución:} \vspace{.3cm}

Comenzaremos con una breve digresión de teoría de grupos necesaria para poder dar una 
demostración completa y formal del enunciado de este problema. Utilizando principalmente 
el contenido básico de \cite{tung}. Además haremos una breve descripción del teorema 
de Noether para simetrías continuas tomado en gran parte de \cite{kleinert}.

\vspace{.3cm}

\definicion (Grupo): Se dice que un conjunto $\{G:a,b,c,\dots\}$ forma un \emph{grupo}
si existe una operación $\cdot$, llamada la \emph{multiplicación de grupo}, que asocia 
cualquier par de elementos (ordenados) dado $a,b \in G$ con un producto bien definido 
$a\cdot b$ que también es un elemento de $G$, tal que se satisfacen las siguientes 
condiciones 

\begin{enumerate}[label=(\alph*)]
 \item La operación $\cdot$ es \emph{asociativa}, i.e. $a\cdot (b\cdot c) = (a\cdot b)\cdot c$ 
 para todo $a,b,c \in G$,
 \item Entre los elementos de $G$, existe un elemento $e$, llamado la identidad, que 
 tiene la propiedad de que $a\cdot e = a$ para todo $a \in G$,
 \item Para cada $a \in G$, existe un elemento $a^{-1} \in G$, llamado la inversa 
 de $a$ que tiene la propiedad de que $a\cdot a^{-1} = e$.
\end{enumerate}

\definicion (Grupo Abeliano): Un \emph{grupo abeliano} $G$ es aquel en que la multiplicación
de grupo es conmutativa, i.e. $a\cdot b = b\cdot a$ para todo $a,b \in G$.

\vspace{.3cm}

De la definición anterior diremos que un grupo puede ser abeliano o no-abeliano.

\vspace{.3cm}

\definicion (Orden): El \emph{orden} de un grupo es el número de elementos del grupo 
(si es finito). 

\vspace{.3cm}

\definicion (Grupo continuo): Los \emph{grupos continuos} consisten en grupos de 
elementos etiquetados por una o más variables continuas, i.e. $(a_1,a_2,\dots,a_r)$, 
donde cada variable tiene un rango bien definido.

\vspace{.3cm}

Claramente todos los grupos continuos tienen un número infinito de elementos, aunque 
puede probarse que no todos los grupos infinitos son continuos. La teoría matemática 
general de grupos continuos es usualmente llamada la teoría de grupos de Lie. Simplificando,
podemos decir que un \emph{grupo de Lie} es un grupo infinito cuyos elementos pueden 
ser parametrizados suave y analíticamente; la formulación precisa de los grupos de 
Lie requiere un trato considerable ya que envuelve nociones de topología y geometría 
diferencial, las cuales salen del marco de desarrollo de este problema.

\vspace{.3cm}

Pasemos ahora a estudiar brevemente el grupo que nos interesa para solucionar este 
problema, \textbf{SO(3)}, el grupo continuo no-abeliano más útil \cite{tung}.

\vspace{.3cm}

\definicion (Grupo \textbf{SO(3)}): El \emph{grupo \textbf{SO(3)}} consiste en todas 
las transformaciones lineales continuas en el espacio euclideano tridimensional 
que dejan la longitud de los vectores coordenados invariantes. 

\vspace{.3cm}

Consideremos un sistema de coordenadas Cartesiano especificado por los vectores 
ortonormales $\hat{\mathbf{e}}_i$, $i = 1,2,3$. Bajo la rotación,

\begin{equation}
 \hat{\mathbf{e}}_i \xrightarrow{\mathbf{R}} \hat{\mathbf{e}}'_i = \hat{\mathbf{e}}_j R^j_i,
\end{equation}

donde $R^j_i$ son elementos de matrices rotacionales $3\times 3$. Sea $\mathbf{x}$ un 
vector arbitrario, $\mathbf{x} = \hat{\mathbf{e}}_i x^i$, entonces 
$\mathbf{x} \rightarrow \mathbf{x}'$ bajo la rotación $\mathbf{R}$ tal que

\begin{equation}
 x'^{i} = R^i_j x^j. 
\end{equation}

El requerimiento que $|\mathbf{x}| = |\mathbf{x}'|$, o $x_i x^j = x'_i x'^{j}$, resulta en 

\begin{equation}
 \mathbf{R}\mathbf{R}^T = \mathbf{R}^T\mathbf{R} = \mathbf{E},
 \label{eq:identidadSO31}
\end{equation}

donde $\mathbf{E}$ es la matriz identidad, para todas las matrices rotacionales. Las 
matrices reales que satisfacen esta condición tienen determinantes igual a $\pm 1$. Debido 
a que todas las rotaciones físicas pueden ser alcanzadas continuamente de la 
la transformación idéntica (ángulo cero de rotación), y debido a que el determinante 
de esta es $+1$, entonces todas las matrices de rotación deben tener determinante $+1$. Entonces 
en adición a la ecuación \mref{eq:identidadSO31}, las matrices $\mathbf{R}$ están restringidas 
por la condición

\begin{equation}
 \text{det } \mathbf{R} = 1.
 \label{eq:unoSO31}
\end{equation}

Ambas condiciones para la matriz rotacional $\mathbf{R}$ pueden ser expresadas 
en términos de \emph{tensores invariantes}. En particular la ecuación \mref{eq:identidadSO31} 
puede escribirse como

\begin{equation}
 R^i_k R^j_l \delta^{kl} = \delta^{ij}, 
 \label{eq:identidadSO32}
\end{equation}

que expresa la invariancia del tensor de segundo orden $\delta^{kl}$ (conocido como
la delta de Kronecker) bajo rotaciones. De forma similar, la ecuación \mref{eq:unoSO31} 
puede escribirse como

\begin{equation}
 R^i_l R^k_m R^k_n \epsilon^{lmn} = \epsilon^{ijk},
 \label{eq:unoSO32}
\end{equation}

donde $\epsilon^{ijk}$ es el tensor unitario de tercer orden totalmente antisimétrico (conocido 
como el tensor de Levi-Civita). Cuando $(i,j,k) = (1,2,3)$, el lado izquierdo 
de \mref{eq:unoSO32} es simplemente la determinante; y esta ecuación coincide con 
\mref{eq:unoSO31}. Para otros valores de $(i,j,k)$, la identidad resulta de la naturaleza 
totalmente antisimétrica de los ambos lados de la ecuación. Esta identidad establece 
el hecho de que $e^{lmn}$ es invariante ante rotaciones. 

\vspace{.3cm}

Pasemos ahora a un breve estudio de simetrías continuas, leyes de conservación y 
el teorema de Noether. Consideremos a sistema mecánico simple con una acción 
generadora

\begin{equation}
 \mathcal{A} = \int_{t_{a}}^{t_{b}} L(q(t),\dot{q}(t),t)dt.
 \label{eq:noet1}
\end{equation}

Supongamos que $\mathcal{A}$ es invariante bajo un conjunto continuo de transformaciones 
de las variables dinámicas:

\begin{equation}
 q(t) \rightarrow q'(t) = f(q(t),\dot{q}(t)),
 \label{eq:noet2}
\end{equation}

donde $f(q(t),\dot{q}(t))$ es alguna funcional de $q(t)$. Estas transformaciones 
son llamadas transformaciones de simetría. Si la acción se somete a dos transformaciones 
de simetría sucesivas, el resultado es de nuevo una transformación de simetría. Entonces, 
las transformaciones de simetría forman el llamado \emph{grupo de simetría} del sistema. 
Para una transformación infinitesimal \mref{eq:noet2}, la diferencia 

\begin{equation}
 \delta_s q(t) \equiv q'(t) - q(t),
 \label{eq:noet3}
\end{equation}

es llamada una \emph{variación de simetría}. Tiene la forma general 

\begin{equation}
 \delta_s = \epsilon \xi((q(t),\dot{q}(t),t)).
 \label{eq:noet4}
\end{equation}

Si calculamos el cambio de la acción bajo la acción de simetría \mref{eq:noet4}, utilizando
la regla de la cadena para la diferenciación, la integración por partes y el hecho
de que deben cumplirse las ecuaciones de Lagrange para las trayectorias de $q(t)$, obtendremos 
que la cantidad 

\begin{equation}
 I(t) = \frac{\partial L}{\partial \dot{q}}\xi(q,\dot{q},t),
 \label{eq:noet5}
\end{equation}

es la misma para los tiempo $t=t_a$ y $t=t_b$. Debido a que $t_b$ es arbitrario, 
$I(t)$ es independiente del tiempo, i.e., satisface

\begin{equation}
 I(T) \equiv I.
 \label{eq:noet6}
\end{equation}

Entonces $I$ es una cantidad conservada, una constante de movimiento. El lado derecho 
de \mref{eq:noet5} es llamada la \emph{carga de Noether}, y lo que hemos enunciado en 
este breve estudio es el conocido \emph{teorema de Noether}.

\vspace{.3cm}

Consideremos ahora la lagrangiana para una partícula libre en el espacio tridimensional

\begin{equation}
 L = L(q^i(t),\dot{q}^i(t),t).
\end{equation}

Bajo el grupo de rotaciones \textbf{SO(3)}, las coordenadas $q^i$ cambian a

\begin{equation}
 q'^i = R^i_j q^j,
\end{equation}

donde $ R^i_j$ es de la forma que hemos definido arriba. Infinitesimalmente, podemos 
escribir esto como

\begin{equation}
 R^i_j = \delta^i_j - \omega_l \epsilon_{kij},
\end{equation}

donde $\mathbf{\omega}$ es un vector de rotación infinitesimal. La correspondiente 
rotación del camino de la partícula es 

\begin{equation}
 \delta_s q^i(t) = q'^i(t) - q^i(t) = - \omega^k\epsilon_{kij}q^j.
\end{equation}

Es útil en este punto introducir el tensor de rotación antisimétrico infinitesimal 

\begin{equation}
 \omega_{ij} \equiv \omega_k \epsilon{kij},
\end{equation}

y en términos de este tenemos que

\begin{equation}
 \delta_s q^i = -\omega_{ij}q^j.
\end{equation}
 
Podemos ahora escribir el cambio en la lagrangiana bajo $\delta q^i$,

\begin{align}
\begin{split}
 \delta_s L &= L(q'^i(t),\dot{q}'^i,t) - L(q^i(t),\dot{q}^i,t), \\
	    &= \frac{\partial L}{\partial q^i}\delta_s q^i + \frac{\partial L}{\partial \dot{q}^i}\delta_s \dot{q}^i,
 \end{split}
\end{align}

como

\begin{equation}
 \delta_s L = - \left(\frac{\partial L}{\partial q^i}q^i +  
  \frac{\partial L}{\partial \dot{q}^i}\dot{q}^i\right)\omega_{ij} = 0.
\end{equation}

Si la lagrangiana depende solamente de invariantes de rotación, y sus potencias, entonces 
el lado derecho se hace cero debido a la antisimetría de $\omega_ij$. Esto asegura 
simetría rotacional. Podemos ahora hacer otra variación simétrica de la lagrangiana 
usando la regla de la cadena, usando las ecuaciones de Lagrange, 

\begin{align}
 \begin{split}
  \delta_s L &= \left( \frac{\partial L}{\partial q^i} - \frac{d}{dt} \frac{\partial L}{\partial \dot{q}^i} 
  \right)\delta_s q^i + \frac{d}{dt}\left( \frac{\partial L}{\partial \dot{q}^i}\delta_s q^i \right), \\
  %
	   &= -\frac{d}{dt}\left( \frac{\partial L}{\partial \dot{q}^i} \right)\omega_{ij} 
	   = \frac{1}{2}\frac{d}{dt}\left(q_i  \frac{\partial L}{\partial \dot{q}^i}  
	   -  q_j\frac{\partial L}{\partial \dot{q}^j}\right)\omega_{ij}.
 \end{split}
\end{align}

El lado derecho de esta ecuación nos da la carga de Noether conservada de la forma 
\mref{eq:noet5}, una para cada par antisimétrico $i,j$:

\begin{equation}
 \boxed{K^{ij} = x^i\frac{\partial L}{\partial \dot{q}^j} - x^j\frac{\partial L}{\partial \dot{q}^i} 
 = x^i p^j - x^j p^i.}
\end{equation}

Estas son las componentes antisimétricas del momentum angular $K^{ij}$. Si trabajamos 
con la forma vectorial original para los ángulos de rotación $\omega^k$, puede obtenerse 
de una manera muy similar el momentum angular en una forma más común:

\begin{equation}
 \boxed{K_k = \frac{1}{2}\epsilon_{kij}K^{ij} = (\mathbf{q} \times \mathbf{p})^k.}
\end{equation}

$\hspace{12cm} \square$

Hemos demostrado entonces lo solicitado, el hecho de que  la lagrangiana de
una partícula libre en tres dimensiones es invariante ante el grupo de rotaciones 
\textbf{SO(3)} que por ello se conserva el vector de momentum angular.

\section{Problema 6}

En el caso de una dependencia explícita del tiempo se utiliza una variedad de configuración 
extendida para tratar el teorema de Noether. En esta variedad se utiliza una lagrangiana 
extendida de la formalidad

$$
L_{ext} = L(q,\frac{\overline{q}}{\overline{t}},t)\overline{t},
$$

donde $L$ es la lagrangiana del sistema. ¿Cuál es la razón de utilizar esta forma y no 
otra que también se redujera a la lagrangiana del sistema cuando $\overline{t}=1$?

\vspace{.3cm}

\underline{Solución:} \vspace{.3cm}

Para entender por completo la solución que propondremos para este problema, tenemos que 
hablar un poco del \emph{método de coordenadas paramétricas}, del cual es que podremos 
entender el por qué de la forma de la lagrangiana extendida, y sus interpretaciones. Principalmente
utilizaremos contenido de \cite{johns} y de \cite{greiner}, con algunas interpretaciones 
propias y comentarios. Es \textbf{importante} resaltar que enunciaremos algunos teoremas 
pero no se probarán, el lector interesado puede acudir a la bibliografía mencionada.

\vspace{.3cm}

El uso de una coordenada como parámetro de integración constituye el llamado 
método de coordenadas paramétricas. Supongamos que, luego de re-arreglar las coordenadas 
si es necesario, denotamos la coordenada seleccionada como el parámetro de integración 
en el método de coordenadas paramétricas por $\xi_1$. Las derivadas de las otras coordenadas 
con respecto a $\xi_1$ serán denotadas por $\xi'_k$ de manera que $\xi'_k = d\xi_k/d\xi_1$. La integral 
a ser extremizada en el método de coordenadas paramétricas puede escribirse como

\begin{equation}
 \mathcal{I} = \int_{\xi_1^{(1)}}^{\xi_1^{(2)}} g(\xi,\xi'_{[1]})d\xi_1,
 \label{eq:6accion1}
\end{equation}

donde $\xi_2,\dots,\xi_N$ son las variables restantes y $\xi'_{[1]} = \xi'_2,\dots,\xi'_N$ son 
sus derivadas con respecto a $\xi_1$. (Acá $\xi$ quiere decir todas las variables, 
$\xi_1,\dots,\xi_N$, y $\xi_{[1]}$ quiere decir todas las variables \emph{excepto} $\xi_1$).

\vspace{.3cm}

\teorema (Coordenadas de Lagrange): Asumamos que la variable $\xi_1$ escogida como 
el parámetro de integración del método de coordenadas paramétricas varía monótonamente 
a lo largo del camino invariado. Entonces la variación de primer orden de la ecuación 
\mref{eq:6accion1} se hace cero, $\delta\mathcal{I}$, para variaciones arbitrarias 
de las variables $\xi_2,\dots,\xi_N$ con puntos finales fijos (y no una variación de 
$\xi_1$), sí y solo sí el camino invariado $\xi_k = \xi_k(\xi_1)$ es una solución 
de las ecuaciones de Lagrange

\begin{equation}
 \frac{d}{d\xi_1}\left(\frac{\partial g(\xi,\xi'_{[1]})}{\partial \xi'_k} \right)
 - \frac{\partial g(\xi,\xi'_{[1]})}{\partial \xi_l} = 0,
 \label{eq:6lagrangeEqs1}
\end{equation}

para $k=2,\dots,N$.

\vspace{.3cm}

Aunque acá no lo haremos, es fácil probar que este método a también puede ser utilizado 
para problemas con constricciones.

\vspace{.3cm}

Uno de los problemas con el método de coordenadas paramétricas es que tenemos $N$ coordenadas 
$\xi_1,\dots,\xi_N$ pero solo $N-1$ ecuaciones de Lagrange. Comparado con el método 
paramétrico general en el cual hay una ecuación de Lagrange para cada coordenada, 
en \mref{eq:6lagrangeEqs1} la ecuación con respecto a $\xi_1$ se ha perdido. Esta 
ecuación perdida puede recuperarse con lo que es llamada la \emph{segunda forma de 
las ecuaciones de Lagrange}. 

\vspace{.3cm}

\teorema (Segunda forma de las ecuaciones de Lagrange): La ecuación de Lagrange perdida 
en el método de coordenadas paramétricas puede recuperarse definiendo la segunda forma 
$h$ como

\begin{equation}
 h = \sum_{k=2}^{N} \xi'_k\frac{g(\xi,\xi'_{[1]})}{\partial \xi'_l} - g(\xi,\xi'_{[1]}),
\label{eq:6segundaForma}
\end{equation}

la ecuación de Lagrange perdida es entonces 

\begin{equation}
 \frac{dh}{d\xi_1} = - \frac{g(\xi,\xi'_{[1]})}{\partial \xi_1}.
 \label{eq:6lagrangeEqs2}
\end{equation}

Por último debemos establecer algo muy importante y relevante: cualquier problema 
en la forma de coordenadas paramétricas puede ser convertido a la forma paramétrica 
general. Introduciendo el parámetro monótono $\tau$ y escribiendo $d\xi = \overline{\xi_1}d\tau$ 
y $\xi'_k = \overline{\xi_k}/\overline{\xi_1}$, la integral de la ecuación 
\mref{eq:6accion1} se puede escribir como

\begin{equation}
  \mathcal{I} = \int_{\xi_1^{(1)}}^{\xi_1^{(2)}} g(\xi,\xi'_{[1]})d\xi_1 = 
  \int_{\tau_1}^{\tau_2} g\left(\xi,\frac{\overline{\xi}_{[1]}}{\overline{x_1}}\right) \overline{\xi_1}d\tau 
  =  \int_{\tau_1}^{\tau_2} f(\xi,\overline{\xi})d\tau,
  \label{eq:6accion2}
\end{equation}

donde

\begin{equation}
 f(\xi,\overline{\xi}) = g\left(\xi,\frac{\overline{\xi}_{[1]}}{\overline{x_1}}\right)\overline{\xi_1}.
 \label{eq:6motivacionLext}
\end{equation}

es el integrando usado para el método paramétrico general. Ahora hay $N$ ecuaciones de 
Lagrange, una para cada coordenada. 

\vspace{.3cm}

Podemos ahora comenzar a entender la forma de la lagrangiana extendida. Comencemos 
por escribir la lagrangiana tradicional

\begin{equation}
 L = L(q,\dot{q},t).
 \label{eq:6lagrangianaNormal}
\end{equation}

La motivación para la lagrangiana extendida surge de los conceptos y teoremas anteriores 
que hemos establecido para el método de coordenadas paramétricas y claramente 
del principio de Hamilton, en este caso el tiempo es removido de su rol como
parámetro de fondo universal, y se considera como una coordenada más, y el 
parámetro que utilizaremos como el nuevo ``tiempo'' será $\tau$. Entonces tendremos
que (recordamos que hemos denotado las derivadas con respecto a $\tau$ por una línea 
arriba de la variable, i.e. $\overline{t}$)

\begin{equation}
 \mathcal{I} = \int_{t_1}^{t_2} L(q,\dot{q},t)dt = 
 \int_{\tau_1}^{\tau_2} L\left(q,\frac{\overline{q}}{\overline{t}},t\right)\overline{t}d\tau.
\end{equation}

De donde podemos definir a partir de la lagrangiana \mref{eq:6lagrangianaNormal} 
la lagrangiana extendida $L_{ext}$ por 

\begin{equation}
 L_{ext} = L(q,\frac{\overline{q}}{\overline{t}},t)\overline{t},
 \label{eq:6lagrangianaExt}
\end{equation}

que tiene la misma forma que la establecida por el enunciado. Ahora entendemos que 
utilizamos esta forma para la lagrangiana extendida y no otra que también se redujera
a la lagrangiana del sistema cuando $\overline{t} = 1$, debido  a que 
es la ecuación a la que se llega en la prescripción descrita arriba para pasar del método 
de coordenadas paramétricas al de paramétricas generales. Esto es claro si identificamos 
que las ecuaciones \mref{eq:6lagrangianaExt} es la misma que \mref{eq:6motivacionLext} 
identificando $g\rightarrow L$, $\xi_1 \rightarrow t$, $\xi_1{[1]} \rightarrow q$ y 
$f \rightarrow L_{ext}$. \textbf{La cual es la forma necesaria para poder tener una ecuación 
de Lagrange para cada coordenada}, el cual es uno de los principales motivos. Con respecto 
al teorema de Noether, esta forma es la que necesitamos para que surja como cantidad 
conservada la negativa de la cantidad de Jacobi, y que cumpla con la motivación que 
vimos arriba y con algunos comentarios que expresamos a continuación.

\vspace{.3cm}

Hay otra motivación más por la cual se escoge esta expresión para $L_{ext}$ y no otra,
y es el hecho que por construcción $L_{ext}$ es una función homogénea de orden uno 
del conjunto de velocidades generalizadas $\overline{q}$. Lo cual, utilizando 
el teorema de Euler de las funciones homogéneas, significa que podemos encontrar 
una identidad para $L_{ext}$ que involucra a estas velocidades $\overline{q}$ y otra
identidad que involucra a las $\overline{t}$, por lo tanto no obtenemos una ecuación 
de movimiento sustancial para $t(\tau)$. En este caso la parametrización del 
tiempo $t(\tau)$ es dejada indeterminada, lo cual refleja el hecho de que la lagrangiana 
convencional no provee ninguna información sobre una parametrización del tiempo.

\vspace{.3cm}

Por otra parte, si la lagrangiana extendida $L_{ext}$ es la función principal que 
describe nuestro sistema, entonces $L_{ext}$ no es una función homogénea, en general. 
En ese caso la identidad de la que habíamos hablado antes, se convierte ahora en una 
función de constricción para el sistema. Y la ecuación de Lagrange que involucra 
$\overline{t}$ resulta en una ecuación de movimiento no trivial para $t(\tau)$. Entonces 
la lagrangiana tradicional se puede deducir de \mref{eq:6lagrangianaExt} por medio 
de la constricción. En este caso el número de grados de libertad no cambia, y el movimiento 
toma lugar en una hiper-superficie, definida por esta identidad, dentro del haz 
tangente $T(\mathbb{M}\times\mathbb{R})$ sobre la variedad de configuración espacio-temporal 
$\mathbb{M}\times\mathbb{R}$. Esto contrasta con la descripción lagrangiana sin 
constricciones en el haz tangente dependiente del tiempo $(T\mathbb{M})\times \mathbb{R}$.


\begin{thebibliography}{20}
\bibitem{ginoux}
C. Ginoux y B. Silvestre-Brac, \emph{Solved Problems in Lagrangian and Hamiltonian Mechanics},
Springer, 2009.
\bibitem{tong}
D. Tong, \emph{Classical Mechanics}, Cambridge University Press, 2005.
\bibitem{tung}
W. Tung, \emph{Group Theory in Physics}, World Scientific Publishing, 1985.
\bibitem{kleinert}
H. Kleinert, \emph{Particles and Quantum Fields}, Notas de clase 
disponibles en \href{http://users.physik.fu-berlin.de/~kleinert/b6/psfiles/}{http://users.physik.fu-berlin.de/~kleinert/b6/psfiles/}.
\bibitem{johns}
O. Johns \emph{Analytical Mechanics for Relativity and Quantum Mechanics}, Oxford 
University Press, 2005.
\bibitem{greiner}
W. Greiner, \emph{Classical Mechanics: Systems of Particles and Hamiltonian Dynamics},
2da edición, Springer-Verlang, 2010.
\end{thebibliography}


\end{document}
