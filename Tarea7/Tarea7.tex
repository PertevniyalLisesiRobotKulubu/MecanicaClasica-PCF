\documentclass[a4paper,10pt]{article}
\usepackage[utf8]{inputenc}
\usepackage[spanish]{babel}
\usepackage[affil-it]{authblk}
\usepackage{enumerate}
\usepackage{graphicx}
\usepackage{hyperref}
\usepackage{amsmath}
\usepackage{amssymb}
\usepackage{cancel}
\usepackage[usenames, dvipsnames]{color}
\usepackage{tikz}
\usepackage{multimedia}
\usepackage{subcaption} %Multiple images
\usepackage{multicol} % Multiple columns
\usepackage{float}
\usepackage{cleveref}
\usepackage[margin=1.4in]{geometry}
\usepackage[labelfont=bf]{caption}
\usepackage[titletoc,toc,title]{appendix}
\usetikzlibrary{calc}
\numberwithin{equation}{section}

%Appendices in spanish
\renewcommand{\appendixname}{Ap\'endices}
\renewcommand{\appendixtocname}{Ap\'endices}
\renewcommand{\appendixpagename}{Ap\'endices}

%Columns separation
\setlength{\columnsep}{1cm}

%Indentation
\setlength{\parindent}{0ex}

%Multiple References

\usepackage{xparse}
\ExplSyntaxOn
\NewDocumentCommand{\mref}{m}{\quinn_mref:n {#1}}
\seq_new:N \l_quinn_mref_seq
\cs_new:Npn \quinn_mref:n #1
 {
  \seq_set_split:Nnn \l_quinn_mref_seq { , } { #1 }
  \seq_pop_right:NN \l_quinn_mref_seq \l_tmpa_tl
  ( % print the left parenthesis
  \seq_map_inline:Nn \l_quinn_mref_seq
    { \ref{##1},\nobreakspace } % print the first references
  \exp_args:NV \ref \l_tmpa_tl 
  ) 
 }
\ExplSyntaxOff


%Boxes

\newcommand*{\boxcolor}{blue}
\makeatletter
\renewcommand{\boxed}[1]{\textcolor{\boxcolor}{%
\tikz[baseline={([yshift=-1ex]current bounding box.center)}] \node [rectangle, minimum width=1ex,rounded corners,draw] {\normalcolor\m@th$\displaystyle#1$};}}
 \makeatother

%Constantes
\newcommand{\euler}{\mathrm{e}}
\newcommand{\im}{i}

%Lemas, teoremas, definiciones y pruebas
\newcommand{\definicion}{\textbf{Definición: }}
\newcommand{\lema}{\textbf{Lema: }}
\newcommand{\teorema}{\textbf{Teorema: }}
\newcommand{\prueba}{\textbf{Prueba: }}


%opening
\title{Mecánica Clásica Tarea \# 7}
\author{Favio Vázquez\thanks{Correo: favio.vazquezp@gmail.com}}\affil{Instituto de Ciencias Nucleares. Universidad Nacional Autónoma de México.}
\date{}

\begin{document}

\makeatletter
\def\@maketitle{%
  \newpage
  \null
  \vskip 2em%
  \begin{center}%
  \let \footnote \thanks
    {\Large\bfseries \@title \par}%
    \vskip 1.5em%
    {\normalsize
      \lineskip .5em%
      \begin{tabular}[t]{c}%
        \@author
      \end{tabular}\par}%
    \vskip 1em%
    {\normalsize \@date}%
  \end{center}%
  \par
  \vskip 1.5em}
\makeatother

\maketitle

\section{Problema 1}

En presencia de la gravedad, dos discos uniformes de masa $m$ y radio $R$ unidos por 
un eje de longitud $l$ en torno al cual ambos giran libremente, descansan sobre un plano 
inclinado por un ángulo $\alpha$. Inicialmente se encuentran en reposo y el eje hace un ángulo 
$\beta$ con la dirección de máximo descenso. Si los discos ruedan sin resbalar determine 
las curvas que, sobre el plano, trazan los dos puntos de contacto entre los disco y el plano.

\vspace{.3cm}

\underline{Solución:} \vspace{.3cm}

\section{Problema 2}

Un sistema consiste de una partícula de masa $m$ en el espacio físico y tiene una 
lagrangiana

$$
L = \frac{1}{2}m(\mathbf{\dot{r}} + \mathbf{\dot{r}}) - V(\mathbf{r}).
$$

La partícula puede describir una trayectoria entre dos puntos $A$ y $B$ en un tiempo 
$t_{AB}$. Demuestre que una partícula de masa $M$ sujeta al mismo potencial podrá 
describir la misma trayectoria entre los puntos $A$ y $B$, pero que el tiempo que 
tardaría sería distinto. Calcule ese tiempo.

\vspace{.3cm}

\underline{Solución:} \vspace{.3cm}

\section{Problema 3}

Una funcional $f(y)$ $(y(t) \in \mathbb{R}^n)$ se define por 

$$
f(y) = \int_{t_{i}}^{t_{f}} F(y,\dot{y},\ddot{y},y)dt.
$$

Encuentre las condiciones o ecuaciones que debe cumplir la curva $y$ para que $f$ sea
extremal cuando fijamos $y(t_i) = y_i$, $y(t_f)=y_f$, $\dot{y}(t_i) = v_i$ y
$\dot{y}(t_f) = v_f$.

\vspace{.3cm}

\underline{Solución:} \vspace{.3cm}

\section{Problema 4}

Demuestre que la formulación lagrangiana de la mecánica es invariante ante el grupo 
de Galileo.

\vspace{.3cm}

\underline{Solución:} \vspace{.3cm}

\section{Problema 5}

Demuestre, con toda formalidad y usando el teorema de Noether, que la lagrangiana 
de una partícula libre en tres dimensiones es invariante ante el grupo de rotaciones 
$SO(3)$ y que por ello se conserva el vector de impulso angular.

\vspace{.3cm}

\underline{Solución:} \vspace{.3cm}

\section{Problema 6}

En el caso de una dependencia explícita del tiempo se utiliza una variedad de configuración 
extendida para tratar el teorema de Noether. En esta variedad se utiliza una lagrangiana 
extendida de la formalidad

$$
L_{ext} = L(q,\frac{\overline{q}}{\overline{t}})\overline{t},
$$

donde $L$ es la lagrangiana del sistema. ¿Cuál es la razón de utilizar esta forma y no 
otra que también se redujera a la lagrangiana del sistema cuando $\overline{r}=1$?

\vspace{.3cm}

\underline{Solución:} \vspace{.3cm}

\end{document}
