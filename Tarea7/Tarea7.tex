\documentclass[a4paper,10pt]{article}
\usepackage[utf8]{inputenc}
\usepackage[spanish]{babel}
\usepackage[affil-it]{authblk}
\usepackage{enumerate}
\usepackage{graphicx}
\usepackage{hyperref}
\usepackage{amsmath}
\usepackage{amssymb}
\usepackage{cancel}
\usepackage[usenames, dvipsnames]{color}
\usepackage{tikz}
\usepackage{multimedia}
\usepackage{subcaption} %Multiple images
\usepackage{multicol} % Multiple columns
\usepackage{float}
\usepackage{cleveref}
\usepackage[margin=1.4in]{geometry}
\usepackage[labelfont=bf]{caption}
\usepackage[titletoc,toc,title]{appendix}
\usetikzlibrary{calc}
\numberwithin{equation}{section}

%Appendices in spanish
\renewcommand{\appendixname}{Ap\'endices}
\renewcommand{\appendixtocname}{Ap\'endices}
\renewcommand{\appendixpagename}{Ap\'endices}

%Zero delimiter
\newcommand{\zerodel}{.\kern-\nulldelimiterspace}

%Columns separation
\setlength{\columnsep}{1cm}

%Indentation
\setlength{\parindent}{0ex}

%Multiple References

\usepackage{xparse}
\ExplSyntaxOn
\NewDocumentCommand{\mref}{m}{\quinn_mref:n {#1}}
\seq_new:N \l_quinn_mref_seq
\cs_new:Npn \quinn_mref:n #1
 {
  \seq_set_split:Nnn \l_quinn_mref_seq { , } { #1 }
  \seq_pop_right:NN \l_quinn_mref_seq \l_tmpa_tl
  ( % print the left parenthesis
  \seq_map_inline:Nn \l_quinn_mref_seq
    { \ref{##1},\nobreakspace } % print the first references
  \exp_args:NV \ref \l_tmpa_tl 
  ) 
 }
\ExplSyntaxOff


%Boxes

\newcommand*{\boxcolor}{blue}
\makeatletter
\renewcommand{\boxed}[1]{\textcolor{\boxcolor}{%
\tikz[baseline={([yshift=-1ex]current bounding box.center)}] \node [rectangle, minimum width=1ex,rounded corners,draw] {\normalcolor\m@th$\displaystyle#1$};}}
 \makeatother

%Constantes
\newcommand{\euler}{\mathrm{e}}
\newcommand{\im}{i}

%Lemas, teoremas, definiciones y pruebas
\newcommand{\definicion}{\textbf{Definición: }}
\newcommand{\lema}{\textbf{Lema: }}
\newcommand{\teorema}{\textbf{Teorema: }}
\newcommand{\prueba}{\textbf{Prueba: }}


%opening
\title{Mecánica Clásica Tarea \# 7}
\author{Favio Vázquez\thanks{Correo: favio.vazquezp@gmail.com}}\affil{Instituto de Ciencias Nucleares. Universidad Nacional Autónoma de México.}
\date{}

\begin{document}

\makeatletter
\def\@maketitle{%
  \newpage
  \null
  \vskip 2em%
  \begin{center}%
  \let \footnote \thanks
    {\Large\bfseries \@title \par}%
    \vskip 1.5em%
    {\normalsize
      \lineskip .5em%
      \begin{tabular}[t]{c}%
        \@author
      \end{tabular}\par}%
    \vskip 1em%
    {\normalsize \@date}%
  \end{center}%
  \par
  \vskip 1.5em}
\makeatother

\maketitle

\section{Problema 1}

En presencia de la gravedad, dos discos uniformes de masa $m$ y radio $R$ unidos por 
un eje de longitud $l$ en torno al cual ambos giran libremente, descansan sobre un plano 
inclinado por un ángulo $\alpha$. Inicialmente se encuentran en reposo y el eje hace un ángulo 
$\beta$ con la dirección de máximo descenso. Si los discos ruedan sin resbalar determine 
las curvas que, sobre el plano, trazan los dos puntos de contacto entre los disco y el plano.

\vspace{.3cm}

\underline{Solución:} \vspace{.3cm}

\section{Problema 2}

Un sistema consiste de una partícula de masa $m$ en el espacio físico y tiene una 
lagrangiana

$$
L = \frac{1}{2}m(\mathbf{\dot{r}} + \mathbf{\dot{r}}) - V(\mathbf{r}).
$$

La partícula puede describir una trayectoria entre dos puntos $A$ y $B$ en un tiempo 
$t_{AB}$. Demuestre que una partícula de masa $M$ sujeta al mismo potencial podrá 
describir la misma trayectoria entre los puntos $A$ y $B$, pero que el tiempo que 
tardaría sería distinto. Calcule ese tiempo.

\vspace{.3cm}

\underline{Solución:} \vspace{.3cm}

\section{Problema 3}

Una funcional $f(y)$ $(y(t) \in \mathbb{R}^n)$ se define por 

$$
f(y) = \int_{t_{i}}^{t_{f}} F(y,\dot{y},\ddot{y},y)dt.
$$

Encuentre las condiciones o ecuaciones que debe cumplir la curva $y$ para que $f$ sea
extremal cuando fijamos $y(t_i) = y_i$, $y(t_f)=y_f$, $\dot{y}(t_i) = v_i$ y
$\dot{y}(t_f) = v_f$.

\vspace{.3cm}

\underline{Solución:} \vspace{.3cm}

Comencemos calculando la diferencial de la funcional dada, claramente esta debe ser 
diferenciable para poder cumplirse lo que demostraremos en este problema. Usando 
el hecho de 

\begin{equation}
 F(c+h) - F(c) = DF_c[h]+O(|h|^2),
 \label{eq:diferencialFuncional1}
\end{equation}

donde $c$ y $h$ son curvas, y $Df_c$ es una funcional lineal, a la cual se le llama 
la diferencial de la funcional $f$ valuada en la curva $c$, tenemos entonces que, 
donde recordamos que $y = y(t)$, $\dot{y} = \dot{y}(t)$,$\ddot{y} = \ddot{y}(t)$ y $h=h(t)$, 
$\dot{h}=\dot{h}(t)$, $\ddot{h}=\ddot{h}(t)$

\begin{align}
\begin{split}
 f(y+h) - f(y) &= \int_{t_{i}}^{t_{f}} F(y+h,\dot{y}+ \dot{h}, \ddot{y} + \ddot{h}, t)dt -
 \int_{t_{i}}^{t_{f}} F(y,\dot{y},\ddot{y},y)dt, \\
	       &= \int_{t_{i}}^{t_{f}} \left[ F(y+h,\dot{y}+ \dot{h}, \ddot{y} + \ddot{h}, t)
	       -F(y,\dot{y},\ddot{y},y) \right]dt.
 \end{split}
\end{align}

Ahora, como $F$ es una función diferenciable de $\mathbb{R}^{n+n+n+1}$ en $R$ podemos 
escribir, utilizando \mref{eq:diferencialFuncional1}

\begin{equation}
 f(y+h) - f(y) =  \int_{t_{i}}^{t_{f}} \left\{DF_y[h,\dot{h},\ddot{h},t] + O(|h|^2) \right\}dt
\end{equation}

Luego,

\begin{align*}
 f(y+h) - f(y) = \int_{t_{i}}^{t_{f}} \left[ \left\zerodel \sum_{i=1}^n  \left( \frac{\partial F}{\partial y_i}h_i + 
 \frac{\partial F}{\partial \dot{y}_i}\dot{h}_i + \frac{\partial F}{\partial \ddot{y}_i}\ddot{h}_i\right)\right|_{y} 
 + O(|h|^2) \right]dt, \\
%  
	      = \sum_{i=1}^n \left[\int_{t_{i}}^{t_{f}}\left\zerodel\frac{\partial F}{\partial y_i}\right|_y h_i dt  
	      + \int_{t_{i}}^{t_{f}} \left\zerodel\frac{\partial F}{\partial \dot{y}_i}\right|_y \dot{h}_i dt
	       + \int_{t_{i}}^{t_{f}} \left\zerodel\frac{\partial F}{\partial \ddot{y}_i}\right|_y \ddot{h}_i dt\right] 
	       +  \int_{t_{i}}^{t_{f}} O(|h|^2)dt
 %
\end{align*}

Integrando por parte el segundo y el primer término tenemos que

\begin{equation}
  \int_{t_{i}}^{t_{f}} \left\zerodel\frac{\partial F}{\partial \dot{y}_i}\right|_y \dot{h}_i dt = 
  - \int_{t_{i}}^{t_{f}} \frac{d}{dt}\left(\left\zerodel\frac{\partial F}{\partial \dot{y}}\right)\right|_y h_i dt + 
  \left\zerodel\frac{\partial F}{\partial \dot{y}}\right|_y \left\zerodel h_i\right|_{t_{i}}^{t_{f}},
\end{equation}

y

\begin{equation}
  \int_{t_{i}}^{t_{f}} \left\zerodel\frac{\partial F}{\partial \ddot{y}_i}\right|_y \ddot{h}_i dt = 
  - \int_{t_{i}}^{t_{f}} \frac{d}{dt}\left(\left\zerodel\frac{\partial F}{\partial \ddot{y}}\right)\right|_y \dot{h}_i dt + 
  \left\zerodel\frac{\partial F}{\partial \ddot{y}}\right|_y \left\zerodel \dot{h}_i\right|_{t_{i}}^{t_{f}},
\end{equation}

Entonces tenemos que,

\begin{align}
 \begin{split}
  f(y+h) - f(y) =  &\sum_{i=1}^n \left[\int_{t_{i}}^{t_{f}}\left\zerodel\frac{\partial F}{\partial y_i}\right|_y h_i dt 
   - \int_{t_{i}}^{t_{f}} \frac{d}{dt}\left(\left\zerodel\frac{\partial F}{\partial \dot{y}}\right)\right|_y h_i dt + 
  \left\zerodel\frac{\partial F}{\partial \dot{y}}\right|_y \left\zerodel h_i\right|_{t_{i}}^{t_{f}}\right\zerodel \\
  %
  &- \left\zerodel\int_{t_{i}}^{t_{f}} \frac{d}{dt}\left(\left\zerodel\frac{\partial F}{\partial \ddot{y}}\right)\right|_y \dot{h}_i dt + 
  \left\zerodel\frac{\partial F}{\partial \ddot{y}}\right|_y \left\zerodel \dot{h}_i\right|_{t_{i}}^{t_{f}}\right]
  +  \int_{t_{i}}^{t_{f}} O(|h|^2)dt \\
  %
  &= \sum_{i=1}^n\left[\int_{t_{i}}^{t_{f}}\left\zerodel\frac{\partial F}{\partial y_i}\right|_y h_i dt 
   - \int_{t_{i}}^{t_{f}} \frac{d}{dt}\left(\left\zerodel\frac{\partial F}{\partial \dot{y}}\right)\right|_y h_i dt + 
  \left\zerodel\frac{\partial F}{\partial \dot{y}}\right|_y \left\zerodel h_i\right|_{t_{i}}^{t_{f}}\right\zerodel \\
  %
  &-\left\zerodel \int_{t_{i}}^{t_{f}} \frac{d^2}{dt^2}\left(\left\zerodel\frac{\partial F}{\partial \ddot{y}}\right)\right|_y h_i dt + 
  \left\zerodel\frac{\partial F}{\partial \ddot{y}}\right|_y \left\zerodel \dot{h}_i\right|_{t_{i}}^{t_{f}}\right]
  +  \int_{t_{i}}^{t_{f}} O(|h|^2)dt
  %
 \end{split}
\end{align}

Tenemos entonces que 

\begin{align*}
 f(y+h) - f(y) &= \sum_{i=1}^n \int_{t_{i}}^{t_{f}} \left[\frac{\partial F}{\partial y_i}  - 
 \frac{d}{dt}\frac{\partial F}{\partial \dot{y}}  
 - \frac{d^2}{dt^2}\frac{\partial F}{\partial \ddot{y}} \right]_y h_i dt \\
 &+\sum_{i=1}^n \left\zerodel \left\zerodel\frac{\partial F}{\partial \dot{y}}\right|_y h_i\right|_{t_{i}}^{t_{f}}
 + \sum_{i=1}^n \left\zerodel \left\zerodel\frac{\partial F}{\partial \ddot{y}}\right|_y\dot{h}_i\right|_{t_{i}}^{t_{f}}
 + \int_{t_{i}}^{t_{f}} O(|h|^2)dt
\end{align*}

Tenemos entonces que la diferencial está dada por

\begin{align}
\begin{split}
 DF_c[h] &=  \sum_{i=1}^n \int_{t_{i}}^{t_{f}} \left[\frac{\partial F}{\partial y_i}  - 
 \frac{d}{dt}\frac{\partial F}{\partial \dot{y}}  
 - \frac{d^2}{dt^2}\frac{\partial F}{\partial \ddot{y}} \right]_y h_i dt \\
 &+\sum_{i=1}^n \left\zerodel \left\zerodel\frac{\partial F}{\partial \dot{y}}\right|_y h_i\right|_{t_{i}}^{t_{f}}
 + \sum_{i=1}^n \left\zerodel \left\zerodel\frac{\partial F}{\partial \ddot{y}}\right|_y\dot{h}_i\right|_{t_{i}}^{t_{f}}
 \end{split}
 \label{eq:diferencialFuncional2}
\end{align}


Para resolver este problema, limitamos las posibles curvas a aquellas que 
para pasan por puntos fijos, y que cumplen con las condiciones del enunciado, con lo 
cual vemos que la variación de $h_i$ y $\dot{h}_i$ es cero evaluada entre $t_i$ y $t_f$, 
haciendo cero a los dos últimos términos de \mref{eq:diferencialFuncional2}, y entonces 
como el resto de las cantidades dentro del primer término de \mref{eq:diferencialFuncional2}
son independientes de $h(t)$, para que se cumpla que la curva $y$ para que $f$ sea 
extremal debe cumplirse que 

\begin{equation}
\boxed{\frac{\partial F}{\partial y_i}  - 
 \frac{d}{dt}\frac{\partial F}{\partial \dot{y}}  
 - \frac{d^2}{dt^2}\frac{\partial F}{\partial \ddot{y}} = 0.}
 \label{eq:condicionExtremal}
\end{equation}

Decimos entonces que la funcional $f$, y dentro del espacio de las curvas que pasan 
por un punto específico para $t=t_0$ y por otro para $t=t_1$, las curvas para que la 
funcional tiene una extremal están dadas por las ecuaciones \mref{eq:condicionExtremal}.



\section{Problema 4}

Demuestre que la formulación lagrangiana de la mecánica es invariante ante el grupo 
de Galileo.

\vspace{.3cm}

\underline{Solución:} \vspace{.3cm}

\section{Problema 5}

Demuestre, con toda formalidad y usando el teorema de Noether, que la lagrangiana 
de una partícula libre en tres dimensiones es invariante ante el grupo de rotaciones 
$SO(3)$ y que por ello se conserva el vector de impulso angular.

\vspace{.3cm}

\underline{Solución:} \vspace{.3cm}

\section{Problema 6}

En el caso de una dependencia explícita del tiempo se utiliza una variedad de configuración 
extendida para tratar el teorema de Noether. En esta variedad se utiliza una lagrangiana 
extendida de la formalidad

$$
L_{ext} = L(q,\frac{\overline{q}}{\overline{t}})\overline{t},
$$

donde $L$ es la lagrangiana del sistema. ¿Cuál es la razón de utilizar esta forma y no 
otra que también se redujera a la lagrangiana del sistema cuando $\overline{r}=1$?

\vspace{.3cm}

\underline{Solución:} \vspace{.3cm}

\end{document}
