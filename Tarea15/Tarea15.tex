\documentclass[a4paper,10pt]{article}
\usepackage[utf8]{inputenc}
\usepackage[spanish]{babel}
\usepackage[affil-it]{authblk}
\usepackage{enumerate}
\usepackage{graphicx}
\usepackage{hyperref}
\usepackage{amsmath}
\usepackage{amssymb}
\usepackage{cancel}
\usepackage[usenames, dvipsnames]{color}
\usepackage{tikz}
\usepackage[labelfont=bf]{caption}
\usepackage{subcaption} %Multiple images
\usepackage{multicol} % Multiple columns
\usepackage{float}
\usepackage{cleveref}
 \usepackage{relsize} % bigger math symbols
\usepackage[margin=1.4in]{geometry}
\usepackage[titletoc,toc,title]{appendix}
\usepackage{enumitem}
\usepackage{etoolbox}
\usepackage{mdframed} %frame theorems
\usetikzlibrary{calc}
\numberwithin{equation}{section}

% Enviroment for theorems
\newmdtheoremenv[frametitle=Teorema]{theo}{Theorem}

% Circled words
\newcommand{\circled}[2][]{%
  \tikz[baseline=(char.base)]{%
    \node[shape = circle, draw, inner sep = 1pt]
    (char) {\phantom{\ifblank{#1}{#2}{#1}}};%
    \node at (char.center) {\makebox[0pt][c]{#2}};}}
\robustify{\circled}

%Appendices in spanish
\renewcommand{\appendixname}{Ap\'endices}
\renewcommand{\appendixtocname}{Ap\'endices}
\renewcommand{\appendixpagename}{Ap\'endices}

%Zero delimiter
\newcommand{\zerodel}{.\kern-\nulldelimiterspace}

%Columns separation
\setlength{\columnsep}{1cm}

%Indentation
\setlength{\parindent}{0ex}

%Multiple References

\crefrangelabelformat{equation}{(#3#1#4--#5\crefstripprefix{#1}{#2}#6)}

\usepackage{xparse}

%Boxes

\newcommand*{\boxcolor}{blue}
\makeatletter
\renewcommand{\boxed}[1]{\textcolor{\boxcolor}{%
\tikz[baseline={([yshift=-1ex]current bounding box.center)}] \node [rectangle, minimum width=1ex,rounded corners,draw] {\normalcolor\m@th$\displaystyle#1$};}}
 \makeatother

%Constantes
\newcommand{\euler}{\mathrm{e}}
\newcommand{\im}{i}

%Lemas, teoremas, definiciones y pruebas
\newcommand{\definicion}{\textbf{Definición: }}
\newcommand{\lema}{\textbf{Lema: }}
\newcommand{\teorema}{\textbf{Teorema: }}
\newcommand{\prueba}{\textbf{Prueba: }}
\newcommand{\proposicion}{\textbf{Proposición: }}
\newcommand{\corolario}{\textbf{Corolario: }}

% Definición de las secciones y su numeración

\makeatletter
\def\@seccntformat#1{%
  \expandafter\ifx\csname c@#1\endcsname\c@section\else
  \csname the#1\endcsname\quad
  \fi}
\makeatother

%opening
\title{Mecánica Clásica Tarea \# 15}
\author{Favio Vázquez\thanks{Correo: favio.vazquezp@gmail.com}}\affil{Instituto de Ciencias Nucleares. Universidad Nacional Autónoma de México.}
\date{}

\begin{document}

\makeatletter
\def\@maketitle{%
  \newpage
  \null
  \vskip 2em%
  \begin{center}%
  \let \footnote \thanks
    {\Large\bfseries \@title \par}%
    \vskip 1.5em%
    {\normalsize
      \lineskip .5em%
      \begin{tabular}[t]{c}%
        \@author
      \end{tabular}\par}%
    \vskip 1em%
    {\normalsize \@date}%
  \end{center}%
  \par
  \vskip 1.5em}
\makeatother

\maketitle

\textbf{NOTA}: En los billares hay trayectorias que son singulares que son aquellas
en las que la partícula llega a incidir exactamente en uno de los vértices. Desprecie 
esta posibilidad limitando sus cálculos al conjunto de trayectorias no singulares.

\vspace{.3cm}

\section{Problema 1}

Una partícula de masa $m$ se mueve libremente en el interior de un rectángulo de 
lados $a$ y $b$ y rebota elásticamente al incidir sobre los lados del rectángulo 
(billar). Encuentre, si es que existen, unas coordenadas o variables de acción 
y ángulo para este sistema. A partir de estas variables encuentre, si es que existen,
las frecuencias asociadas al movimiento. 

\vspace{.3cm}

\underline{Solución:} \vspace{.3cm}

\section{Problema 2}

Encuentre coordenadas de acción y ángulo para el péndulo esférico (puede dejar 
indicadas algunas integrales). Trace figuras, similares a las que se trazaron 
en clase para el problema de Kepler, en las que se muestra que las trayectorias 
están sobre toros de dos dimensiones inmersos en el espacio fase de cuatro. 
(Use la computadora para hacer las figuras).

\vspace{.3cm}

\underline{Solución:} \vspace{.3cm}

\section{Problema 3}

Una partícula de masa $m$ se mueve libremente en el interior de un triángulo  
equilátero de lado $a$ y rebota elásticamente al incidir sobre los lados del triángulo 
(billar). Encuentre, si es que existen, unas coordenadas o variables de acción 
y ángulo para este sistema. A partir de estas variables encuentre, si es que existen,
las frecuencias asociadas al movimiento. 

\vspace{.3cm}

\underline{Solución:} \vspace{.3cm}

\end{document}