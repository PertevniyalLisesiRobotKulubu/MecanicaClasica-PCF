\documentclass[a4paper,10pt]{article}
\usepackage[utf8]{inputenc}
\usepackage[spanish]{babel}
\usepackage[affil-it]{authblk}
\usepackage{enumerate}
\usepackage{graphicx}
\usepackage{hyperref}
\usepackage{amsmath}
\usepackage{amssymb}
\usepackage{cancel}
\usepackage[usenames, dvipsnames]{color}
\usepackage{tikz}
\usepackage{multimedia}
\usepackage{subcaption} %Multiple images
\usepackage{multicol} % Multiple columns
\usepackage{float}
\usepackage{cleveref}
 \usepackage{relsize} %bigger math symbols
\usepackage[margin=1.4in]{geometry}
\usepackage[labelfont=bf]{caption}
\usepackage[titletoc,toc,title]{appendix}
\usepackage{enumitem}
\usetikzlibrary{calc}
\numberwithin{equation}{section}

%Appendices in spanish
\renewcommand{\appendixname}{Ap\'endices}
\renewcommand{\appendixtocname}{Ap\'endices}
\renewcommand{\appendixpagename}{Ap\'endices}

%Zero delimiter
\newcommand{\zerodel}{.\kern-\nulldelimiterspace}

%Columns separation
\setlength{\columnsep}{1cm}

%Indentation
\setlength{\parindent}{0ex}

%Multiple References

\crefrangelabelformat{equation}{(#3#1#4--#5\crefstripprefix{#1}{#2}#6)}

\usepackage{xparse}
\ExplSyntaxOn
\NewDocumentCommand{\mref}{m}{\quinn_mref:n {#1}}
\seq_new:N \l_quinn_mref_seq
\cs_new:Npn \quinn_mref:n #1
 {
  \seq_set_split:Nnn \l_quinn_mref_seq { , } { #1 }
  \seq_pop_right:NN \l_quinn_mref_seq \l_tmpa_tl
  ( % print the left parenthesis
  \seq_map_inline:Nn \l_quinn_mref_seq
    { \ref{##1},\nobreakspace } % print the first references
  \exp_args:NV \ref \l_tmpa_tl 
  ) 
 }
\ExplSyntaxOff


%Boxes

\newcommand*{\boxcolor}{blue}
\makeatletter
\renewcommand{\boxed}[1]{\textcolor{\boxcolor}{%
\tikz[baseline={([yshift=-1ex]current bounding box.center)}] \node [rectangle, minimum width=1ex,rounded corners,draw] {\normalcolor\m@th$\displaystyle#1$};}}
 \makeatother

%Constantes
\newcommand{\euler}{\mathrm{e}}
\newcommand{\im}{i}

%Lemas, teoremas, definiciones y pruebas
\newcommand{\definicion}{\textbf{Definición: }}
\newcommand{\lema}{\textbf{Lema: }}
\newcommand{\teorema}{\textbf{Teorema: }}
\newcommand{\prueba}{\textbf{Prueba: }}
\newcommand{\proposicion}{\textbf{Proposición: }}
\newcommand{\corolario}{\textbf{Corolario: }}


%opening
\title{Mecánica Clásica Tarea \# 9}
\author{Favio Vázquez\thanks{Correo: favio.vazquezp@gmail.com}}\affil{Instituto de Ciencias Nucleares. Universidad Nacional Autónoma de México.}
\date{}

\begin{document}

\makeatletter
\def\@maketitle{%
  \newpage
  \null
  \vskip 2em%
  \begin{center}%
  \let \footnote \thanks
    {\Large\bfseries \@title \par}%
    \vskip 1.5em%
    {\normalsize
      \lineskip .5em%
      \begin{tabular}[t]{c}%
        \@author
      \end{tabular}\par}%
    \vskip 1em%
    {\normalsize \@date}%
  \end{center}%
  \par
  \vskip 1.5em}
\makeatother

\maketitle

\section{Problema 1}

Tres masas iguales están acopladas por resortes iguales en las siguientes configuraciones:

$$
M - - k - - M - - k - - M
$$

$$
] - - k - - M - - k - - M - - k - - M
$$

$$ 
] - - k - - M - - k - - M - - k - - M - - k - -[
$$

$$
(M = \text{masa}\quad - - k - - = \text{resorte} \quad ][ = \text{pared})
$$

Para cada una de ellas encuentre los modos y las frecuencias normales de oscilación.

\vspace{.3cm}

\underline{Solución:} \vspace{.3cm}

\section{Problema 2}

Cuatro masas están acopladas cíclicamente por resortes iguales de constante $K$ con 
excepción de uno que tiene constante $k \ll K$. Encuentre a primer orden en $k$ los 
modos normales de oscilación. Hágalo para el caso en que las masas son distintas 
y para el caso en que son iguales. ¿Se podrá resolver este sistema de forma exacta?

\vspace{.3cm}

\underline{Solución:} \vspace{.3cm}

\section{Problema 3}

Encuentre las frecuencias normales de oscilación de un péndulo esférico doble: (una 
masa $m_1$ se mueve sobre una esfera de radio $R_1$ centrada en un punto fijo del 
espacio, otra masa $m_2$ se mueve sobre una esfera de radio $R_2$ centrada en la 
primera masa, todo ellos en presencia de la gravedad). Describa los modos normales 
de este sistema.

\vspace{.3cm}

\underline{Solución:} \vspace{.3cm}

\section{Problema 4}

¿Seŕa válida alguna versión del teorema de Noether para densidades lagrangianas? Argumente 
su respuesta.

\vspace{.3cm}

\underline{Solución:} \vspace{.3cm}

\section{Problema 5}

Obtenga las ecuaciones de movimiento correspondientes a la densidad lagrangiana 

$$
\mathcal{L} = i\frac{a}{2} \left( \phi^*dot{\phi} - \dot{\phi}^*\phi \right)  
- \frac{a^2}{2m}\nabla \phi^* \cdot \nabla \phi - \phi^*\phi V(x,y,z),
$$

donde el asterisco denota conjugación completa, $a$ es una constante, $V$ un 
potencial e $i=\sqrt{-1}$. Comente sobre las ecuaciones que resulten.

\vspace{.3cm}

\underline{Solución:} \vspace{.3cm}

\end{document}