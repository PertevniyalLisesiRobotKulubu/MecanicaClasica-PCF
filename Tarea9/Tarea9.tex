\documentclass[a4paper,10pt]{article}
\usepackage[utf8]{inputenc}
\usepackage[spanish]{babel}
\usepackage[affil-it]{authblk}
\usepackage{enumerate}
\usepackage{graphicx}
\usepackage{hyperref}
\usepackage{amsmath}
\usepackage{amssymb}
\usepackage{cancel}
\usepackage[usenames, dvipsnames]{color}
\usepackage{tikz}
\usepackage{multimedia}
\usepackage{subcaption} %Multiple images
\usepackage{multicol} % Multiple columns
\usepackage{float}
\usepackage{cleveref}
 \usepackage{relsize} %bigger math symbols
\usepackage[margin=1.4in]{geometry}
\usepackage[labelfont=bf]{caption}
\usepackage[titletoc,toc,title]{appendix}
\usepackage{enumitem}
\usetikzlibrary{calc}
\numberwithin{equation}{section}

%Appendices in spanish
\renewcommand{\appendixname}{Ap\'endices}
\renewcommand{\appendixtocname}{Ap\'endices}
\renewcommand{\appendixpagename}{Ap\'endices}

%Zero delimiter
\newcommand{\zerodel}{.\kern-\nulldelimiterspace}

%Columns separation
\setlength{\columnsep}{1cm}

%Indentation
\setlength{\parindent}{0ex}

%Multiple References

\crefrangelabelformat{equation}{(#3#1#4--#5\crefstripprefix{#1}{#2}#6)}

\usepackage{xparse}
\ExplSyntaxOn
\NewDocumentCommand{\mref}{m}{\quinn_mref:n {#1}}
\seq_new:N \l_quinn_mref_seq
\cs_new:Npn \quinn_mref:n #1
 {
  \seq_set_split:Nnn \l_quinn_mref_seq { , } { #1 }
  \seq_pop_right:NN \l_quinn_mref_seq \l_tmpa_tl
  ( % print the left parenthesis
  \seq_map_inline:Nn \l_quinn_mref_seq
    { \ref{##1},\nobreakspace } % print the first references
  \exp_args:NV \ref \l_tmpa_tl 
  ) 
 }
\ExplSyntaxOff


%Boxes

\newcommand*{\boxcolor}{blue}
\makeatletter
\renewcommand{\boxed}[1]{\textcolor{\boxcolor}{%
\tikz[baseline={([yshift=-1ex]current bounding box.center)}] \node [rectangle, minimum width=1ex,rounded corners,draw] {\normalcolor\m@th$\displaystyle#1$};}}
 \makeatother

%Constantes
\newcommand{\euler}{\mathrm{e}}
\newcommand{\im}{i}

%Lemas, teoremas, definiciones y pruebas
\newcommand{\definicion}{\textbf{Definición: }}
\newcommand{\lema}{\textbf{Lema: }}
\newcommand{\teorema}{\textbf{Teorema: }}
\newcommand{\prueba}{\textbf{Prueba: }}
\newcommand{\proposicion}{\textbf{Proposición: }}
\newcommand{\corolario}{\textbf{Corolario: }}


%opening
\title{Mecánica Clásica Tarea \# 9}
\author{Favio Vázquez\thanks{Correo: favio.vazquezp@gmail.com}}\affil{Instituto de Ciencias Nucleares. Universidad Nacional Autónoma de México.}
\date{}

\begin{document}

\makeatletter
\def\@maketitle{%
  \newpage
  \null
  \vskip 2em%
  \begin{center}%
  \let \footnote \thanks
    {\Large\bfseries \@title \par}%
    \vskip 1.5em%
    {\normalsize
      \lineskip .5em%
      \begin{tabular}[t]{c}%
        \@author
      \end{tabular}\par}%
    \vskip 1em%
    {\normalsize \@date}%
  \end{center}%
  \par
  \vskip 1.5em}
\makeatother

\maketitle

\section{Problema 1}

Tres masas iguales están acopladas por resortes iguales en las siguientes configuraciones:

$$
M - - k - - M - - k - - M
$$

$$
] - - k - - M - - k - - M - - k - - M
$$

$$ 
] - - k - - M - - k - - M - - k - - M - - k - -[
$$

$$
(M = \text{masa}\quad - - k - - = \text{resorte} \quad ][ = \text{pared})
$$

Para cada una de ellas encuentre los modos y las frecuencias normales de oscilación.

\vspace{.3cm}

\underline{Solución:} \vspace{.3cm}

\section{Problema 2}

Cuatro masas están acopladas cíclicamente por resortes iguales de constante $K$ con 
excepción de uno que tiene constante $k \ll K$. Encuentre a primer orden en $k$ los 
modos normales de oscilación. Hágalo para el caso en que las masas son distintas 
y para el caso en que son iguales. ¿Se podrá resolver este sistema de forma exacta?

\vspace{.3cm}

\underline{Solución:} \vspace{.3cm}

\section{Problema 3}

Encuentre las frecuencias normales de oscilación de un péndulo esférico doble: (una 
masa $m_1$ se mueve sobre una esfera de radio $R_1$ centrada en un punto fijo del 
espacio, otra masa $m_2$ se mueve sobre una esfera de radio $R_2$ centrada en la 
primera masa, todo ellos en presencia de la gravedad). Describa los modos normales 
de este sistema.

\vspace{.3cm}

\underline{Solución:} \vspace{.3cm}

\section{Problema 4}

¿Seŕa válida alguna versión del teorema de Noether para densidades lagrangianas? Argumente 
su respuesta.

\vspace{.3cm}

\underline{Solución:} \vspace{.3cm}

Sin duda uno de los artículos más importantes y relevantes en la carrera de 
Emmy Noether fue su \emph{Invariante Variationsprobleme} \cite{noether}, el cual también 
fue muy importante para el desarrollo de le mecánica teórica en el resto del siglo 
XX. El artículo contenía dos teoremas, que fueron prácticamente olvidados unos pocos 
años después de sus publicación, pero su influencia desde 1950 fue muy importante. 
El primero tenía que ver con un problema variacional bajo la acción de un grupo de 
Lie con número finito de generadores infinitesimales independientes, el cual es 
la situación típica en mecánica clásica y relatividad especial. En este teorema, 
el cual es comúnmente referido como ``\emph{el} teorema de Noether'', ella formuló, 
con completa generalidad, la correspondencia entre simetrías de un problema variacional 
y las leyes de conservación para las ecuaciones variacionales asociadas. Este teorema 
tendría unas importantes consecuencias en la mecánica cuántica, sirviendo de guía 
en la correspondencia de cantidades conservadas asociadas con invariancias, y se 
convirtió en la base para la teoría de corrientes. Su segundo teorema trataba con 
la invariancia de un problema variacional bajo la acción de un grupo que involucraba 
funciones arbitrarias, una situación fundamental en relatividad general y las 
teorías de ``gauge'' (referidas en español como teorías de calibre en algunos países).

\vspace{.3cm}

En anteriores tareas hemos demostrado algunas propiedades fundamentales del teorema de 
Noether para la dinámica de partículas, pero de hecho, Noether probó este teorema 
originalmente para campos, no para partículas. Veremos entonces que pueden encontrarse 
constantes de movimiento, o cantidades conservadas en la dinámica continua, usando 
una formulación de densidades lagrangianas vía el teorema de Noether.

\vspace{.3cm}

Para derivar el teorema en la mecánica de partículas, se muestra que cada $q$-simetría 
de la lagrangiana $L$ corresponde a una constante de movimiento $I(q,\dot{q})$. Una 
$q$-simetría es una $q$-transformación de la forma $q(t) \mapsto q(t;\epsilon)$, que 
deja a $L$ invariante. El que $I$ sea una constante de movimiento significa que 
$dI(q,\dot{q})/dt = 0$. El análogo en dinámica de sistemas continuos, en teorías 
de campo, de las $q$-transformaciones son transformaciones de las variables de 
campo $\phi_s$. Las simetrías serán de la densidad lagrangiana $\mathcal{L}$ en vez 
de la lagrangiana $L$. El análogo del tiempo $t$ será el conjunto de variables 
$x^0, x^1, x^2, x^3$, por lo que la conclusión del problema no involucrará sólo 
derivadas temporales de alguna función, sino que también incluirá derivadas con 
respecto a todas las $x^\alpha$, donde $\alpha \in [0,3]$. 

\vspace{.3cm}

Consideremos una $\epsilon$-familia $\phi_s(x;\epsilon)$ de transformaciones de las 
variables de campo. En $\epsilon = 0$ los campos $\phi_s(x;0)$ son soluciones de 
las ecuaciones de Lagrange (i.e., $\delta L / \delta \phi_s = 0$). Las transformaciones 
no están restringidas a alguna región $R$ del 4-espacio y no se hacen cero en 
ninguna subvariedad. Si escribimos la variación de la funcional de acción para este 
caso como

\begin{equation}
 \delta S \equiv \delta \int_R \mathcal{L} d^4 x = 0,
 \label{eq:noet1}
\end{equation}

donde $R$ es una región del 4-espacio, luego de alguna manipulaciones llegamos a\footnote{ 
$\partial \phi_{s,\alpha} = \partial \phi_s / \partial \alpha$.} 

\begin{equation}
 \delta \mathcal{L} = \left[ \frac{\partial \mathcal{L}}{\partial \phi_s} 
 - \frac{\partial}{\partial x^\alpha}\frac{\mathcal{L}}{\partial \phi_{s,\alpha}} \right] 
 \delta \phi_s + \frac{\partial}{\partial x^\alpha}\left[\frac{\mathcal{L}}{\partial \phi_{s,\alpha}} 
 \delta \phi_s \right].
 \label{eq:noet2}
\end{equation}

Entonces la ecuación \mref{eq:noet1} se convierte en 

\begin{equation}
 0 = \int_R \delta \mathcal{L} = \left[ \frac{\partial \mathcal{L}}{\partial \phi_s} 
 - \frac{\partial}{\partial x^\alpha}\frac{\mathcal{L}}{\partial \phi_{s,\alpha}} \right] 
 \delta \phi_s d^4 x + \int_R \frac{\partial}{\partial x^\alpha}\left[\frac{\mathcal{L}}{\partial \phi_{s,\alpha}} 
 \delta \phi_s \right]d^4 x.
 \label{eq:noet3}
\end{equation}

Debido al teorema de Stoke, nos damos cuenta que el segundo término de \mref{eq:noet3} 
se hace cero, quedando solo el primer término. Debido a que esta ecuación debe 
ser cierta para una $\epsilon$-familia arbitraria, cada expresión dentro de los corchetes 
den el integrando debe hacerse cero. Podemos escribir estas expresiones como 

\begin{equation}
 \frac{\partial}{\partial x^\alpha}\frac{\partial \mathcal{L}}{\partial \phi_{s,\alpha}} 
 - \frac{\partial \mathcal{L}}{\partial \phi_s} \equiv - \frac{\delta L}{\delta \phi_s} = 0.
\label{eq:noet4}
\end{equation}

Las ecuaciones \mref{eq:noet4} son las ecuaciones de Lagrange para sistemas continuos. 
Los índices pueden sustraerse de para simplificar la forma de estas expresiones, y 
utilizando esto llegamos a que podemos escribir las ecuaciones \mref{eq:noet4} como

\begin{equation}
 \nabla \frac{\partial \mathcal{L}}{\partial \nabla \phi} 
 - \frac{\partial \mathcal{L}}{\partial \phi} \equiv - \frac{\delta L}{\delta \phi} = 0.
 \label{eq:noet5}
\end{equation}

Y el segundo término de la ecuación \mref{eq:noet2} se convierte en 

\begin{equation}
 \nabla \cdot \left[ \frac{\partial \mathcal{L}}{\partial \nabla \phi}\right] \delta \phi.
 \label{eq:noet6}
\end{equation}

Entonces de acuerdo a las ecuaciones \mref{eq:noet2} y \mref{eq:noet6}, cuando 
se satisfacen las ecuaciones de campo $(\delta = \left\zerodel\frac{d}{d\epsilon}\right|_{\epsilon=0})$

\begin{equation}
 \delta \mathcal{L} =  \nabla \cdot \left[ \frac{\partial \mathcal{L}}{\partial \nabla \phi} \delta \phi\right].
 \label{eq:noet7}
\end{equation}

Por lo tanto, la invariancia de $\mathcal{L}$ sobre la $\epsilon$-familia implica 
que el lado derecho de \mref{eq:noet7} se hace cero. $\mathcal{L}$ no tiene que 
ser estrictamente invariante, esta pueda cambiar por una 4-divergencia. Entonces 
si existen cuatro funciones $\Phi^\alpha(\phi,x)$ tales que 

\begin{equation}
 \delta \mathcal{L} = \frac{\partial \Phi^\alpha}{\partial x^\alpha} \equiv \nabla \cdot \Phi,
 \label{eq:noet8}
\end{equation}

entonces 

\begin{equation}
 \nabla \cdot \left[ \frac{\partial \mathcal{L}}{\partial \nabla \phi} \delta \phi \right] 
 \equiv \nabla \cdot G \equiv \frac{\partial G^\alpha}{\partial \alpha} = 0,
 \label{eq:noet9}
\end{equation}

donde 

\begin{equation}
 \boxed{G^\alpha = \frac{\partial \mathcal{L}}{\partial \phi_{s,\alpha}}\delta \phi_s - 
 \Phi^\alpha.}
 \label{eq:noet10}
\end{equation}

Este resultado es análogo que al de mecánica de partículas, i.e. las cuatro funciones 
$G$ no son constantes del movimiento, pero $\partial G^\alpha / \partial \alpha = 0$. Este nuevo 
teorema requiere cuatro funciones $\Phi^\alpha$ que dependen de las $\phi_s$ y de las 
$x^\alpha$ pero no de $\phi_{s,\alpha}$. Aquí las $\partial/\partial \alpha$ son derivadas 
de la dependencia explícita de $x^\alpha$ de las $\Phi^\alpha$ y de la dependencia de 
$x^\alpha$ a través de $\phi_s$. 

\vspace{.3cm}

Este resultado es el teorema de Noether para teoría de campos, y para sistemas continuos 
en relación a la densidad lagrangiana $\mathcal{L}$. Nos dice que una cuasi-simetría 
de la densidad lagrangiana corresponde a una \emph{corriente conservada}, es decir, 
a un conjunto $G$ de cuatro funciones $G^\alpha$ que satisfacen la ecuación 
\mref{eq:noet9}: la divergencia de $G$ es cero.


\section{Problema 5}

Obtenga las ecuaciones de movimiento correspondientes a la densidad lagrangiana 

$$
\mathcal{L} = i\frac{a}{2} \left( \phi^*\dot{\phi} - \dot{\phi}^*\phi \right)  
- \frac{a^2}{2m}\nabla \phi^* \cdot \nabla \phi - \phi^*\phi V(x,y,z),
$$

donde el asterisco denota conjugación compleja, $a$ es una constante, $V$ un 
potencial e $i=\sqrt{-1}$. Comente sobre las ecuaciones que resulten.

\vspace{.3cm}

\underline{Solución:} \vspace{.3cm}

En el anterior problema derivamos las ecuaciones de Lagrange para sistemas continuos, 
y las pusimos en una forma muy compacta la cual nos servirá para resolver este problema. 
Recordando la ecuación \mref{eq:noet4}, vemos que podemos escribirla como, para 
la variable conjugada $\phi^*$,

\begin{equation}
 - \frac{\delta \mathcal{L}}{\delta \phi^*} \equiv 
 \left[ \nabla \cdot \frac{\partial \mathcal{L}}{\partial \nabla \phi^*} \right] 
 + \frac{\partial}{\partial t}\frac{\partial \mathcal{L}}{\partial \dot{\phi^*}} 
 - \frac{\partial \mathcal{L}}{\partial \phi^*} = 0,
 \label{eq:shro1}
\end{equation}

y para $\phi$ 

\begin{equation}
 - \frac{\delta\mathcal{L}}{\delta \phi} \equiv 
 \left[ \nabla \cdot \frac{\partial \mathcal{L}}{\partial \nabla \phi} \right] 
 + \frac{\partial}{\partial t}\frac{\partial \mathcal{L}}{\partial \dot{\phi}} 
 - \frac{\partial \mathcal{L}}{\partial \phi} = 0.
 \label{eq:shro2}
\end{equation}

Entonces aplicando \mref{eq:shro1} a la densidad lagrangiana obtenemos 

\begin{equation}
 \nabla \cdot \left[ - \frac{a^2}{2m}\nabla\phi \right] - 
 i\frac{a}{2}\frac{\partial}{\partial t} \phi + - i\frac{a}{2}\dot{\phi} + 
 \phi V(x,y,z) = 0,
\end{equation}

\begin{equation}
 -\frac{a}{2m}\nabla^2 \phi - ia\dot{\phi} + \phi V(x,y,z) = 0,
\end{equation}

\begin{equation}
 \boxed{ia\dot{\phi} = \left[-\frac{a}{2m}\nabla^2 + V(x,y,z)\right]\phi = 0.}
 \label{eq:shro3}
\end{equation}

Si en \mref{eq:shro3} hacemos el cambio $a \rightarrow \hbar$ y $\phi \rightarrow \psi$, 
resultando en

\begin{equation}
 \boxed{i\hbar\dot{\psi} = \left[-\frac{\hbar}{2m}\nabla^2 + V(x,y,z)\right]\psi = 0,}
 \label{eq:shro4}
\end{equation}

vemos que hemos encontrado la ecuación de Schrödinger lineal no relativista, y que 
por lo tanto la densidad lagrangiana del enunciado la podemos asociar con el 
campo de Schrödinger. La otra ecuación de Lagrange, para $\phi$, es la compleja 
conjugada de la ecuación obtenida que podemos escribir como, utilizando nuestra 
nueva notación que esclarece la ecuación resultante


\begin{equation}
 \boxed{i\hbar\dot{\psi^*} = \left[-\frac{\hbar}{2m}\nabla^2 + V(x,y,z)\right]\psi^* = 0.}
 \label{eq:shro5}
\end{equation}






\begin{thebibliography}{10}
\bibitem{noether}
E. Noether, \emph{Invariante Variationsprobleme}, Göttinger Nachirchten, pp. 235-275
(presentado por F. Klein en la conferencia del 26 de julio de 1918), 1918.
\bibitem{kosmann}
Y. Kosmann-Schwarzbach, \emph{The Noether Theorems: Invariance and Conservation Laws 
in the Twentieth Century}, Springer, 2011.


\end{thebibliography}


\end{document}