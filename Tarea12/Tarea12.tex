\documentclass[a4paper,10pt]{article}
\usepackage[utf8]{inputenc}
\usepackage[spanish]{babel}
\usepackage[affil-it]{authblk}
\usepackage{enumerate}
\usepackage{graphicx}
\usepackage{hyperref}
\usepackage{amsmath}
\usepackage{amssymb}
\usepackage{cancel}
\usepackage[usenames, dvipsnames]{color}
\usepackage{tikz}
\usepackage[labelfont=bf]{caption}
\usepackage{subcaption} %Multiple images
\usepackage{multicol} % Multiple columns
\usepackage{float}
\usepackage{cleveref}
 \usepackage{relsize} % bigger math symbols
\usepackage[margin=1.4in]{geometry}
\usepackage[titletoc,toc,title]{appendix}
\usepackage{enumitem}
\usepackage{etoolbox}
\usetikzlibrary{calc}
\numberwithin{equation}{section}

\graphicspath{Tarea10/}

% Circled words
\newcommand{\circled}[2][]{%
  \tikz[baseline=(char.base)]{%
    \node[shape = circle, draw, inner sep = 1pt]
    (char) {\phantom{\ifblank{#1}{#2}{#1}}};%
    \node at (char.center) {\makebox[0pt][c]{#2}};}}
\robustify{\circled}

%Appendices in spanish
\renewcommand{\appendixname}{Ap\'endices}
\renewcommand{\appendixtocname}{Ap\'endices}
\renewcommand{\appendixpagename}{Ap\'endices}

%Zero delimiter
\newcommand{\zerodel}{.\kern-\nulldelimiterspace}

%Columns separation
\setlength{\columnsep}{1cm}

%Indentation
\setlength{\parindent}{0ex}

%Multiple References

\crefrangelabelformat{equation}{(#3#1#4--#5\crefstripprefix{#1}{#2}#6)}

\usepackage{xparse}

%Boxes

\newcommand*{\boxcolor}{blue}
\makeatletter
\renewcommand{\boxed}[1]{\textcolor{\boxcolor}{%
\tikz[baseline={([yshift=-1ex]current bounding box.center)}] \node [rectangle, minimum width=1ex,rounded corners,draw] {\normalcolor\m@th$\displaystyle#1$};}}
 \makeatother

%Constantes
\newcommand{\euler}{\mathrm{e}}
\newcommand{\im}{i}

%Lemas, teoremas, definiciones y pruebas
\newcommand{\definicion}{\textbf{Definición: }}
\newcommand{\lema}{\textbf{Lema: }}
\newcommand{\teorema}{\textbf{Teorema: }}
\newcommand{\prueba}{\textbf{Prueba: }}
\newcommand{\proposicion}{\textbf{Proposición: }}
\newcommand{\corolario}{\textbf{Corolario: }}

% Definición de las secciones y su numeración

\makeatletter
\def\@seccntformat#1{%
  \expandafter\ifx\csname c@#1\endcsname\c@section\else
  \csname the#1\endcsname\quad
  \fi}
\makeatother

%opening
\title{Mecánica Clásica Tarea \# 12}
\author{Favio Vázquez\thanks{Correo: favio.vazquezp@gmail.com}}\affil{Instituto de Ciencias Nucleares. Universidad Nacional Autónoma de México.}
\date{}

\begin{document}

\makeatletter
\def\@maketitle{%
  \newpage
  \null
  \vskip 2em%
  \begin{center}%
  \let \footnote \thanks
    {\Large\bfseries \@title \par}%
    \vskip 1.5em%
    {\normalsize
      \lineskip .5em%
      \begin{tabular}[t]{c}%
        \@author
      \end{tabular}\par}%
    \vskip 1em%
    {\normalsize \@date}%
  \end{center}%
  \par
  \vskip 1.5em}
\makeatother

\maketitle

\section{Problema 1}

Demuestre que entre los paréntesis de Poisson y los de Lagrange existe la relación 
de inversión

$$
\sum_{\alpha=1}^{2n} [u_\alpha,u_\beta]\{u_\alpha,u_\gamma\} = \delta_{\beta\gamma},
$$

donde $u_\alpha(q^1,\dots,q^n,p_1,\dots,p_n)$, $\alpha = 1, \dots, 2n$ y $(q,p)$ es un 
sistema canónico de coordenadas.

\vspace{.3cm}

\underline{Solución:} \vspace{.3cm}

Para demostrar esto utilizaremos la definición de cada uno de los paréntesis y 
las sustituiremos directamente. El paréntesis de Lagrange para las coordenadas 
canónicas $(q,p)$ y aplicados al problema en cuestión se escribe como

\begin{equation}
 [u_\alpha,u_\beta] = \sum_i^n \left( \frac{\partial q^i}{\partial u_\alpha}
 \frac{\partial p_i}{\partial u_\beta}- \frac{\partial q^i}{\partial u_\beta}
 \frac{\partial p_i}{\partial u_\alpha}\right),
\end{equation}

y el paréntesis de Poisson se escribe como 

\begin{equation}
 \{u_\alpha, u_\gamma\} = \sum_j^n \left( \frac{\partial u_\alpha}{\partial q^j}
 \frac{\partial u_\gamma}{\partial p_j} - \frac{\partial u_\alpha}{\partial p_j}
 \frac{\partial u_\gamma}{\partial q^j} \right).
\end{equation}

Tenemos entonces que 

\begin{align*}
\sum_{\alpha=1}^{2n} [u_\alpha,u_\beta]\{u_\alpha,u_\gamma\} &= 
\sum_{\alpha=1}^{2n} \left[\sum_i^n \left( \frac{\partial q^i}{\partial u_\alpha}
 \frac{\partial p_i}{\partial u_\beta}- \frac{\partial q^i}{\partial u_\beta}
 \frac{\partial p_i}{\partial u_\alpha}\right) \right] \left[ \sum_j^n \left( \frac{\partial u_\alpha}{\partial q^j}
 \frac{\partial u_\gamma}{\partial p_j} - \frac{\partial u_\alpha}{\partial p_j}
 \frac{\partial u_\gamma}{\partial q^j} \right)\right] \\
 &= \sum_{\alpha=1}^{2n} \sum_i^n \sum_j^n \left( \frac{\partial q^i}{\partial u_\alpha}
 \frac{\partial p_i}{\partial u_\beta}- \frac{\partial q^i}{\partial u_\beta}
 \frac{\partial p_i}{\partial u_\alpha}\right)\left( \frac{\partial u_\alpha}{\partial q^j}
 \frac{\partial u_\gamma}{\partial p_j} - \frac{\partial u_\alpha}{\partial p_j}
 \frac{\partial u_\gamma}{\partial q^j} \right) \\
 &= \sum_{\alpha=1}^{2n} \sum_i^n \sum_j^n \left[\frac{\partial q^i}{\partial u_\alpha}
 \frac{\partial p_i}{\partial u_\beta} \frac{\partial u_\alpha}{\partial q^j}
 \frac{\partial u_\gamma}{\partial p_j} - \frac{\partial q^i}{\partial u_\alpha}
 \frac{\partial p_i}{\partial u_\beta}\frac{\partial u_\alpha}{\partial p_j}
 \frac{\partial u_\gamma}{\partial q^j}  \right\zerodel \\
 &\left\zerodel- \frac{\partial q^i}{\partial u_\beta}\frac{\partial p_i}{\partial u_\alpha}
 \frac{\partial u_\alpha}{\partial q^j}\frac{\partial u_\gamma}{\partial p_j} + 
 \frac{\partial q^i}{\partial u_\beta}\frac{\partial p_i}{\partial u_\alpha}
 \frac{\partial u_\alpha}{\partial p_j}\frac{\partial u_\gamma}{\partial q^j}\right].
\end{align*}

Ahora evaluando estos términos vemos que 

\begin{align}
 \sum_{\alpha=1}^{2n} \frac{\partial q^i}{\partial u_\alpha}
 \frac{\partial u_\alpha}{\partial q^j} = \delta^i_j, \\
  \sum_{\alpha=1}^{2n} \frac{\partial q^i}{\partial u_\alpha}
 \frac{\partial u_\alpha}{\partial p_j} = 0, \\
  \sum_{\alpha=1}^{2n} \frac{\partial p_i}{\partial u_\alpha}
 \frac{\partial u_\alpha}{\partial q^j} = 0, \\
  \sum_{\alpha=1}^{2n} \frac{\partial p_i}{\partial u_\alpha}
 \frac{\partial u_\alpha}{\partial p_jj} = \delta^i_j.
\end{align}

Donde $\delta^i_j$ es la delta de Kronecker. Entonces tenemos que 

\begin{align}
\begin{split}
 \sum_{\alpha=1}^{2n} [u_\alpha,u_\beta]\{u_\alpha,u_\gamma\} &= \sum_i^n \sum_j^n
 \left( \frac{\partial p_i}{\partial u_\beta}\frac{\partial u_\gamma}{\partial p_j}\delta^i_j
 + \frac{\partial q^i}{\partial u_\beta}\frac{\partial u_\gamma}{\partial q^j}\delta^i_j\right) \\
 &= \sum_i^n \left( \frac{\partial p_i}{\partial u_\beta}\frac{\partial u_\gamma}{\partial p_i}
 + \frac{\partial q^i}{\partial u_\beta}\frac{\partial u_\gamma}{\partial q^i}\right).
\end{split}
\end{align}

Haciendo ahora la sumatoria en $i$ vemos que esta expresión debe ser igual a la 
delta de Kronecker para $\beta$ y $\gamma$, por lo tanto 

\begin{equation}
 \boxed{\sum_{\alpha=1}^{2n} [u_\alpha,u_\beta]\{u_\alpha,u_\gamma\} = \delta_{\beta\gamma}.}
\end{equation}

Que era lo que se pidió demostrar, con lo cual vemos que los paréntesis de Poisson y 
los paréntesis de Lagrange forman matrices inversas la una para la otra.

\section{Problema 2}

Demuestre por tres vías distintas que las transformaciones 

\begin{align*}
 q^1 &= \frac{\sqrt{2P_1}\sen{Q^1}+P_2}{\sqrt{m\omega}} \\
 q^2 &= \frac{\sqrt{2P_1}\cos{Q^1}+Q^2}{\sqrt{m\omega}} \\
 p_1 &= \frac{\sqrt{m\omega}(\sqrt{2P_1}\cos{Q^1 - Q^2})}{2} \\
 p_2 &= \frac{\sqrt{m\omega}(\sqrt{2P_1}\sen{Q^1 - P_2})}{2}, 
\end{align*}

y

\begin{align*}
 q^1 &= Q^1\cos{\lambda} + \frac{P_2\sen{\lambda}}{m\omega} \\
 q^2 &= Q^2\cos{\lambda} + \frac{P_1\sen{\lambda}}{m\omega}\\
 p_1 &= - m\omega Q^2\sen{\lambda} + P_1 \cos{\lambda}\\
 p_2 &= - m\omega Q^1\sen{\lambda} + P_2 \cos{\lambda},
\end{align*}

son canónicas.

\vspace{.3cm}

\underline{Solución:} \vspace{.3cm}

\section{Problema 3}

Considere la función $F(q,p) = pq - \frac{1}{2m}p^2 t$ encuentre TODAS las transformaciones 
canónicas generadas por esta función.

\vspace{.3cm}

\underline{Solución:} \vspace{.3cm}

Para dar solución a este problema consideremos una transformación de 
coordenadas en el espacio extendido (dependiente del tiempo) 

\begin{align*}
 Q &= Q(q,p,t) \\
 P &= P(q,p,t) \\
 T &= T(q,p,t) = t.
\end{align*}

y su inversa 

\begin{align*}
 q &= q(Q,P,T) \\
 p &= p(Q,P,T) \\
 t &= y(Q,P,T) = t.
\end{align*}

que será canónica si al ser $(q,p,t)$ un sistema de coordenadas canónicas, $(Q,P,T)$ 
también lo es. Véase que el tiempo no se transforma $T = t$, lo cual indica que 
las coordenadas $(Q,P)$ son coordenadas canónicas para todo valor del tiempo. Recordemos 
que para las transformaciones canónicas dependientes del tiempo tenemos que la 
uno forma diferencial para las coordenadas $(q,p)$ se expresa como (para una dimensión)

\begin{equation}
 \omega = pdq - Hdt,
\end{equation}

donde $H(q,p,t)$ es la función hamiltoniana, y para las coordenadas $(Q,P,T)$ 

\begin{equation}
 \Omega = PdQ - H'dt
\end{equation}

donde $H'(q,p,t)$ es la nueva hamiltoniana. Ahora si la transformación es canónica 
debe cumplirse que las diferenciales de $\omega$ y $\Omega$ sean iguales, lo que 
quiere decir que la diferencia entre $\omega$ y $\Omega$ debe ser la diferencial 
de una función el espacio de fase, que es la que nos da el problema. Tenemos entonces 
que 

\begin{equation}
 pdq - Hdt - PdQ + H'dt = dF,
\end{equation}

diferenciando con respecto la tiempo tenemos 

\begin{equation}
 p\dot{q} - H - P\dot{Q} + H' = \dot{F}
\end{equation}

pero $Q = Q(q,f,t)$ y  $F = F(q,p,t)$ por lo tanto 

\begin{equation}
 p\dot{q} - H - P\left(\frac{\partial Q}{\partial q}\dot{q} 
 + \frac{\partial Q}{\partial p}\dot{p} + \frac{\partial Q}{\partial t} \right) + 
 H' = \frac{\partial F}{\partial q}\dot{q} 
 + \frac{\partial F}{\partial p}\dot{p} + \frac{\partial F}{\partial t},
\end{equation}

Para que esta igualdad se mantenga, deben ser iguales los coeficientes 
de $\dot{q}$ y $\dot{p}$, entonces para $\dot{q}$

\begin{equation}
 p - \frac{\partial Q}{\partial q}P = \frac{\partial F}{\partial q} = p,
\end{equation}

de esta expresión vemos que 

\begin{equation}
 \frac{\partial Q}{\partial q}= 0\quad \text{ó} \quad P = 0,
 \label{eq:Qcero}
\end{equation}

ahora para $\dot{p}$

\begin{equation}
 - \frac{\partial Q}{\partial p} P = \frac{\partial F}{\partial p} = q - \frac{pt}{m},
\label{eq:Pfinal}
\end{equation}

de donde vemos claramente que $P \ne 0$, por lo tanto de \eqref{eq:Qcero} vemos que 
obligatoriamente $\partial Q / \partial q = 0$, entonces necesariamente $Q$ es solo 
una función de $(p,t)$. Entonces de lo que acabamos de probar tenemos que 

\begin{equation}
  \boxed{Q = \xi(p,t),}
\end{equation}

con $\xi(p,t)$ arbitraria y de \eqref{eq:Pfinal} vemos que 

\begin{equation}
 \boxed{P = \frac{pt/m - q}{\partial \xi/\partial p}.}
\end{equation}

Para verificar que esta transformación es canónica entonces debe cumplirse que 
$\{Q,P\} = 1$, 

\begin{equation}
 \{Q,P\} = \cancelto{0}{\frac{\partial Q}{\partial q}}\frac{\partial P}{\partial p} - 
 \frac{\partial Q}{\partial p}\frac{\partial P}{\partial q} = 
 \left(\frac{\partial \xi}{\partial p}\right)\left(- \frac{1}{\frac{\xi}{\partial p}} \right) = 
 - \frac{\partial \xi}{\partial p}\frac{\partial p}{\partial \xi} = -1.
\end{equation}

Lo cual finaliza el problema. Claramente tenemos una gran cantidad de transformaciones 
canónicas que puede generar esta función generadora debido a que $\xi(p,t)$ es arbitraria.



\section{Problema 4}

Encuentre las funciones generadoras para las transformaciones canónicas del problema 
2.

\vspace{.3cm}

\underline{Solución:} \vspace{.3cm}

En el problema 2 demostramos que estas transformaciones son canónicas, ahora para 
encontrar las funciones generadoras para estas transformaciones canónicas debemos 
partir de la tabla de la página 50 de las notas, verificar que tipo ( o tipos) de 
función generadora es, y luego integrar las ecuaciones que resultan hasta encontrar
la (o las) funciones generadoras para cada transformación canónica. La primera 
transformación canónica es 

\begin{align*}
 q^1 &= \frac{\sqrt{2P_1}\sen{Q^1}+P_2}{\sqrt{m\omega}} \\
 q^2 &= \frac{\sqrt{2P_1}\cos{Q^1}+Q^2}{\sqrt{m\omega}} \\
 p_1 &= \frac{\sqrt{m\omega}(\sqrt{2P_1}\cos{Q^1 - Q^2})}{2} \\
 p_2 &= \frac{\sqrt{m\omega}(\sqrt{2P_1}\sen{Q^1 - P_2})}{2}, 
\end{align*}



Si asumimos que la función generadora es de tipo 1 entonces debe cumplirse que 
$F_3 = F_3(p,Q)$, y que 

\begin{align*}
 q^1 = - \frac{\partial F_3}{\partial p_1}, \\
 q^2 = - \frac{\partial F_3}{\partial p_2}, \\
 P_1 = - \frac{\partial F_3}{\partial Q^1}, \\
 P_2 = - \frac{\partial F_3}{\partial Q^2}.
\end{align*}

Tenemos entonces que 

\begin{equation}
 q^1 = \frac{\sqrt{2P_1}\sen{Q^1}+P_2}{\sqrt{m\omega}} = 
 \frac{\partial F_3}{\partial p_1},
\end{equation}

\begin{equation}
 q^2 = \frac{\sqrt{2P_1}\cos{Q^1}+Q^2}{\sqrt{m\omega}}  = 
 \frac{\partial F_3}{\partial p_2},
\end{equation}

\begin{equation}
 P_1 = \frac{1}{2}\sqrt{\frac{q^1\sqrt{m\omega}-P_2}{\sen{Q^1}}} = 
 \frac{\partial F_3}{\partial Q^1}
\end{equation}

\begin{equation}
 P_2 = \sqrt{2P_1}\sen{Q^1} + q^1\sqrt{m\omega} = 
 \frac{\partial F_3}{\partial Q^2}.
\end{equation}

De la primera ecuación vemos que 

\begin{equation}
 F_3 = \frac{p_1(\sqrt{2P_1}\sen{Q^1}+P_2)}{\sqrt{m\omega}} + G(Q^1,P_2,P_1),
\end{equation}

de la segunda ecuación vemos que 

\begin{equation}
 F_3 = \frac{p_2(\sqrt{2P_1}\cos{Q^1}+Q^2)}{\sqrt{m\omega}} + G(Q^1,Q^2,P_1),
\end{equation}

% TERMINAR 

\vspace{.3cm}

Calculemos la función generadora ahora para la segunda transformación, 

\begin{align*}
 q^1 &= Q^1\cos{\lambda} + \frac{P_2\sen{\lambda}}{m\omega} \\
 q^2 &= Q^2\cos{\lambda} + \frac{P_1\sen{\lambda}}{m\omega}\\
 p_1 &= - m\omega Q^2\sen{\lambda} + P_1 \cos{\lambda}\\
 p_2 &= - m\omega Q^1\sen{\lambda} + P_2 \cos{\lambda},
\end{align*}

Proponemos que la función generadora de esta transformación canónica es de tipo 
$F_4 = F_4(p,P)$, es decir que son independientes las $p_i$ de las $P_1$, esto podemos 
verlo ya que podemos escribir a las otras variables en términos de $p_1,p_2,P_1,P_2$, 

\begin{align*}
 q^1 &= \left( \frac{p_2 - P_2\cos{\lambda}}{-m\omega}\right)\cos{\lambda} 
 + \frac{P_2\sen{\lambda}}{m\omega} \\
 q^2 &= \left(\frac{p_1 - P_1\cos{\lambda}}{-m\omega}\right)\cos{\lambda} 
 + \frac{P_1\sen{\lambda}}{m\omega} \\
 Q^1 &= \frac{P_2\cos{\lambda} - p_2}{m\omega\sen{\lambda}}\\
 Q^2 &= \frac{P_1\cos{\lambda} - p_1}{m\omega\sen{\lambda}}
\end{align*}

Proponemos entonces que 

\begin{equation}
 dF_4(p_1,p_2,P_1,P_2) = -q^1dp_1 - q^2dp_2 + Q^1dP_1 + Q^2dP_2 
\end{equation}

\begin{align}
 \begin{split}
 dF_4(p_1,p_2,P_1,P_2) = &- \left[ \left( \frac{p_2 - P_2\cos{\lambda}}{-m\omega}\right)\cos{\lambda} 
 + \frac{P_2\sen{\lambda}}{m\omega}\right]dp_1 \\
 &- \left[ \left(\frac{p_1 - P_1\cos{\lambda}}{-m\omega}\right)\cos{\lambda} 
 + \frac{P_1\sen{\lambda}}{m\omega}\right]dp_2 \\
 &+ \left(\frac{P_2\cos{\lambda} - p_2}{m\omega\sen{\lambda}} \right)dP_1 
 + \left(\frac{P_1\cos{\lambda} - p_1}{m\omega\sen{\lambda}} \right)dP_2 \\
 &= - \left[\frac{p_2\cos{\lambda} - P_2\cos^2{\lambda}}{-m\omega} 
 + \frac{P_2\sen{\lambda}}{m\omega}\right]dp_1 \\
 &- \left[\frac{p_1\cos{\lambda}  - P_1\cos^2{\lambda}}{-m\omega}
 + \frac{P_1\sen{\lambda}}{m\omega}\right]dp_2 \\
 &+ \left(\frac{P_2\cos{\lambda} - p_2}{m\omega\sen{\lambda}} \right)dP_1 
 + \left(\frac{P_1\cos{\lambda} - p_1}{m\omega\sen{\lambda}} \right)dP_2 \\
 &= \left[\frac{p_2\cos{\lambda} - P_2\cos^2{\lambda}}{m\omega} 
 - \frac{P_2\sen{\lambda}}{m\omega}\right]dp_1 \\
 & \left[\frac{p_1\cos{\lambda}  - P_1\cos^2{\lambda}}{m\omega}
 - \frac{P_1\sen{\lambda}}{m\omega}\right]dp_2 \\
 &+ \left(\frac{P_2\cos{\lambda} - p_2}{m\omega\sen{\lambda}} \right)dP_1 
 + \left(\frac{P_1\cos{\lambda} - p_1}{m\omega\sen{\lambda}} \right)dP_2 \\
 &= \frac{p_2\cos{\lambda}dp_1}{m\omega} - \frac{P_2\cos^2{\lambda}dp_1}{m\omega}
 - \frac{P_2\sen{\lambda}dp_1}{m\omega} \\
 &+ \frac{p_1\cos{\lambda}dp_2}{m\omega} - \frac{P_1\cos^2{\lambda}dp_2}{m\omega}
 - \frac{P_1\sen{\lambda}dp_2}{m\omega} \\
 &+ \frac{P_2\cos{\lambda}dP_1}{m\omega\sen{\lambda}} - 
 \frac{p_2dP_1}{m\omega\sen{\lambda}} \\
 &+ \frac{P_1\cos{\lambda}dP_2}{m\omega\sen{\lambda}} - 
 \frac{p_1dP_2}{m\omega\sen{\lambda}}
 \end{split}
\end{align}

\begin{equation*}
 \therefore dF_4(p_1,p_2,P_1,P_2) = d\left(\frac{\cos{\lambda}}{m\omega}p_1p_2 + 
 \frac{\cos{\lambda}}{m\omega\sen{\lambda}}P_1P_2 -
 \frac{1}{m\omega\sen{\lambda}}p_1P_2 - \frac{\cos^2{\lambda}}{m\omega}p_2P1 \right),
\end{equation*}

debido a que esta es una diferencial exacta entonces tenemos que 

\begin{equation}
 \boxed{F_4(p_1,p_2,P_1,P_2) = \frac{\cos{\lambda}}{m\omega}p_1p_2 + 
 \frac{\cos{\lambda}}{m\omega\sen{\lambda}}P_1P_2 -
 \frac{1}{m\omega\sen{\lambda}}p_1P_2 - \frac{\cos^2{\lambda}}{m\omega}p_2P1}
\end{equation}






\section{Problema 5}

Considere una de las regiones acotadas (aquellas donde las trayectorias son regulares 
y acotadas) del espacio de fase de un péndulo simple, ¿será posible encontrar una 
transformación canónica de coordenadas de tal forma que, en las nuevas coordenadas, 
tengamos una coordenada ignorable?, note que esto es similar al caso del oscilador 
armónico que estudiamos como ejemplo. En caso de una respuesta afirmativa calcule 
esta transformación (puede dejar algunas integrales indicadas) ¿Será posible 
esto en las tres regiones acotadas? ¿Será posible esto de manera global?, esto es,
una sola transformación para todo el espacio de fase (excepto por algunos puntos o 
lineas).

\vspace{.3cm}

\underline{Solución:} \vspace{.3cm}

\section{Problema 6}

Al hacer una transformación de punto dependiente del tiempo la hamiltoniana 
debe cambiarse por medio de 

$$
H' = H + \sum_j p_j \frac{\partial q^j}{\partial t},
$$

ver fórmula (189) de las notas.

\vspace{.3cm}

Al aplicar una transformación canónica de coordenadas dependiente del tiempo la 
hamiltoniana debe cambiar por 

$$
H' = H + \frac{\partial F}{\partial t}
$$

donde $F$ es la función generadora de la transformación. 

\vspace{.3cm}

Encuentre la relación entre estos dos resultados. Así mimos encuentre la relación 
entre la fórmula para la integral de movimiento ante una simetría que se vio en el 
contexto de la formulación lagrangiana y los generadores infinitesimales del grupo 
de dicha simetría.

\vspace{.3cm}

\underline{Solución:} \vspace{.3cm}

\section{Problema 7}

Cuando la hamiltoniana no depende explícitamente del tiempo las fórmulas para el 
campo vectorial hamiltoniano $\gamma[\bullet,V_H] = dH$, en el espacio de fase, 
y $\Gamma[V_H,\bullet] = 0$ en el espacio de fase extendido, son equivalentes. 
Demuestre esto de manera global, esto es, sin utilizar coordenadas.

\vspace{.3cm}

\underline{Solución:} \vspace{.3cm}

\end{document}